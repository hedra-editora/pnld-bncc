\BNCC{EF06LP01}
 %Reconhecer a impossibilidade de uma neutralidade absoluta no relato de fatos e identificar diferentes graus de parcialidade/ imparcialidade dados pelo recorte feito e pelos efeitos de sentido advindos de escolhas feitas pelo autor, de forma a poder desenvolver uma atitude crítica frente aos textos jornalísticos e tornar-se consciente das escolhas feitas enquanto produtor de textos.
\BNCC{EF06LP02}
 %Estabelecer relação entre os diferentes gêneros jornalísticos, compreendendo a centralidade da notícia.
\BNCC{EF06LP03}
 %Analisar diferenças de sentido entre palavras de uma série sinonímica.
\BNCC{EF06LP04}
 %Analisar a função e as flexões de substantivos e adjetivos e de verbos nos modos Indicativo, Subjuntivo e Imperativo: afirmativo e negativo.
\BNCC{EF06LP05}
 %Identificar os efeitos de sentido dos modos verbais, considerando o gênero textual e a intenção comunicativa.
\BNCC{EF06LP06}
 %Empregar, adequadamente, as regras de concordância nominal (relações entre os substantivos e seus determinantes) e as regras de concordância verbal (relações entre o verbo e o sujeito simples e composto).
\BNCC{EF06LP07}
 %Identificar, em textos, períodos compostos por orações separadas por vírgula sem a utilização de conectivos, nomeando-os como períodos compostos por coordenação.
\BNCC{EF06LP08}
 %Identificar, em texto ou sequência textual, orações como unidades constituídas em torno de um núcleo verbal e períodos como conjunto de orações conectadas.
\BNCC{EF06LP09}
 %Classificar, em texto ou sequência textual, os períodos simples compostos.
\BNCC{EF06LP10}
 %Identificar sintagmas nominais e verbais como constituintes imediatos da oração.
\BNCC{EF06LP11}
 %Utilizar, ao produzir texto, conhecimentos linguísticos e gramaticais: tempos verbais, concordância nominal e verbal, regras ortográficas, pontuação etc.
\BNCC{EF06LP12}
 %Utilizar, ao produzir texto, recursos de coesão referencial (nome e pronomes), recursos semânticos de sinonímia, antonímia e homonímia e mecanismos de representação de diferentes vozes (discurso direto e indireto).
\BNCC{EF67LP01}
 %Analisar a estrutura e funcionamento dos hiperlinks em textos noticiosos publicados na Web e vislumbrar possibilidades de uma escrita hipertextual.
\BNCC{EF67LP02}
 %Explorar o espaço reservado ao leitor nos jornais, revistas, impressos e on-line, sites noticiosos etc., destacando notícias, fotorreportagens, entrevistas, charges, assuntos, temas, debates em foco, posicionando-se de maneira ética e respeitosa frente a esses textos e opiniões a eles relacionadas, e publicar notícias, notas jornalísticas, fotorreportagem de interesse geral nesses espaços do leitor.
\BNCC{EF67LP03}
 %Comparar informações sobre um mesmo fato divulgadas em diferentes veículos e mídias, analisando e avaliando a confiabilidade.
\BNCC{EF67LP04}
 %Distinguir, em segmentos descontínuos de textos, fato da opinião enunciada em relação a esse mesmo fato.
\BNCC{EF67LP05}
 %Identificar e avaliar teses/opiniões/posicionamentos explícitos e argumentos em textos argumentativos (carta de leitor, comentário, artigo de opinião, resenha crítica etc.), manifestando concordância ou discordância.
\BNCC{EF67LP06}
 %Identificar os efeitos de sentido provocados pela seleção lexical, topicalização de elementos e seleção e hierarquização de informações, uso de 3ª pessoa etc.
\BNCC{EF67LP07}
 %Identificar o uso de recursos persuasivos em textos argumentativos diversos (como a elaboração do título, escolhas lexicais, construções metafóricas, a explicitação ou a ocultação de fontes de informação) e perceber seus efeitos de sentido.
\BNCC{EF67LP08}
 %Identificar os efeitos de sentido devidos à escolha de imagens estáticas, sequenciação ou sobreposição de imagens, definição de figura/fundo, ângulo, profundidade e foco, cores/tonalidades, relação com o escrito (relações de reiteração, complementação ou oposição) etc. em notícias, reportagens, fotorreportagens, foto-denúncias, memes, gifs, anúncios publicitários e propagandas publicados em jornais, revistas, sites na internet etc.
\BNCC{EF67LP09}
 %Planejar notícia impressa e para circulação em outras mídias (rádio ou TV/vídeo), tendo em vista as condições de produção, do texto – objetivo, leitores/espectadores, veículos e mídia de circulação etc. –, a partir da escolha do fato a ser noticiado (de relevância para a turma, escola ou comunidade), do levantamento de dados e informações sobre o fato – que pode envolver entrevistas com envolvidos ou com especialistas, consultas a fontes, análise de documentos, cobertura de eventos etc.–, do registro dessas informações e dados, da escolha de fotos ou imagens a produzir ou a utilizar etc. e a previsão de uma estrutura hipertextual (no caso de publicação em sites ou blogs noticiosos).
\BNCC{EF67LP10}
 %Produzir notícia impressa tendo em vista características do gênero – título ou manchete com verbo no tempo presente, linha fina (opcional), lide, progressão dada pela ordem decrescente de importância dos fatos, uso de 3ª pessoa, de palavras que indicam precisão –, e o estabelecimento adequado de coesão e produzir notícia para TV, rádio e internet, tendo em vista, além das características do gênero, os recursos de mídias disponíveis e o manejo de recursos de captação e edição de áudio e imagem.
\BNCC{EF67LP11}
 %Planejar resenhas, vlogs, vídeos e podcasts variados, e textos e vídeos de apresentação e apreciação próprios das culturas juvenis (algumas possibilidades: fanzines, fanclipes, e-zines, gameplay, detonado etc.), dentre outros, tendo em vista as condições de produção do texto – objetivo, leitores/espectadores, veículos e mídia de circulação etc. –, a partir da escolha de uma produção ou evento cultural para analisar – livro, filme, série, game, canção, videoclipe, fanclipe, show, saraus, slams etc. – da busca de informação sobre a produção ou evento escolhido, da síntese de informações sobre a obra/evento e do elenco/seleção de aspectos, elementos ou recursos que possam ser destacados positiva ou negativamente ou da roteirização do passo a passo do game para posterior gravação dos vídeos.
\BNCC{EF67LP12}
 %Produzir resenhas críticas, vlogs, vídeos, podcasts variados e produções e gêneros próprios das culturas juvenis (algumas possibilidades: fanzines, fanclipes, e-zines, gameplay, detonado etc.), que apresentem/descrevam e/ou avaliem produções culturais (livro, filme, série, game, canção, disco, videoclipe etc.) ou evento (show, sarau, slam etc.), tendo em vista o contexto de produção dado, as características do gênero, os recursos das mídias envolvidas e a textualização adequada dos textos e/ou produções.
\BNCC{EF67LP13}
 %Produzir, revisar e editar textos publicitários, levando em conta o contexto de produção dado, explorando recursos multissemióticos, relacionando elementos verbais e visuais, utilizando adequadamente estratégias discursivas de persuasão e/ou convencimento e criando título ou slogan que façam o leitor motivar-se a interagir com o texto produzido e se sinta atraído pelo serviço, ideia ou produto em questão.
\BNCC{EF67LP14}
 %Definir o contexto de produção da entrevista (objetivos, o que se pretende conseguir, porque aquele entrevistado etc.), levantar informações sobre o entrevistado e sobre o acontecimento ou tema em questão, preparar o roteiro de perguntar e realizar entrevista oral com envolvidos ou especialistas relacionados com o fato noticiado ou com o tema em pauta, usando roteiro previamente elaborado e formulando outras perguntas a partir das respostas dadas e, quando for o caso, selecionar partes, transcrever e proceder a uma edição escrita do texto, adequando-o a seu contexto de publicação, à construção composicional do gênero e garantindo a relevância das informações mantidas e a continuidade temática.
\BNCC{EF67LP15}
 %Identificar a proibição imposta ou o direito garantido, bem como as circunstâncias de sua aplicação, em artigos relativos a normas, regimentos escolares, regimentos e estatutos da sociedade civil, regulamentações para o mercado publicitário, Código de Defesa do Consumidor, Código Nacional de Trânsito, ECA, Constituição, dentre outros.
\BNCC{EF67LP16}
 %Explorar e analisar espaços de reclamação de direitos e de envio de solicitações (tais como ouvidorias, SAC, canais ligados a órgãos públicos, plataformas do consumidor, plataformas de reclamação), bem como de textos pertencentes a gêneros que circulam nesses espaços, reclamação ou carta de reclamação, solicitação ou carta de solicitação, como forma de ampliar as possibilidades de produção desses textos em casos que remetam a reivindicações que envolvam a escola, a comunidade ou algum de seus membros como forma de se engajar na busca de solução de problemas pessoais, dos outros e coletivos.
\BNCC{EF67LP17}
 %Analisar, a partir do contexto de produção, a forma de organização das cartas de solicitação e de reclamação (datação, forma de início, apresentação contextualizada do pedido ou da reclamação, em geral, acompanhada de explicações, argumentos e/ou relatos do problema, fórmula de finalização mais ou menos cordata, dependendo do tipo de carta e subscrição) e algumas das marcas linguísticas relacionadas à argumentação, explicação ou relato de fatos, como forma de possibilitar a escrita fundamentada de cartas como essas ou de postagens em canais próprios de reclamações e solicitações em situações que envolvam questões relativas à escola, à comunidade ou a algum dos seus membros.
\BNCC{EF67LP18}
 %Identificar o objeto da reclamação e/ou da solicitação e sua sustentação, explicação ou justificativa, de forma a poder analisar a pertinência da solicitação ou justificação.
\BNCC{EF67LP19}
 %Realizar levantamento de questões, problemas que requeiram a denúncia de desrespeito a direitos, reivindicações, reclamações, solicitações que contemplem a comunidade escolar ou algum de seus membros e examinar normas e legislações.
\BNCC{EF67LP20}
 %Realizar pesquisa, a partir de recortes e questões definidos previamente, usando fontes indicadas e abertas.
\BNCC{EF67LP21}
 %Divulgar resultados de pesquisas por meio de apresentações orais, painéis, artigos de divulgação científica, verbetes de enciclopédia, podcasts científicos etc.
\BNCC{EF67LP22}
 %Produzir resumos, a partir das notas e/ou esquemas feitos, com o uso adequado de paráfrases e citações.
\BNCC{EF67LP23}
 %Respeitar os turnos de fala, na participação em conversações e em discussões ou atividades coletivas, na sala de aula e na escola e formular perguntas coerentes e adequadas em momentos oportunos em situações de aulas, apresentação oral, seminário etc.
\BNCC{EF67LP24}
 %Tomar nota de aulas, apresentações orais, entrevistas (ao vivo, áudio, TV, vídeo), identificando e hierarquizando as informações principais, tendo em vista apoiar o estudo e a produção de sínteses e reflexões pessoais ou outros objetivos em questão.
\BNCC{EF67LP25}
 %Reconhecer e utilizar os critérios de organização tópica (do geral para o específico, do específico para o geral etc.), as marcas linguísticas dessa organização (marcadores de ordenação e enumeração, de explicação, definição e exemplificação, por exemplo) e os mecanismos de paráfrase, de maneira a organizar mais adequadamente a coesão e a progressão temática de seus textos.
\BNCC{EF67LP26}
 %Reconhecer a estrutura de hipertexto em textos de divulgação científica e proceder à remissão a conceitos e relações por meio de notas de rodapés ou boxes.
\BNCC{EF67LP27}
 %Analisar, entre os textos literários e entre estes e outras manifestações artísticas (como cinema, teatro, música, artes visuais e midiáticas), referências explícitas ou implícitas a outros textos, quanto aos temas, personagens e recursos literários e semióticos
\BNCC{EF67LP28}
 %Ler, de forma autônoma, e compreender – selecionando procedimentos e estratégias de leitura adequados a diferentes objetivos e levando em conta características dos gêneros e suportes –, romances infantojuvenis, contos populares, contos de terror, lendas brasileiras, indígenas e africanas, narrativas de aventuras, narrativas de enigma, mitos, crônicas, autobiografias, histórias em quadrinhos, mangás, poemas de forma livre e fixa (como sonetos e cordéis), vídeo-poemas, poemas visuais, dentre outros, expressando avaliação sobre o texto lido e estabelecendo preferências por gêneros, temas, autores.
\BNCC{EF67LP29}
 %Identificar, em texto dramático, personagem, ato, cena, fala e indicações cênicas e a organização do texto: enredo, conflitos, ideias principais, pontos de vista, universos de referência.
\BNCC{EF67LP30}
 %Criar narrativas ficcionais, tais como contos populares, contos de suspense, mistério, terror, humor, narrativas de enigma, crônicas, histórias em quadrinhos, dentre outros, que utilizem cenários e personagens realistas ou de fantasia, observando os elementos da estrutura narrativa próprios ao gênero pretendido, tais como enredo, personagens, tempo, espaço e narrador, utilizando tempos verbais adequados à narração de fatos passados, empregando conhecimentos sobre diferentes modos de se iniciar uma história e de inserir os discursos direto e indireto.
\BNCC{EF67LP31}
 %Criar poemas compostos por versos livres e de forma fixa (como quadras e sonetos), utilizando recursos visuais, semânticos e sonoros, tais como cadências, ritmos e rimas, e poemas visuais e vídeo-poemas, explorando as relações entre imagem e texto verbal, a distribuição da mancha gráfica (poema visual) e outros recursos visuais e sonoros.
\BNCC{EF67LP32}
 %Escrever palavras com correção ortográfica, obedecendo as convenções da língua escrita.
\BNCC{EF67LP33}
 %Pontuar textos adequadamente.
\BNCC{EF67LP34}
 %Formar antônimos com acréscimo de prefixos que expressam noção de negação.
\BNCC{EF67LP35}
 %Distinguir palavras derivadas por acréscimo de afixos e palavras compostas.
\BNCC{EF67LP36}
 %Utilizar, ao produzir texto, recursos de coesão referencial (léxica e pronominal) e sequencial e outros recursos expressivos adequados ao gênero textual.
\BNCC{EF67LP37}
 %Analisar, em diferentes textos, os efeitos de sentido decorrentes do uso de recursos linguístico-discursivos de prescrição, causalidade, sequências descritivas e expositivas e ordenação de eventos.
\BNCC{EF67LP38}
 %Analisar os efeitos de sentido do uso de figuras de linguagem, como comparação, metáfora, metonímia, personificação, hipérbole, dentre outras.
\BNCC{EF69LP01}
 %Diferenciar liberdade de expressão de discursos de ódio, posicionando-se contrariamente a esse tipo de discurso e vislumbrando possibilidades de denúncia quando for o caso.
\BNCC{EF69LP02}
 %Analisar e comparar peças publicitárias variadas (cartazes, folhetos, outdoor, anúncios e propagandas em diferentes mídias, spots, jingle, vídeos etc.), de forma a perceber a articulação entre elas em campanhas, as especificidades das várias semioses e mídias, a adequação dessas peças ao público-alvo, aos objetivos do anunciante e/ou da campanha e à construção composicional e estilo dos gêneros em questão, como forma de ampliar suas possibilidades de compreensão (e produção) de textos pertencentes a esses gêneros.
\BNCC{EF69LP03}
 %Identificar, em notícias, o fato central, suas principais circunstâncias e eventuais decorrências; em reportagens e fotorreportagens o fato ou a temática retratada e a perspectiva de abordagem, em entrevistas os principais temas/subtemas abordados, explicações dadas ou teses defendidas em relação a esses subtemas; em tirinhas, memes, charge, a crítica, ironia ou humor presente.
\BNCC{EF69LP04}
 %Identificar e analisar os efeitos de sentido que fortalecem a persuasão nos textos publicitários, relacionando as estratégias de persuasão e apelo ao consumo com os recursos linguístico-discursivos utilizados, como imagens, tempo verbal, jogos de palavras, figuras de linguagem etc., com vistas a fomentar práticas de consumo conscientes.
\BNCC{EF69LP05}
 %Inferir e justificar, em textos multissemióticos – tirinhas, charges, memes, gifs etc. –, o efeito de humor, ironia e/ou crítica pelo uso ambíguo de palavras, expressões ou imagens ambíguas, de clichês, de recursos iconográficos, de pontuação etc.
\BNCC{EF69LP06}
 %Produzir e publicar notícias, fotodenúncias, fotorreportagens, reportagens, reportagens multimidiáticas, infográficos, podcasts noticiosos, entrevistas, cartas de leitor, comentários, artigos de opinião de interesse local ou global, textos de apresentação e apreciação de produção cultural – resenhas e outros próprios das formas de expressão das culturas juvenis, tais como vlogs e podcasts culturais, gameplay, detonado etc.– e cartazes, anúncios, propagandas, spots, jingles de campanhas sociais, dentre outros em várias mídias, vivenciando de forma significativa o papel de repórter, de comentador, de analista, de crítico, de editor ou articulista, de booktuber, de vlogger (vlogueiro) etc., como forma de compreender as condições de produção que envolvem a circulação desses textos e poder participar e vislumbrar possibilidades de participação nas práticas de linguagem do campo jornalístico e do campo midiático de forma ética e responsável, levando-se em consideração o contexto da Web 2.0, que amplia a possibilidade de circulação desses textos e “funde” os papéis de leitor e autor, de consumidor e produtor.
\BNCC{EF69LP07}
 %Produzir textos em diferentes gêneros, considerando sua adequação ao contexto produção e circulação – os enunciadores envolvidos, os objetivos, o gênero, o suporte, a circulação -, ao modo (escrito ou oral; imagem estática ou em movimento etc.), à variedade linguística e/ou semiótica apropriada a esse contexto, à construção da textualidade relacionada às propriedades textuais e do gênero), utilizando estratégias de planejamento, elaboração, revisão, edição, reescrita/redesign e avaliação de textos, para, com a ajuda do professor e a colaboração dos colegas, corrigir e aprimorar as produções realizadas, fazendo cortes, acréscimos, reformulações, correções de concordância, ortografia, pontuação em textos e editando imagens, arquivos sonoros, fazendo cortes, acréscimos, ajustes, acrescentando/ alterando efeitos, ordenamentos etc.
\BNCC{EF69LP08}
 %Revisar/editar o texto produzido – notícia, reportagem, resenha, artigo de opinião, dentre outros –, tendo em vista sua adequação ao contexto de produção, a mídia em questão, características do gênero, aspectos relativos à textualidade, a relação entre as diferentes semioses, a formatação e uso adequado das ferramentas de edição (de texto, foto, áudio e vídeo, dependendo do caso) e adequação à norma culta.
\BNCC{EF69LP09}
 %Planejar uma campanha publicitária sobre questões/problemas, temas, causas significativas para a escola e/ou comunidade, a partir de um levantamento de material sobre o tema ou evento, da definição do público-alvo, do texto ou peça a ser produzido – cartaz, banner, folheto, panfleto, anúncio impresso e para internet, spot, propaganda de rádio, TV etc. –, da ferramenta de edição de texto, áudio ou vídeo que será utilizada, do recorte e enfoque a ser dado, das estratégias de persuasão que serão utilizadas etc.
\BNCC{EF69LP10}
 %Produzir notícias para rádios, TV ou vídeos, podcasts noticiosos e de opinião, entrevistas, comentários, vlogs, jornais radiofônicos e televisivos, dentre outros possíveis, relativos a fato e temas de interesse pessoal, local ou global e textos orais de apreciação e opinião – podcasts e vlogs noticiosos, culturais e de opinião, orientando-se por roteiro ou texto, considerando o contexto de produção e demonstrando domínio dos gêneros.
\BNCC{EF69LP11}
 %Identificar e analisar posicionamentos defendidos e refutados na escuta de interações polêmicas em entrevistas, discussões e debates (televisivo, em sala de aula, em redes sociais etc.), entre outros, e se posicionar frente a eles.
\BNCC{EF69LP12}
 %Desenvolver estratégias de planejamento, elaboração, revisão, edição, reescrita/ redesign (esses três últimos quando não for situação ao vivo) e avaliação de textos orais, áudio e/ou vídeo, considerando sua adequação aos contextos em que foram produzidos, à forma composicional e estilo de gêneros, a clareza, progressão temática e variedade linguística empregada, os elementos relacionados à fala, tais como modulação de voz, entonação, ritmo, altura e intensidade, respiração etc., os elementos cinésicos, tais como postura corporal, movimentos e gestualidade significativa, expressão facial, contato de olho com plateia etc.
\BNCC{EF69LP13}
 %Engajar-se e contribuir com a busca de conclusões comuns relativas a problemas, temas ou questões polêmicas de interesse da turma e/ou de relevância social.
\BNCC{EF69LP14}
 %Formular perguntas e decompor, com a ajuda dos colegas e dos professores, tema/questão polêmica, explicações e ou argumentos relativos ao objeto de discussão para análise mais minuciosa e buscar em fontes diversas informações ou dados que permitam analisar partes da questão e compartilhá-los com a turma.
\BNCC{EF69LP15}
 %Apresentar argumentos e contra-argumentos coerentes, respeitando os turnos de fala, na participação em discussões sobre temas controversos e/ou polêmicos.
\BNCC{EF69LP16}
 %Analisar e utilizar as formas de composição dos gêneros jornalísticos da ordem do relatar, tais como notícias (pirâmide invertida no impresso X blocos noticiosos hipertextuais e hipermidiáticos no digital, que também pode contar com imagens de vários tipos, vídeos, gravações de áudio etc.), da ordem do argumentar, tais como artigos de opinião e editorial (contextualização, defesa de tese/opinião e uso de argumentos) e das entrevistas: apresentação e contextualização do entrevistado e do tema, estrutura pergunta e resposta etc.
\BNCC{EF69LP17}
 %Perceber e analisar os recursos estilísticos e semióticos dos gêneros jornalísticos e publicitários, os aspectos relativos ao tratamento da informação em notícias, como a ordenação dos eventos, as escolhas lexicais, o efeito de imparcialidade do relato, a morfologia do verbo, em textos noticiosos e argumentativos, reconhecendo marcas de pessoa, número, tempo, modo, a distribuição dos verbos nos gêneros textuais (por exemplo, as formas de pretérito em relatos; as formas de presente e futuro em gêneros argumentativos; as formas de imperativo em gêneros publicitários), o uso de recursos persuasivos em textos argumentativos diversos (como a elaboração do título, escolhas lexicais, construções metafóricas, a explicitação ou a ocultação de fontes de informação) e as estratégias de persuasão e apelo ao consumo com os recursos linguístico-discursivos utilizados (tempo verbal, jogos de palavras, metáforas, imagens).
\BNCC{EF69LP18}
 %Utilizar, na escrita/reescrita de textos argumentativos, recursos linguísticos que marquem as relações de sentido entre parágrafos e enunciados do texto e operadores de conexão adequados aos tipos de argumento e à forma de composição de textos argumentativos, de maneira a garantir a coesão, a coerência e a progressão temática nesses textos (“primeiramente, mas, no entanto, em primeiro/segundo/terceiro lugar, finalmente, em conclusão” etc.).
\BNCC{EF69LP19}
 %Analisar, em gêneros orais que envolvam argumentação, os efeitos de sentido de elementos típicos da modalidade falada, como a pausa, a entonação, o ritmo, a gestualidade e expressão facial, as hesitações etc.
\BNCC{EF69LP20}
 %Identificar, tendo em vista o contexto de produção, a forma de organização dos textos normativos e legais, a lógica de hierarquização de seus itens e subitens e suas partes: parte inicial (título – nome e data – e ementa), blocos de artigos (parte, livro, capítulo, seção, subseção), artigos (caput e parágrafos e incisos) e parte final (disposições pertinentes à sua implementação) e analisar efeitos de sentido causados pelo uso de vocabulário técnico, pelo uso do imperativo, de palavras e expressões que indicam circunstâncias, como advérbios e locuções adverbiais, de palavras que indicam generalidade, como alguns pronomes indefinidos, de forma a poder compreender o caráter imperativo, coercitivo e generalista das leis e de outras formas de regulamentação.
\BNCC{EF69LP21}
 %Posicionar-se em relação a conteúdos veiculados em práticas não institucionalizadas de participação social, sobretudo àquelas vinculadas a manifestações artísticas, produções culturais, intervenções urbanas e práticas próprias das culturas juvenis que pretendam denunciar, expor uma problemática ou “convocar” para uma reflexão/ação, relacionando esse texto/produção com seu contexto de produção e relacionando as partes e semioses presentes para a construção de sentidos.
\BNCC{EF69LP22}
 %Produzir, revisar e editar textos reivindicatórios ou propositivos sobre problemas que afetam a vida escolar ou da comunidade, justificando pontos de vista, reivindicações e detalhando propostas (justificativa, objetivos, ações previstas etc.), levando em conta seu contexto de produção e as características dos gêneros em questão.
\BNCC{EF69LP23}
 %Contribuir com a escrita de textos normativos, quando houver esse tipo de demanda na escola – regimentos e estatutos de organizações da sociedade civil do âmbito da atuação das crianças e jovens (grêmio livre, clubes de leitura, associações culturais etc.) – e de regras e regulamentos nos vários âmbitos da escola – campeonatos, festivais, regras de convivência etc., levando em conta o contexto de produção e as características dos gêneros em questão.
\BNCC{EF69LP24}
 %Discutir casos, reais ou simulações, submetidos a juízo, que envolvam (supostos) desrespeitos a artigos, do ECA, do Código de Defesa do Consumidor, do Código Nacional de Trânsito, de regulamentações do mercado publicitário etc., como forma de criar familiaridade com textos legais – seu vocabulário, formas de organização, marcas de estilo etc. -, de maneira a facilitar a compreensão de leis, fortalecer a defesa de direitos, fomentar a escrita de textos normativos (se e quando isso for necessário) e possibilitar a compreensão do caráter interpretativo das leis e as várias perspectivas que podem estar em jogo.
\BNCC{EF69LP25}
 %Posicionar-se de forma consistente e sustentada em uma discussão, assembleia, reuniões de colegiados da escola, de agremiações e outras situações de apresentação de propostas e defesas de opiniões, respeitando as opiniões contrárias e propostas alternativas e fundamentando seus posicionamentos, no tempo de fala previsto, valendo-se de sínteses e propostas claras e justificadas.
\BNCC{EF69LP26}
 %Tomar nota em discussões, debates, palestras, apresentação de propostas, reuniões, como forma de documentar o evento e apoiar a própria fala (que pode se dar no momento do evento ou posteriormente, quando, por exemplo, for necessária a retomada dos assuntos tratados em outros contextos públicos, como diante dos representados).
\BNCC{EF69LP27}
 %Analisar a forma composicional de textos pertencentes a gêneros normativos/ jurídicos e a gêneros da esfera política, tais como propostas, programas políticos (posicionamento quanto a diferentes ações a serem propostas, objetivos, ações previstas etc.), propaganda política (propostas e sua sustentação, posicionamento quanto a temas em discussão) e textos reivindicatórios: cartas de reclamação, petição (proposta, suas justificativas e ações a serem adotadas) e suas marcas linguísticas, de forma a incrementar a compreensão de textos pertencentes a esses gêneros e a possibilitar a produção de textos mais adequados e/ou fundamentados quando isso for requerido.
\BNCC{EF69LP28}
 %Observar os mecanismos de modalização adequados aos textos jurídicos, as modalidades deônticas, que se referem ao eixo da conduta (obrigatoriedade/permissibilidade) como, por exemplo: Proibição: “Não se deve fumar em recintos fechados.”; Obrigatoriedade: “A vida tem que valer a pena.”; Possibilidade: “É permitido a entrada de menores acompanhados de adultos responsáveis”, e os mecanismos de modalização adequados aos textos políticos e propositivos, as modalidades apreciativas, em que o locutor exprime um juízo de valor (positivo ou negativo) acerca do que enuncia. Por exemplo: “Que belo discurso!”, “Discordo das escolhas de Antônio.” “Felizmente, o buraco ainda não causou acidentes mais graves.”
\BNCC{EF69LP29}
 %Refletir sobre a relação entre os contextos de produção dos gêneros de divulgação científica – texto didático, artigo de divulgação científica, reportagem de divulgação científica, verbete de enciclopédia (impressa e digital), esquema, infográfico (estático e animado), relatório, relato multimidiático de campo, podcasts e vídeos variados de divulgação científica etc. – e os aspectos relativos à construção composicional e às marcas linguística características desses gêneros, de forma a ampliar suas possibilidades de compreensão (e produção) de textos pertencentes a esses gêneros.
\BNCC{EF69LP30}
 %Comparar, com a ajuda do professor, conteúdos, dados e informações de diferentes fontes, levando em conta seus contextos de produção e referências, identificando coincidências, complementaridades e contradições, de forma a poder identificar erros/imprecisões conceituais, compreender e posicionar-se criticamente sobre os conteúdos e informações em questão.
\BNCC{EF69LP31}
 %Utilizar pistas linguísticas – tais como “em primeiro/segundo/terceiro lugar”, “por outro lado”, “dito de outro modo”, isto é”, “por exemplo” – para compreender a hierarquização das proposições, sintetizando o conteúdo dos textos.
\BNCC{EF69LP32}
 %Selecionar informações e dados relevantes de fontes diversas (impressas, digitais, orais etc.), avaliando a qualidade e a utilidade dessas fontes, e organizar, esquematicamente, com ajuda do professor, as informações necessárias (sem excedê-las) com ou sem apoio de ferramentas digitais, em quadros, tabelas ou gráficos.
\BNCC{EF69LP33}
 %Articular o verbal com os esquemas, infográficos, imagens variadas etc. na (re)construção dos sentidos dos textos de divulgação científica e retextualizar do discursivo para o esquemático – infográfico, esquema, tabela, gráfico, ilustração etc. – e, ao contrário, transformar o conteúdo das tabelas, esquemas, infográficos, ilustrações etc. em texto discursivo, como forma de ampliar as possibilidades de compreensão desses textos e analisar as características das multissemioses e dos gêneros em questão.
\BNCC{EF69LP34}
 %Grifar as partes essenciais do texto, tendo em vista os objetivos de leitura, produzir marginálias (ou tomar notas em outro suporte), sínteses organizadas em itens, quadro sinóptico, quadro comparativo, esquema, resumo ou resenha do texto lido (com ou sem comentário/análise), mapa conceitual, dependendo do que for mais adequado, como forma de possibilitar uma maior compreensão do texto, a sistematização de conteúdos e informações e
\BNCC{EF69LP35}
 %Planejar textos de divulgação científica, a partir da elaboração de esquema que considere as pesquisas feitas anteriormente, de notas e sínteses de leituras ou de registros de experimentos ou de estudo de campo, produzir, revisar e editar textos voltados para a divulgação do conhecimento e de dados e resultados de pesquisas, tais como artigo de divulgação científica, artigo de opinião, reportagem científica, verbete de enciclopédia, verbete de enciclopédia digital colaborativa , infográfico, relatório, relato de experimento científico, relato (multimidiático) de campo, tendo em vista seus contextos de produção, que podem envolver a disponibilização de informações e conhecimentos em circulação em um formato mais acessível para um público específico ou a divulgação de conhecimentos advindos de pesquisas bibliográficas, experimentos científicos e estudos de campo realizados.
\BNCC{EF69LP36}
 %Produzir, revisar e editar textos voltados para a divulgação do conhecimento e de dados e resultados de pesquisas, tais como artigos de divulgação científica, verbete de enciclopédia, infográfico, infográfico animado, podcast ou vlog científico, relato de experimento, relatório, relatório multimidiático de campo, dentre outros, considerando o contexto de produção e as regularidades dos gêneros em termos de suas construções composicionais e estilos.
\BNCC{EF69LP37}
 %Produzir roteiros para elaboração de vídeos de diferentes tipos (vlog científico, vídeo-minuto, programa de rádio, podcasts) para divulgação de conhecimentos científicos e resultados de pesquisa, tendo em vista seu contexto de produção, os elementos e a construção composicional dos roteiros.
\BNCC{EF69LP38}
 %Organizar os dados e informações pesquisados em painéis ou slides de apresentação, levando em conta o contexto de produção, o tempo disponível, as características do gênero apresentação oral, a multissemiose, as mídias e tecnologias que serão utilizadas, ensaiar a apresentação, considerando também elementos paralinguísticos e cinésicos e proceder à exposição oral de resultados de estudos e pesquisas, no tempo determinado, a partir do planejamento e da definição de diferentes formas de uso da fala – memorizada, com apoio da leitura ou fala espontânea.
\BNCC{EF69LP39}
 %Definir o recorte temático da entrevista e o entrevistado, levantar informações sobre o entrevistado e sobre o tema da entrevista, elaborar roteiro de perguntas, realizar entrevista, a partir do roteiro, abrindo possibilidades para fazer perguntas a partir da resposta, se o contexto permitir, tomar nota, gravar ou salvar a entrevista e usar adequadamente as informações obtidas, de acordo com os objetivos estabelecidos.
\BNCC{EF69LP40}
 %Analisar, em gravações de seminários, conferências rápidas, trechos de palestras, dentre outros, a construção composicional dos gêneros de apresentação – abertura/saudação, introdução ao tema, apresentação do plano de exposição, desenvolvimento dos conteúdos, por meio do encadeamento de temas e subtemas (coesão temática), síntese final e/ou conclusão, encerramento –, os elementos paralinguísticos (tais como: tom e volume da voz, pausas e hesitações – que, em geral, devem ser minimizadas –, modulação de voz e entonação, ritmo, respiração etc.) e cinésicos (tais como: postura corporal, movimentos e gestualidade significativa, expressão facial, contato de olho com plateia, modulação de voz e entonação, sincronia da fala com ferramenta de apoio etc.), para melhor performar apresentações orais no campo da divulgação do conhecimento.
\BNCC{EF69LP41}
 %Usar adequadamente ferramentas de apoio a apresentações orais, escolhendo e usando tipos e tamanhos de fontes que permitam boa visualização, topicalizando e/ou organizando o conteúdo em itens, inserindo de forma adequada imagens, gráficos, tabelas, formas e elementos gráficos, dimensionando a quantidade de texto (e imagem) por slide, usando progressivamente e de forma harmônica recursos mais sofisticados como efeitos de transição, slides mestres, layouts personalizados etc.
\BNCC{EF69LP42}
 %Analisar a construção composicional dos textos pertencentes a gêneros relacionados à divulgação de conhecimentos: título, (olho), introdução, divisão do texto em subtítulos, imagens ilustrativas de conceitos, relações, ou resultados complexos (fotos, ilustrações, esquemas, gráficos, infográficos, diagramas, figuras, tabelas, mapas) etc., exposição, contendo definições, descrições, comparações, enumerações, exemplificações e remissões a conceitos e relações por meio de notas de rodapé, boxes ou links; ou título, contextualização do campo, ordenação temporal ou temática por tema ou subtema, intercalação de trechos verbais com fotos, ilustrações, áudios, vídeos etc. e reconhecer traços da linguagem dos textos de divulgação científica, fazendo uso consciente das estratégias de impessoalização da linguagem (ou de pessoalização, se o tipo de publicação e objetivos assim o demandarem, como em alguns podcasts e vídeos de divulgação científica), 3ª pessoa, presente atemporal, recurso à citação, uso de vocabulário técnico/especializado etc., como forma de ampliar suas capacidades de compreensão e produção de textos nesses gêneros.
\BNCC{EF69LP43}
 %Identificar e utilizar os modos de introdução de outras vozes no texto – citação literal e sua formatação e paráfrase –, as pistas linguísticas responsáveis por introduzir no texto a posição do autor e dos outros autores citados (“Segundo X; De acordo com Y; De minha/nossa parte, penso/amos que”...) e os elementos de normatização (tais como as regras de inclusão e formatação de citações e paráfrases, de organização de referências bibliográficas) em textos científicos, desenvolvendo reflexão sobre o modo como a intertextualidade e a retextualização ocorrem nesses textos.
\BNCC{EF69LP44}
 %Inferir a presença de valores sociais, culturais e humanos e de diferentes visões de mundo, em textos literários, reconhecendo nesses textos formas de estabelecer múltiplos olhares sobre as identidades, sociedades e culturas e considerando a autoria e o contexto social e histórico de sua produção.
\BNCC{EF69LP45}
 %Posicionar-se criticamente em relação a textos pertencentes a gêneros como quarta-capa, programa (de teatro, dança, exposição etc.), sinopse, resenha crítica, comentário em blog/vlog cultural etc., para selecionar obras literárias e outras manifestações artísticas (cinema, teatro, exposições, espetáculos, CD´s, DVD´s etc.), diferenciando as sequências descritivas e avaliativas e reconhecendo-os como gêneros que apoiam a escolha do livro ou produção cultural e consultando-os no momento de fazer escolhas, quando for o caso.
\BNCC{EF69LP46}
 %Participar de práticas de compartilhamento de leitura/recepção de obras literárias/ manifestações artísticas, como rodas de leitura, clubes de leitura, eventos de contação de histórias, de leituras dramáticas, de apresentações teatrais, musicais e de filmes, cineclubes, festivais de vídeo, saraus, slams, canais de booktubers, redes sociais temáticas (de leitores, de cinéfilos, de música etc.), dentre outros, tecendo, quando possível, comentários de ordem estética e afetiva
\BNCC{EF69LP47}
 %Analisar, em textos narrativos ficcionais, as diferentes formas de composição próprias de cada gênero, os recursos coesivos que constroem a passagem do tempo e articulam suas partes, a escolha lexical típica de cada gênero para a caracterização dos cenários e dos personagens e os efeitos de sentido decorrentes dos tempos verbais, dos tipos de discurso, dos verbos de enunciação e das variedades linguísticas (no discurso direto, se houver) empregados, identificando o enredo e o foco narrativo e percebendo como se estrutura a narrativa nos diferentes gêneros e os efeitos de sentido decorrentes do foco narrativo típico de cada gênero, da caracterização dos espaços físico e psicológico e dos tempos cronológico e psicológico, das diferentes vozes no texto (do narrador, de personagens em discurso direto e indireto), do uso de pontuação expressiva, palavras e expressões conotativas e processos figurativos e do uso de recursos linguístico-gramaticais próprios a cada gênero narrativo.
\BNCC{EF69LP48}
 %Interpretar, em poemas, efeitos produzidos pelo uso de recursos expressivos sonoros (estrofação, rimas, aliterações etc), semânticos (figuras de linguagem, por exemplo), gráfico- espacial (distribuição da mancha gráfica no papel), imagens e sua relação com o texto verbal.
\BNCC{EF69LP49}
 %Mostrar-se interessado e envolvido pela leitura de livros de literatura e por outras produções culturais do campo e receptivo a textos que rompam com seu universo de expectativas, que representem um desafio em relação às suas possibilidades atuais e suas experiências anteriores de leitura, apoiando-se nas marcas linguísticas, em seu conhecimento sobre os gêneros e a temática e nas orientações dadas pelo professor.
\BNCC{EF69LP50}
 %Elaborar texto teatral, a partir da adaptação de romances, contos, mitos, narrativas de enigma e de aventura, novelas, biografias romanceadas, crônicas, dentre outros, indicando as rubricas para caracterização do cenário, do espaço, do tempo; explicitando a caracterização física e psicológica dos personagens e dos seus modos de ação; reconfigurando a inserção do discurso direto e dos tipos de narrador; explicitando as marcas de variação linguística (dialetos, registros e jargões) e retextualizando o tratamento da temática.
\BNCC{EF69LP51}
 %Engajar-se ativamente nos processos de planejamento, textualização, revisão/ edição e reescrita, tendo em vista as restrições temáticas, composicionais e estilísticas dos textos pretendidos e as configurações da situação de produção – o leitor pretendido, o suporte, o contexto de circulação do texto, as finalidades etc. – e considerando a imaginação, a estesia e a verossimilhança próprias ao texto literário.
\BNCC{EF69LP52}
 %Representar cenas ou textos dramáticos, considerando, na caracterização dos personagens, os aspectos linguísticos e paralinguísticos das falas (timbre e tom de voz, pausas e hesitações, entonação e expressividade, variedades e registros linguísticos), os gestos e os deslocamentos no espaço cênico, o figurino e a maquiagem e elaborando as rubricas indicadas pelo autor por meio do cenário, da trilha sonora e da exploração dos modos de interpretação.
\BNCC{EF69LP53}
 %Ler em voz alta textos literários diversos – como contos de amor, de humor, de suspense, de terror; crônicas líricas, humorísticas, críticas; bem como leituras orais capituladas (compartilhadas ou não com o professor) de livros de maior extensão, como romances, narrativas de enigma, narrativas de aventura, literatura infantojuvenil, – contar/recontar histórias tanto da tradição oral (causos, contos de esperteza, contos de animais, contos de amor, contos de encantamento, piadas, dentre outros) quanto da tradição literária escrita, expressando a compreensão e interpretação do texto por meio de uma leitura ou fala expressiva e fluente, que respeite o ritmo, as pausas, as hesitações, a entonação indicados tanto pela pontuação quanto por outros recursos gráfico-editoriais, como negritos, itálicos, caixa-alta, ilustrações etc., gravando essa leitura ou esse conto/reconto, seja para análise posterior, seja para produção de audiobooks de textos literários diversos ou de podcasts de leituras dramáticas com ou sem efeitos especiais e ler e/ou declamar poemas diversos, tanto de forma livre quanto de forma fixa (como quadras, sonetos, liras, haicais etc.), empregando os recursos linguísticos, paralinguísticos e cinésicos necessários aos efeitos de sentido pretendidos, como o ritmo e a entonação, o emprego de pausas e prolongamentos, o tom e o timbre vocais, bem como eventuais recursos de gestualidade e pantomima que convenham ao gênero poético e à situação de compartilhamento em questão.
\BNCC{EF69LP54}
 %Analisar os efeitos de sentido decorrentes da interação entre os elementos linguísticos e os recursos paralinguísticos e cinésicos, como as variações no ritmo, as modulações no tom de voz, as pausas, as manipulações do estrato sonoro da linguagem, obtidos por meio da estrofação, das rimas e de figuras de linguagem como as aliterações, as assonâncias, as onomatopeias, dentre outras, a postura corporal e a gestualidade, na declamação de poemas, apresentações musicais e teatrais, tanto em gêneros em prosa quanto nos gêneros poéticos, os efeitos de sentido decorrentes do emprego de figuras de linguagem, tais como comparação, metáfora, personificação, metonímia, hipérbole, eufemismo, ironia, paradoxo e antítese e os efeitos de sentido decorrentes do emprego de palavras e expressões denotativas e conotativas (adjetivos, locuções adjetivas, orações subordinadas adjetivas etc.), que funcionam como modificadores, percebendo sua função na caracterização dos espaços, tempos, personagens e ações próprios de cada gênero narrativo.
\BNCC{EF69LP55}
 %Reconhecer as variedades da língua falada, o conceito de norma-padrão e o de preconceito linguístico.
\BNCC{EF69LP56}
 %Fazer uso consciente e reflexivo de regras e normas da norma-padrão em situações de fala e escrita nas quais ela deve ser usada.
\BNCC{EF07LP01}
 %Distinguir diferentes propostas editoriais – sensacionalismo, jornalismo investigativo etc. –, de forma a identificar os recursos utilizados para impactar/chocar o leitor que podem comprometer uma análise crítica da notícia e do fato noticiado.
\BNCC{EF07LP02}
 %Comparar notícias e reportagens sobre um mesmo fato divulgadas em diferentes mídias, analisando as especificidades das mídias, os processos de (re)elaboração dos textos e a convergência das mídias em notícias ou reportagens multissemióticas.
\BNCC{EF07LP03}
 %Formar, com base em palavras primitivas, palavras derivadas com os prefixos e sufixos mais produtivos no português.
\BNCC{EF07LP04}
 %Reconhecer, em textos, o verbo como o núcleo das orações.
\BNCC{EF07LP05}
 %Identificar, em orações de textos lidos ou de produção própria, verbos de predicação completa e incompleta: intransitivos e transitivos.
\BNCC{EF07LP06}
 %Empregar as regras básicas de concordância nominal e verbal em situações comunicativas e na produção de textos.
\BNCC{EF07LP07}
 %Identificar, em textos lidos ou de produção própria, a estrutura básica da oração: sujeito, predicado, complemento (objetos direto e indireto).
\BNCC{EF07LP08}
 %Identificar, em textos lidos ou de produção própria, adjetivos que ampliam o sentido do substantivo sujeito ou complemento verbal.
\BNCC{EF07LP09}
 %Identificar, em textos lidos ou de produção própria, advérbios e locuções adverbiais que ampliam o sentido do verbo núcleo da oração.
\BNCC{EF07LP10}
 %Utilizar, ao produzir texto, conhecimentos linguísticos e gramaticais: modos e tempos verbais, concordância nominal e verbal, pontuação etc.
\BNCC{EF07LP11}
 %Identificar, em textos lidos ou de produção própria, períodos compostos nos quais duas orações são conectadas por vírgula, ou por conjunções que expressem soma de sentido (conjunção “e”) ou oposição de sentidos (conjunções “mas”, “porém”).
\BNCC{EF07LP12}
 %Reconhecer recursos de coesão referencial: substituições lexicais (de substantivos por sinônimos) ou pronominais (uso de pronomes anafóricos – pessoais, possessivos, demonstrativos).
\BNCC{EF07LP13}
 %Estabelecer relações entre partes do texto, identificando substituições lexicais (de substantivos por sinônimos) ou pronominais (uso de pronomes anafóricos – pessoais, possessivos, demonstrativos), que contribuem para a continuidade do texto.
\BNCC{EF07LP14}
 %Identificar, em textos, os efeitos de sentido do uso de estratégias de modalização e argumentatividade.
\BNCC{EF08LP01}
 %Identificar e comparar as várias editorias de jornais impressos e digitais e de sites noticiosos, de forma a refletir sobre os tipos de fato que são noticiados e comentados, as escolhas sobre o que noticiar e o que não noticiar e o destaque/enfoque dado e a fidedignidade da informação.
\BNCC{EF08LP02}
 %Justificar diferenças ou semelhanças no tratamento dado a uma mesma informação veiculada em textos diferentes, consultando sites e serviços de checadores de fatos.
\BNCC{EF08LP03}
 %Produzir artigos de opinião, tendo em vista o contexto de produção dado, a defesa de um ponto de vista, utilizando argumentos e contra-argumentos e articuladores de coesão que marquem relações de oposição, contraste, exemplificação, ênfase.
\BNCC{EF08LP04}
 %Utilizar, ao produzir texto, conhecimentos linguísticos e gramaticais: ortografia, regências e concordâncias nominal e verbal, modos e tempos verbais, pontuação etc.
\BNCC{EF08LP05}
 %Analisar processos de formação de palavras por composição (aglutinação e justaposição), apropriando-se de regras básicas de uso do hífen em palavras compostas.
\BNCC{EF08LP06}
 %Identificar, em textos lidos ou de produção própria, os termos constitutivos da oração (sujeito e seus modificadores, verbo e seus complementos e modificadores).
\BNCC{EF08LP07}
 %Diferenciar, em textos lidos ou de produção própria, complementos diretos e indiretos de verbos transitivos, apropriando-se da regência de verbos de uso frequente.
\BNCC{EF08LP08}
 %Identificar, em textos lidos ou de produção própria, verbos na voz ativa e na voz passiva, interpretando os efeitos de sentido de sujeito ativo e passivo (agente da passiva).
\BNCC{EF08LP09}
 %Interpretar efeitos de sentido de modificadores (adjuntos adnominais – artigos definido ou indefinido, adjetivos, expressões adjetivas) em substantivos com função de sujeito ou de complemento verbal, usando-os para enriquecer seus próprios textos.
\BNCC{EF08LP10}
 %Interpretar, em textos lidos ou de produção própria, efeitos de sentido de modificadores do verbo (adjuntos adverbiais – advérbios e expressões adverbiais), usando-os para enriquecer seus próprios textos.
\BNCC{EF08LP11}
 %Identificar, em textos lidos ou de produção própria, agrupamento de orações em períodos, diferenciando coordenação de subordinação.
\BNCC{EF08LP12}
 %Identificar, em textos lidos, orações subordinadas com conjunções de uso frequente, incorporando-as às suas próprias produções.
\BNCC{EF08LP13}
 %Inferir efeitos de sentido decorrentes do uso de recursos de coesão sequencial: conjunções e articuladores textuais.
\BNCC{EF08LP14}
 %Utilizar, ao produzir texto, recursos de coesão sequencial (articuladores) e referencial (léxica e pronominal), construções passivas e impessoais, discurso direto e indireto e outros recursos expressivos adequados ao gênero textual.
\BNCC{EF08LP15}
 %Estabelecer relações entre partes do texto, identificando o antecedente de um pronome relativo ou o referente comum de uma cadeia de substituições lexicais.
\BNCC{EF08LP16}
 %Explicar os efeitos de sentido do uso, em textos, de estratégias de modalização e argumentatividade (sinais de pontuação, adjetivos, substantivos, expressões de grau, verbos e perífrases verbais, advérbios etc.).
\BNCC{EF89LP01}
 %Analisar os interesses que movem o campo jornalístico, os efeitos das novas tecnologias no campo e as condições que fazem da informação uma mercadoria, de forma a poder desenvolver uma atitude crítica frente aos textos jornalísticos.
\BNCC{EF89LP02}
 %Analisar diferentes práticas (curtir, compartilhar, comentar, curar etc.) e textos pertencentes a diferentes gêneros da cultura digital (meme, gif, comentário, charge digital etc.) envolvidos no trato com a informação e opinião, de forma a possibilitar uma presença mais crítica e ética nas redes.
\BNCC{EF89LP03}
 %Analisar textos de opinião (artigos de opinião, editoriais, cartas de leitores, comentários, posts de blog e de redes sociais, charges, memes, gifs etc.) e posicionar-se de forma crítica e fundamentada, ética e respeitosa frente a fatos e opiniões relacionados a esses textos.
\BNCC{EF89LP04}
 %Identificar e avaliar teses/opiniões/posicionamentos explícitos e implícitos, argumentos e contra-argumentos em textos argumentativos do campo (carta de leitor, comentário, artigo de opinião, resenha crítica etc.), posicionando-se frente à questão controversa de forma sustentada.
\BNCC{EF89LP05}
 %Analisar o efeito de sentido produzido pelo uso, em textos, de recurso a formas de apropriação textual (paráfrases, citações, discurso direto, indireto ou indireto livre).
\BNCC{EF89LP06}
 %Analisar o uso de recursos persuasivos em textos argumentativos diversos (como a elaboração do título, escolhas lexicais, construções metafóricas, a explicitação ou a ocultação de fontes de informação) e seus efeitos de sentido.
\BNCC{EF89LP07}
 %Analisar, em notícias, reportagens e peças publicitárias em várias mídias, os efeitos de sentido devidos ao tratamento e à composição dos elementos nas imagens em movimento, à performance, à montagem feita (ritmo, duração e sincronização entre as linguagens – complementaridades, interferências etc.) e ao ritmo, melodia, instrumentos e sampleamentos das músicas e efeitos sonoros.
\BNCC{EF89LP08}
 %Planejar reportagem impressa e em outras mídias (rádio ou TV/vídeo, sites), tendo em vista as condições de produção do texto – objetivo, leitores/espectadores, veículos e mídia de circulação etc. – a partir da escolha do fato a ser aprofundado ou do tema a ser focado (de relevância para a turma, escola ou comunidade), do levantamento de dados e informações sobre o fato ou tema – que pode envolver entrevistas com envolvidos ou com especialistas, consultas a fontes diversas, análise de documentos, cobertura de eventos etc. -, do registro dessas informações e dados, da escolha de fotos ou imagens a produzir ou a utilizar etc., da produção de infográficos, quando for o caso, e da organização hipertextual (no caso a publicação em sites ou blogs noticiosos ou mesmo de jornais impressos, por meio de boxes variados).
\BNCC{EF89LP09}
 %Produzir reportagem impressa, com título, linha fina (optativa), organização composicional (expositiva, interpretativa e/ou opinativa), progressão temática e uso de recursos linguísticos compatíveis com as escolhas feitas e reportagens multimidiáticas, tendo em vista as condições de produção, as características do gênero, os recursos e mídias disponíveis, sua organização hipertextual e o manejo adequado de recursos de captação e edição de áudio e imagem e adequação à norma-padrão.
\BNCC{EF89LP10}
 %Planejar artigos de opinião, tendo em vista as condições de produção do texto – objetivo, leitores/espectadores, veículos e mídia de circulação etc. –, a partir da escolha do tema ou questão a ser discutido(a), da relevância para a turma, escola ou comunidade, do levantamento de dados e informações sobre a questão, de argumentos relacionados a diferentes posicionamentos em jogo, da definição – o que pode envolver consultas a fontes diversas, entrevistas com especialistas, análise de textos, organização esquemática das informações e argumentos – dos (tipos de) argumentos e estratégias que pretende utilizar para convencer os leitores.
\BNCC{EF89LP11}
 %Produzir, revisar e editar peças e campanhas publicitárias, envolvendo o uso articulado e complementar de diferentes peças publicitárias: cartaz, banner, indoor, folheto, panfleto, anúncio de jornal/revista, para internet, spot, propaganda de rádio, TV, a partir da escolha da questão/problema/causa significativa para a escola e/ou a comunidade escolar, da definição do público-alvo, das peças que serão produzidas, das estratégias de persuasão e convencimento que serão utilizadas.
\BNCC{EF89LP12}
 %Planejar coletivamente a realização de um debate sobre tema previamente definido, de interesse coletivo, com regras acordadas e planejar, em grupo, participação em debate a partir do levantamento de informações e argumentos que possam sustentar o posicionamento a ser defendido (o que pode envolver entrevistas com especialistas, consultas a fontes diversas, o registro das informações e dados obtidos etc.), tendo em vista as condições de produção do debate – perfil dos ouvintes e demais participantes, objetivos do debate, motivações para sua realização, argumentos e estratégias de convencimento mais eficazes etc. e participar de debates regrados, na condição de membro de uma equipe de debatedor, apresentador/mediador, espectador (com ou sem direito a perguntas), e/ou de juiz/avaliador, como forma de compreender o funcionamento do debate, e poder participar de forma convincente, ética, respeitosa e crítica e desenvolver uma atitude de respeito e diálogo para com as ideias divergentes.
\BNCC{EF89LP13}
 %Planejar entrevistas orais com pessoas ligadas ao fato noticiado, especialistas etc., como forma de obter dados e informações sobre os fatos cobertos sobre o tema ou questão discutida ou temáticas em estudo, levando em conta o gênero e seu contexto de produção, partindo do levantamento de informações sobre o entrevistado e sobre a temática e da elaboração de um roteiro de perguntas, garantindo a relevância das informações mantidas e a continuidade temática, realizar entrevista e fazer edição em áudio ou vídeo, incluindo uma contextualização inicial e uma fala de encerramento para publicação da entrevista isoladamente ou como parte integrante de reportagem multimidiática, adequando-a a seu contexto de publicação e garantindo a relevância das informações mantidas e a continuidade temática.
\BNCC{EF89LP14}
 %Analisar, em textos argumentativos e propositivos, os movimentos argumentativos de sustentação, refutação e negociação e os tipos de argumentos, avaliando a força/tipo dos argumentos utilizados.
\BNCC{EF89LP15}
 %Utilizar, nos debates, operadores argumentativos que marcam a defesa de ideia e de diálogo com a tese do outro: concordo, discordo, concordo parcialmente, do meu ponto de vista, na perspectiva aqui assumida etc.
\BNCC{EF89LP16}
 %Analisar a modalização realizada em textos noticiosos e argumentativos, por meio das modalidades apreciativas, viabilizadas por classes e estruturas gramaticais como adjetivos, locuções adjetivas, advérbios, locuções adverbiais, orações adjetivas e adverbiais, orações relativas restritivas e explicativas etc., de maneira a perceber a apreciação ideológica sobre os fatos noticiados ou as posições implícitas ou assumidas.
\BNCC{EF89LP17}
 %Relacionar textos e documentos legais e normativos de importância universal, nacional ou local que envolvam direitos, em especial, de crianças, adolescentes e jovens – tais como a Declaração dos Direitos Humanos, a Constituição Brasileira, o ECA -, e a regulamentação da organização escolar – por exemplo, regimento escolar -, a seus contextos de produção, reconhecendo e analisando possíveis motivações, finalidades e sua vinculação com experiências humanas e fatos históricos e sociais, como forma de ampliar a compreensão dos direitos e deveres, de fomentar os princípios democráticos e uma atuação pautada pela ética da responsabilidade (o outro tem direito a uma vida digna tanto quanto eu tenho).
\BNCC{EF89LP18}
 %Explorar e analisar instâncias e canais de participação disponíveis na escola (conselho de escola, outros colegiados, grêmio livre), na comunidade (associações, coletivos, movimentos, etc.), no munícipio ou no país, incluindo formas de participação digital, como canais e plataformas de participação (como portal e-cidadania), serviços, portais e ferramentas de acompanhamentos do trabalho de políticos e de tramitação de leis, canais de educação política, bem como de propostas e proposições que circulam nesses canais, de forma a participar do debate de ideias e propostas na esfera social e a engajar-se com a busca de soluções para problemas ou questões que envolvam a vida da escola e da comunidade.
\BNCC{EF89LP19}
 %Analisar, a partir do contexto de produção, a forma de organização das cartas abertas, abaixo-assinados e petições on-line (identificação dos signatários, explicitação da reivindicação feita, acompanhada ou não de uma breve apresentação da problemática e/ou de justificativas que visam sustentar a reivindicação) e a proposição, discussão e aprovação de propostas políticas ou de soluções para problemas de interesse público, apresentadas ou lidas nos canais digitais de participação, identificando suas marcas linguísticas, como forma de possibilitar a escrita ou subscrição consciente de abaixo-assinados e textos dessa natureza e poder se posicionar de forma crítica e fundamentada frente às propostas
\BNCC{EF89LP20}
 %Comparar propostas políticas e de solução de problemas, identificando o que se pretende fazer/implementar, por que (motivações, justificativas), para que (objetivos, benefícios e consequências esperados), como (ações e passos), quando etc. e a forma de avaliar a eficácia da proposta/solução, contrastando dados e informações de diferentes fontes, identificando coincidências, complementaridades e contradições, de forma a poder compreender e posicionar-se criticamente sobre os dados e informações usados em fundamentação de propostas e analisar a coerência entre os elementos, de forma a tomar decisões fundamentadas.
\BNCC{EF89LP21}
 %Realizar enquetes e pesquisas de opinião, de forma a levantar prioridades, problemas a resolver ou propostas que possam contribuir para melhoria da escola ou da comunidade, caracterizar demanda/necessidade, documentando-a de diferentes maneiras por meio de diferentes procedimentos, gêneros e mídias e, quando for o caso, selecionar informações e dados relevantes de fontes pertinentes diversas (sites, impressos, vídeos etc.), avaliando a qualidade e a utilidade dessas fontes, que possam servir de contextualização e fundamentação de propostas, de forma a justificar a proposição de propostas, projetos culturais e ações de intervenção.
\BNCC{EF89LP22}
 %Compreender e comparar as diferentes posições e interesses em jogo em uma discussão ou apresentação de propostas, avaliando a validade e força dos argumentos e as consequências do que está sendo proposto e, quando for o caso, formular e negociar propostas de diferentes naturezas relativas a interesses coletivos envolvendo a escola ou comunidade escolar.
\BNCC{EF89LP23}
 %Analisar, em textos argumentativos, reivindicatórios e propositivos, os movimentos argumentativos utilizados (sustentação, refutação e negociação), avaliando a força dos argumentos utilizados.
\BNCC{EF89LP24}
 %Realizar pesquisa, estabelecendo o recorte das questões, usando fontes abertas e confiáveis.
\BNCC{EF89LP25}
 %Divulgar o resultado de pesquisas por meio de apresentações orais, verbetes de enciclopédias colaborativas, reportagens de divulgação científica, vlogs científicos, vídeos de diferentes tipos etc.
\BNCC{EF89LP26}
 %Produzir resenhas, a partir das notas e/ou esquemas feitos, com o manejo adequado das vozes envolvidas (do resenhador, do autor da obra e, se for o caso, também dos autores citados na obra resenhada), por meio do uso de paráfrases, marcas do discurso reportado e citações.
\BNCC{EF89LP27}
 %Tecer considerações e formular problematizações pertinentes, em momentos oportunos, em situações de aulas, apresentação oral, seminário etc.
\BNCC{EF89LP28}
 %Tomar nota de videoaulas, aulas digitais, apresentações multimídias, vídeos de divulgação científica, documentários e afins, identificando, em função dos objetivos, informações principais para apoio ao estudo e realizando, quando necessário, uma síntese final que destaque e reorganize os pontos ou conceitos centrais e suas relações e que, em alguns casos, seja acompanhada de reflexões pessoais, que podem conter dúvidas, questionamentos, considerações etc.
\BNCC{EF89LP29}
 %Utilizar e perceber mecanismos de progressão temática, tais como retomadas anafóricas (“que, cujo, onde”, pronomes do caso reto e oblíquos, pronomes demonstrativos, nomes correferentes etc.), catáforas (remetendo para adiante ao invés de retomar o já dito), uso de organizadores textuais, de coesivos etc., e analisar os mecanismos de reformulação e paráfrase utilizados nos textos de divulgação do conhecimento.
\BNCC{EF89LP30}
 %Analisar a estrutura de hipertexto e hiperlinks em textos de divulgação científica que circulam na Web e proceder à remissão a conceitos e relações por meio de links.
\BNCC{EF89LP31}
 %Analisar e utilizar modalização epistêmica, isto é, modos de indicar uma avaliação sobre o valor de verdade e as condições de verdade de uma proposição, tais como os asseverativos – quando se concorda com (“realmente, evidentemente, naturalmente, efetivamente, claro, certo, lógico, sem dúvida” etc.) ou discorda de (“de jeito nenhum, de forma alguma”) uma ideia; e os quase-asseverativos, que indicam que se considera o conteúdo como quase certo (“talvez, assim, possivelmente, provavelmente, eventualmente”).
\BNCC{EF89LP32}
 %Analisar os efeitos de sentido decorrentes do uso de mecanismos de intertextualidade (referências, alusões, retomadas) entre os textos literários, entre esses textos literários e outras manifestações artísticas (cinema, teatro, artes visuais e midiáticas, música), quanto aos temas, personagens, estilos, autores etc., e entre o texto original e paródias, paráfrases, pastiches, trailer honesto, vídeos-minuto, vidding, dentre outros.
\BNCC{EF89LP33}
 %Ler, de forma autônoma, e compreender – selecionando procedimentos e estratégias de leitura adequados a diferentes objetivos e levando em conta características dos gêneros e suportes – romances, contos contemporâneos, minicontos, fábulas contemporâneas, romances juvenis, biografias romanceadas, novelas, crônicas visuais, narrativas de ficção científica, narrativas de suspense, poemas de forma livre e fixa (como haicai), poema concreto, ciberpoema, dentre outros, expressando avaliação sobre o texto lido e estabelecendo preferências por gêneros, temas, autores.
\BNCC{EF89LP34}
 %Analisar a organização de texto dramático apresentado em teatro, televisão, cinema, identificando e percebendo os sentidos decorrentes dos recursos linguísticos e semióticos que sustentam sua realização como peça teatral, novela, filme etc.
\BNCC{EF89LP35}
 %Criar contos ou crônicas (em especial, líricas), crônicas visuais, minicontos, narrativas de aventura e de ficção científica, dentre outros, com temáticas próprias ao gênero, usando os conhecimentos sobre os constituintes estruturais e recursos expressivos típicos dos gêneros narrativos pretendidos, e, no caso de produção em grupo, ferramentas de escrita colaborativa.
\BNCC{EF89LP36}
 %Parodiar poemas conhecidos da literatura e criar textos em versos (como poemas concretos, ciberpoemas, haicais, liras, microrroteiros, lambe-lambes e outros tipos de poemas), explorando o uso de recursos sonoros e semânticos (como figuras de linguagem e jogos de palavras) e visuais (como relações entre imagem e texto verbal e distribuição da mancha gráfica), de forma a propiciar diferentes efeitos de sentido.
\BNCC{EF89LP37}
 %Analisar os efeitos de sentido do uso de figuras de linguagem como ironia, eufemismo, antítese, aliteração, assonância, dentre outras.
\BNCC{EF09LP01}
 %Analisar o fenômeno da disseminação de notícias falsas nas redes sociais e desenvolver estratégias para reconhecê-las, a partir da verificação/avaliação do veículo, fonte, data e local da publicação, autoria, URL, da análise da formatação, da comparação de diferentes fontes, da consulta a sites de curadoria que atestam a fidedignidade do relato dos fatos e denunciam boatos etc.
\BNCC{EF09LP02}
 %Analisar e comentar a cobertura da imprensa sobre fatos de relevância social, comparando diferentes enfoques por meio do uso de ferramentas de curadoria.
\BNCC{EF09LP03}
 %Produzir artigos de opinião, tendo em vista o contexto de produção dado, assumindo posição diante de tema polêmico, argumentando de acordo com a estrutura própria desse tipo de texto e utilizando diferentes tipos de argumentos – de autoridade, comprovação, exemplificação princípio etc.
\BNCC{EF09LP04}
 %Escrever textos corretamente, de acordo com a norma-padrão, com estruturas sintáticas complexas no nível da oração e do período.
\BNCC{EF09LP05}
 %Identificar, em textos lidos e em produções próprias, orações com a estrutura sujeito-verbo de ligação-predicativo.
\BNCC{EF09LP06}
 %Diferenciar, em textos lidos e em produções próprias, o efeito de sentido do uso dos verbos de ligação “ser”, “estar”, “ficar”, “parecer” e “permanecer”.
\BNCC{EF09LP07}
 %Comparar o uso de regência verbal e regência nominal na norma-padrão com seu uso no português brasileiro coloquial oral.
\BNCC{EF09LP08}
 %Identificar, em textos lidos e em produções próprias, a relação que conjunções (e locuções conjuntivas) coordenativas e subordinativas estabelecem entre as orações que conectam.
\BNCC{EF09LP09}
 %Identificar efeitos de sentido do uso de orações adjetivas restritivas e explicativas em um período composto.
\BNCC{EF09LP10}
 %Comparar as regras de colocação pronominal na norma-padrão com o seu uso no português brasileiro coloquial.
\BNCC{EF09LP11}
 %Inferir efeitos de sentido decorrentes do uso de recursos de coesão sequencial (conjunções e articuladores textuais).
\BNCC{EF09LP12}
 %Identificar estrangeirismos, caracterizando-os segundo a conservação, ou não, de sua forma gráfica de origem, avaliando a pertinência, ou não, de seu uso.
\BNCC{EF69AR01}
 %Pesquisar, apreciar e analisar formas distintas das artes visuais tradicionais e contemporâneas, em obras de artistas brasileiros e estrangeiros de diferentes épocas e em diferentes matrizes estéticas e culturais, de modo a ampliar a experiência com diferentes contextos e práticas artístico-visuais e cultivar a percepção, o imaginário, a capacidade de simbolizar e o repertório imagético.
\BNCC{EF69AR02}
 %Pesquisar e analisar diferentes estilos visuais, contextualizando-os no tempo e no espaço.
\BNCC{EF69AR03}
 %Analisar situações nas quais as linguagens das artes visuais se integram às linguagens audiovisuais (cinema, animações, vídeos etc.), gráficas (capas de livros, ilustrações de textos diversos etc.), cenográficas, coreográficas, musicais etc.
\BNCC{EF69AR04}
 %Analisar os elementos constitutivos das artes visuais (ponto, linha, forma, direção, cor, tom, escala, dimensão, espaço, movimento etc.) na apreciação de diferentes produções artísticas.
\BNCC{EF69AR05}
 %Experimentar e analisar diferentes formas de expressão artística (desenho, pintura, colagem, quadrinhos, dobradura, escultura, modelagem, instalação, vídeo, fotografia, performance etc.).
\BNCC{EF69AR06}
 %Desenvolver processos de criação em artes visuais, com base em temas ou interesses artísticos, de modo individual, coletivo e colaborativo, fazendo uso de materiais, instrumentos e recursos convencionais, alternativos e digitais.
\BNCC{EF69AR07}
 %Dialogar com princípios conceituais, proposições temáticas, repertórios imagéticos e processos de criação nas suas produções visuais.
\BNCC{EF69AR08}
 %Diferenciar as categorias de artista, artesão, produtor cultural, curador, designer, entre outras, estabelecendo relações entre os profissionais do sistema das artes visuais.
\BNCC{EF69AR09}
 %Pesquisar e analisar diferentes formas de expressão, representação e encenação da dança, reconhecendo e apreciando composições de dança de artistas e grupos brasileiros e estrangeiros de diferentes épocas.
\BNCC{EF69AR10}
 %Explorar elementos constitutivos do movimento cotidiano e do movimento dançado, abordando, criticamente, o desenvolvimento das formas da dança em sua história tradicional e contemporânea.
\BNCC{EF69AR11}
 %Experimentar e analisar os fatores de movimento (tempo, peso, fluência e espaço) como elementos que, combinados, geram as ações corporais e o movimento dançado.
\BNCC{EF69AR12}
 %Investigar e experimentar procedimentos de improvisação e criação do movimento como fonte para a construção de vocabulários e repertórios próprios.
\BNCC{EF69AR13}
 %Investigar brincadeiras, jogos, danças coletivas e outras práticas de dança de diferentes matrizes estéticas e culturais como referência para a criação e a composição de danças autorais, individualmente e em grupo.
\BNCC{EF69AR14}
 %Analisar e experimentar diferentes elementos (figurino, iluminação, cenário, trilha sonora etc.) e espaços (convencionais e não convencionais) para composição cênica e apresentação coreográfica.
\BNCC{EF69AR15}
 %Discutir as experiências pessoais e coletivas em dança vivenciadas na escola e em outros contextos, problematizando estereótipos e preconceitos.
\BNCC{EF69AR16}
 %Analisar criticamente, por meio da apreciação musical, usos e funções da música em seus contextos de produção e circulação, relacionando as práticas musicais às diferentes dimensões da vida social, cultural, política, histórica, econômica, estética e ética
\BNCC{EF69AR17}
 %Explorar e analisar, criticamente, diferentes meios e equipamentos culturais de circulação da música e do conhecimento musical.
\BNCC{EF69AR18}
 %Reconhecer e apreciar o papel de músicos e grupos de música brasileiros e estrangeiros que contribuíram para o desenvolvimento de formas e gêneros musicais.
\BNCC{EF69AR19}
 %Identificar e analisar diferentes estilos musicais, contextualizando-os no tempo e no espaço, de modo a aprimorar a capacidade de apreciação da estética musical.
\BNCC{EF69AR20}
 %Explorar e analisar elementos constitutivos da música (altura, intensidade, timbre, melodia, ritmo etc.), por meio de recursos tecnológicos (games e plataformas digitais), jogos, canções e práticas diversas de composição/criação, execução e apreciação musicais.
\BNCC{EF69AR21}
 %Explorar e analisar fontes e materiais sonoros em práticas de composição/criação, execução e apreciação musical, reconhecendo timbres e características de instrumentos musicais diversos.
\BNCC{EF69AR22}
 %Explorar e identificar diferentes formas de registro musical (notação musical tradicional, partituras criativas e procedimentos da música contemporânea), bem como procedimentos e técnicas de registro em áudio e audiovisual.
\BNCC{EF69AR23}
 %Explorar e criar improvisações, composições, arranjos, jingles, trilhas sonoras, entre outros, utilizando vozes, sons corporais e/ou instrumentos acústicos ou eletrônicos, convencionais ou não convencionais, expressando ideias musicais de maneira individual, coletiva e colaborativa.
\BNCC{EF69AR24}
 %Reconhecer e apreciar artistas e grupos de teatro brasileiros e estrangeiros de diferentes épocas, investigando os modos de criação, produção, divulgação, circulação e organização da atuação profissional em teatro.
\BNCC{EF69AR25}
 %Identificar e analisar diferentes estilos cênicos, contextualizando-os no tempo e no espaço de modo a aprimorar a capacidade de apreciação da estética teatral.
\BNCC{EF69AR26}
 %Explorar diferentes elementos envolvidos na composição dos acontecimentos cênicos (figurinos, adereços, cenário, iluminação e sonoplastia) e reconhecer seus vocabulários.
\BNCC{EF69AR27}
 %Pesquisar e criar formas de dramaturgias e espaços cênicos para o acontecimento teatral, em diálogo com o teatro contemporâneo.
\BNCC{EF69AR28}
 %Investigar e experimentar diferentes funções teatrais e discutir os limites e desafios do trabalho artístico coletivo e colaborativo.
\BNCC{EF69AR29}
 %Experimentar a gestualidade e as construções corporais e vocais de maneira imaginativa na improvisação teatral e no jogo cênico.
\BNCC{EF69AR30}
 %Compor improvisações e acontecimentos cênicos com base em textos dramáticos ou outros estímulos (música, imagens, objetos etc.), caracterizando personagens (com figurinos e adereços), cenário, iluminação e sonoplastia e considerando a relação com o espectador.
\BNCC{EF69AR31}
 %Relacionar as práticas artísticas às diferentes dimensões da vida social, cultural, política, histórica, econômica, estética e ética.
\BNCC{EF69AR32}
 %Analisar e explorar, em projetos temáticos, as relações processuais entre diversas linguagens artísticas.
\BNCC{EF69AR33}
 %Analisar aspectos históricos, sociais e políticos da produção artística, problematizando as narrativas eurocêntricas e as diversas categorizações da arte (arte, artesanato, folclore, design etc.).
\BNCC{EF69AR34}
 %Analisar e valorizar o patrimônio cultural, material e imaterial, de culturas diversas, em especial a brasileira, incluindo suas matrizes indígenas, africanas e europeias, de diferentes épocas, e favorecendo a construção de vocabulário e repertório relativos às diferentes linguagens artísticas.
\BNCC{EF69AR35}
 %Identificar e manipular diferentes tecnologias e recursos digitais para acessar, apreciar, produzir, registrar e compartilhar práticas e repertórios artísticos, de modo reflexivo, ético e responsável.
\BNCC{EF67EF01}
 %Experimentar e fruir, na escola e fora dela, jogos eletrônicos diversos, valorizando e respeitando os sentidos e significados atribuídos a eles por diferentes grupos sociais e etários.
\BNCC{EF67EF02}
 %Identificar as transformações nas características dos jogos eletrônicos em função dos avanços das tecnologias e nas respectivas exigências corporais colocadas por esses diferentes tipos de jogos.
\BNCC{EF67EF03}
 %Experimentar e fruir esportes de marca, precisão, invasão e técnico-combinatórios, valorizando o trabalho coletivo e o protagonismo.
\BNCC{EF67EF04}
 %Praticar um ou mais esportes de marca, precisão, invasão e técnico-combinatórios oferecidos pela escola, usando habilidades técnico-táticas básicas e respeitando regras.
\BNCC{EF67EF05}
 %Planejar e utilizar estratégias para solucionar os desafios técnicos e táticos, tanto nos esportes de marca, precisão, invasão e técnico-combinatórios como nas modalidades esportivas escolhidas para praticar de forma específica.
\BNCC{EF67EF06}
 %Analisar as transformações na organização e na prática dos esportes em suas diferentes manifestações (profissional e comunitário/lazer).
\BNCC{EF67EF07}
 %Propor e produzir alternativas para experimentação dos esportes não disponíveis e/ou acessíveis na comunidade e das demais práticas corporais tematizadas na escola.
\BNCC{EF67EF08}
 %Experimentar e fruir exercícios físicos que solicitem diferentes capacidades físicas, identificando seus tipos (força, velocidade, resistência, flexibilidade) e as sensações corporais provocadas pela sua prática.
\BNCC{EF67EF09}
 %Construir, coletivamente, procedimentos e normas de convívio que viabilizem a participação de todos na prática de exercícios físicos, com o objetivo de promover a saúde.
\BNCC{EF67EF10}
 %Diferenciar exercício físico de atividade física e propor alternativas para a prática de exercícios físicos dentro e fora do ambiente escolar.
\BNCC{EF67EF11}
 %"Experimentar, fruir e recriar danças urbanas, identificando seus elementos constitutivos (ritmo, espaço, gestos).
"
\BNCC{EF67EF12}
 %Planejar e utilizar estratégias para aprender elementos constitutivos das danças urbanas.
\BNCC{EF67EF13}
 %Diferenciar as danças urbanas das demais manifestações da dança, valorizando e respeitando os sentidos e significados atribuídos a eles por diferentes grupos sociais.
\BNCC{EF67EF14}
 %"Experimentar, fruir e recriar diferentes lutas do Brasil, valorizando a própria segurança e integridade física, bem como as dos demais.
"
\BNCC{EF67EF15}
 %Planejar e utilizar estratégias básicas das lutas do Brasil, respeitando o colega como oponente.
\BNCC{EF67EF16}
 %Identificar as características (códigos, rituais, elementos técnico-táticos, indumentária, materiais, instalações, instituições) das lutas do Brasil.
\BNCC{EF67EF17}
 %Problematizar preconceitos e estereótipos relacionados ao universo das lutas e demais práticas corporais, propondo alternativas para superá-los, com base na solidariedade, na justiça, na equidade e no respeito.
\BNCC{EF67EF18}
 %Experimentar e fruir diferentes práticas corporais de aventura urbanas, valorizando a própria segurança e integridade física, bem como as dos demais.
\BNCC{EF67EF19}
 %Identificar os riscos durante a realização de práticas corporais de aventura urbanas e planejar estratégias para sua superação.
\BNCC{EF67EF20}
 %Executar práticas corporais de aventura urbanas, respeitando o patrimônio público e utilizando alternativas para a prática segura em diversos espaços.
\BNCC{EF67EF21}
 %Identificar a origem das práticas corporais de aventura e as possibilidades de recriá-las, reconhecendo as características (instrumentos, equipamentos de segurança, indumentária, organização) e seus tipos de práticas.
\BNCC{EF89EF01}
 %"Experimentar diferentes papéis (jogador, árbitro e técnico) e fruir os esportes de rede/parede, campo e taco, invasão e combate, valorizando o trabalho coletivo e o protagonismo.
"
\BNCC{EF89EF02}
 %Praticar um ou mais esportes de rede/parede, campo e taco, invasão e combate oferecidos pela escola, usando habilidades técnico-táticas básicas.
\BNCC{EF89EF03}
 %Formular e utilizar estratégias para solucionar os desafios técnicos e táticos, tanto nos esportes de campo e taco, rede/parede, invasão e combate como nas modalidades esportivas escolhidas para praticar de forma específica.
\BNCC{EF89EF04}
 %Identificar os elementos técnicos ou técnico-táticos individuais, combinações táticas, sistemas de jogo e regras das modalidades esportivas praticadas, bem como diferenciar as modalidades esportivas com base nos critérios da lógica interna das categorias de esporte: rede/parede, campo e taco, invasão e combate.
\BNCC{EF89EF05}
 %Identificar as transformações históricas do fenômeno esportivo e discutir alguns de seus problemas (doping, corrupção, violência etc.) e a forma como as mídias os apresentam.
\BNCC{EF89EF06}
 %Verificar locais disponíveis na comunidade para a prática de esportes e das demais práticas corporais tematizadas na escola, propondo e produzindo alternativas para utilizá-los no tempo livre.
\BNCC{EF89EF07}
 %"Experimentar e fruir um ou mais programas de exercícios físicos, identificando as exigências corporais desses diferentes programas e reconhecendo a importância de uma prática individualizada, adequada às características e necessidades de cada sujeito.
"
\BNCC{EF89EF08}
 %Discutir as transformações históricas dos padrões de desempenho, saúde e beleza, considerando a forma como são apresentados nos diferentes meios (científico, midiático etc.).
\BNCC{EF89EF09}
 %Problematizar a prática excessiva de exercícios físicos e o uso de medicamentos para a ampliação do rendimento ou potencialização das transformações corporais.
\BNCC{EF89EF10}
 %Experimentar e fruir um ou mais tipos de ginástica de conscientização corporal, identificando as exigências corporais dos mesmos.
\BNCC{EF89EF11}
 %Identificar as diferenças e semelhanças entre a ginástica de conscientização corporal e as de condicionamento físico e discutir como a prática de cada uma dessas manifestações pode contribuir para a melhoria das condições de vida, saúde, bem-estar e cuidado consigo mesmo.
\BNCC{EF89EF12}
 %Experimentar, fruir e recriar danças de salão, valorizando a diversidade cultural e respeitando a tradição dessas culturas.
\BNCC{EF89EF13}
 %Planejar e utilizar estratégias para se apropriar dos elementos constitutivos (ritmo, espaço, gestos) das danças de salão.
\BNCC{EF89EF14}
 %Discutir estereótipos e preconceitos relativos às danças de salão e demais práticas corporais e propor alternativas para sua superação.
\BNCC{EF89EF15}
 %Analisar as características (ritmos, gestos, coreografias e músicas) das danças de salão, bem como suas transformações históricas e os grupos de origem.
\BNCC{EF89EF16}
 %Experimentar e fruir a execução dos movimentos pertencentes às lutas do mundo, adotando procedimentos de segurança e respeitando o oponente.
\BNCC{EF89EF17}
 %Planejar e utilizar estratégias básicas das lutas experimentadas, reconhecendo as suas características técnico-táticas.
\BNCC{EF89EF18}
 %Discutir as transformações históricas, o processo de esportivização e a midiatização de uma ou mais lutas, valorizando e respeitando as culturas de origem.
\BNCC{EF89EF19}
 %Experimentar e fruir diferentes práticas corporais de aventura na natureza, valorizando a própria segurança e integridade física, bem como as dos demais, respeitando o patrimônio natural e minimizando os impactos de degradação ambiental.
\BNCC{EF89EF20}
 %Identificar riscos, formular estratégias e observar normas de segurança para superar os desafios na realização de práticas corporais de aventura na natureza.
\BNCC{EF89EF21}
 %Identificar as características (equipamentos de segurança, instrumentos, indumentária, organização) das práticas corporais de aventura na natureza, bem como suas transformações históricas.
\BNCC{EF06LI01}
 %Interagir em situações de intercâmbio oral, demonstrando iniciativa para utilizar a língua inglesa.
\BNCC{EF06LI02}
 %Coletar informações do grupo, perguntando e respondendo sobre a família, os amigos, a escola e a comunidade..
\BNCC{EF06LI03}
 %Solicitar esclarecimentos em língua inglesa sobre o que não entendeu e o significado de palavras ou expressões desconhecidas.
\BNCC{EF06LI04}
 %Reconhecer, com o apoio de palavras cognatas e pistas do contexto discursivo, o assunto e as informações principais em textos orais sobre temas familiares.
\BNCC{EF06LI05}
 %Aplicar os conhecimentos da língua inglesa para falar de si e de outras pessoas, explicitando informações pessoais e características relacionadas a gostos, preferências e rotinas.
\BNCC{EF06LI06}
 %Planejar apresentação sobre a família, a comunidade e a escola, compartilhando-a oralmente com o grupo.
\BNCC{EF06LI07}
 %Formular hipóteses sobre a finalidade de um texto em língua inglesa, com base em sua estrutura, organização textual e pistas gráficas.
\BNCC{EF06LI08}
 %Identificar o assunto de um texto, reconhecendo sua organização textual e palavras cognatas.
\BNCC{EF06LI09}
 %Localizar informações específicas em texto.
\BNCC{EF06LI10}
 %Conhecer a organização de um dicionário bilíngue (impresso e/ou on-line) para construir repertório lexical.
\BNCC{EF06LI11}
 %Explorar ambientes virtuais e/ou aplicativos para construir repertório lexical na língua inglesa.
\BNCC{EF06LI12}
 %Interessar-se pelo texto lido, compartilhando suas ideias sobre o que o texto informa/comunica.
\BNCC{EF06LI13}
 %Listar ideias para a produção de textos, levando em conta o tema e o assunto.
\BNCC{EF06LI14}
 %Organizar ideias, selecionando-as em função da estrutura e do objetivo do texto.
\BNCC{EF06LI15}
 %Produzir textos escritos em língua inglesa (histórias em quadrinhos, cartazes, chats, blogues, agendas, fotolegendas, entre outros), sobre si mesmo, sua família, seus amigos, gostos, preferências e rotinas, sua comunidade e seu contexto escolar.
\BNCC{EF06LI16}
 %Construir repertório relativo às expressões usadas para o convívio social e o uso da língua inglesa em sala de aula.
\BNCC{EF06LI17}
 %Construir repertório lexical relativo a temas familiares (escola, família, rotina diária, atividades de lazer, esportes, entre outros).
\BNCC{EF06LI18}
 %Reconhecer semelhanças e diferenças na pronúncia de palavras da língua inglesa e da língua materna e/ou outras línguas conhecidas.
\BNCC{EF06LI19}
 %Utilizar o presente do indicativo para identificar pessoas (verbo to be) e descrever rotinas diárias.
\BNCC{EF06LI20}
 %Utilizar o presente contínuo para descrever ações em progresso.
\BNCC{EF06LI21}
 %Reconhecer o uso do imperativo em enunciados de atividades, comandos e instruções.
\BNCC{EF06LI22}
 %Descrever relações por meio do uso de apóstrofo (’) + s.
\BNCC{EF06LI23}
 %Empregar, de forma inteligível, os adjetivos possessivos.
\BNCC{EF06LI24}
 %Investigar o alcance da língua inglesa no mundo: como língua materna e/ou oficial (primeira ou segunda língua).
\BNCC{EF06LI25}
 %Identificar a presença da língua inglesa na sociedade brasileira/comunidade (palavras, expressões, suportes e esferas de circulação e consumo) e seu significado.
\BNCC{EF06LI26}
 %Avaliar, problematizando elementos/produtos culturais de países de língua inglesa absorvidos pela sociedade brasileira/comunidade.
\BNCC{EF07LI01}
 %Interagir em situações de intercâmbio oral para realizar as atividades em sala de aula, de forma respeitosa e colaborativa, trocando ideias e engajando-se em brincadeiras e jogos.
\BNCC{EF07LI02}
 %Entrevistar os colegas para conhecer suas histórias de vida.
\BNCC{EF07LI03}
 %Mobilizar conhecimentos prévios para compreender texto oral.
\BNCC{EF07LI04}
 %Identificar o contexto, a finalidade, o assunto e os interlocutores em textos orais presentes no cinema, na internet, na televisão, entre outros.
\BNCC{EF07LI05}
 %Compor, em língua inglesa, narrativas orais sobre fatos, acontecimentos e personalidades marcantes do passado.
\BNCC{EF07LI06}
 %Antecipar o sentido global de textos em língua inglesa por inferências, com base em leitura rápida, observando títulos, primeiras e últimas frases de parágrafos e palavras-chave repetidas.
\BNCC{EF07LI07}
 %Identificar a(s) informação(ões)-chave de partes de um texto em língua inglesa (parágrafos).
\BNCC{EF07LI08}
 %Relacionar as partes de um texto (parágrafos) para construir seu sentido global.
\BNCC{EF07LI09}
 %Selecionar, em um texto, a informação desejada como objetivo de leitura.
\BNCC{EF07LI10}
 %Escolher, em ambientes virtuais, textos em língua inglesa, de fontes confiáveis, para estudos/pesquisas escolares.
\BNCC{EF07LI11}
 %Participar de troca de opiniões e informações sobre textos, lidos na sala de aula ou em outros ambientes.
\BNCC{EF07LI12}
 %Planejar a escrita de textos em função do contexto (público, finalidade, layout e suporte).
\BNCC{EF07LI13}
 %Organizar texto em unidades de sentido, dividindo-o em parágrafos ou tópicos e subtópicos, explorando as possibilidades de organização gráfica, de suporte e de formato do texto.
\BNCC{EF07LI14}
 %Produzir textos diversos sobre fatos, acontecimentos e personalidades do passado (linha do tempo/ timelines, biografias, verbetes de enciclopédias, blogues, entre outros).
\BNCC{EF07LI15}
 %Construir repertório lexical relativo a verbos regulares e irregulares (formas no passado), preposições de tempo (in, on, at) e conectores (and, but, because, then, so, before, after, entre outros).
\BNCC{EF07LI16}
 %Reconhecer a pronúncia de verbos regulares no passado (-ed).
\BNCC{EF07LI17}
 %Explorar o caráter polissêmico de palavras de acordo com o contexto de uso.
\BNCC{EF07LI18}
 %Utilizar o passado simples e o passado contínuo para produzir textos orais e escritos, mostrando relações de sequência e causalidade.
\BNCC{EF07LI19}
 %Discriminar sujeito de objeto utilizando pronomes a eles relacionados.
\BNCC{EF07LI20}
 %Empregar, de forma inteligível, o verbo modal can para descrever habilidades (no presente e no passado).
\BNCC{EF07LI21}
 %Analisar o alcance da língua inglesa e os seus contextos de uso no mundo globalizado.
\BNCC{EF07LI22}
 %Explorar modos de falar em língua inglesa, refutando preconceitos e reconhecendo a variação linguística como fenômeno natural das línguas.
\BNCC{EF07LI23}
 %Reconhecer a variação linguística como manifestação de formas de pensar e expressar o mundo.
\BNCC{EF08LI01}
 %Fazer uso da língua inglesa para resolver mal-entendidos, emitir opiniões e esclarecer informações por meio de paráfrases ou justificativas.
\BNCC{EF08LI02}
 %Explorar o uso de recursos linguísticos (frases incompletas, hesitações, entre outros) e paralinguísticos (gestos, expressões faciais, entre outros) em situações de interação oral.
\BNCC{EF08LI03}
 %Construir o sentido global de textos orais, relacionando suas partes, o assunto principal e informações relevantes.
\BNCC{EF08LI04}
 %Utilizar recursos e repertório linguísticos apropriados para informar/comunicar/falar do futuro: planos, previsões, possibilidades e probabilidades.
\BNCC{EF08LI05}
 %Inferir informações e relações que não aparecem de modo explícito no texto para construção de sentidos.
\BNCC{EF08LI06}
 %Apreciar textos narrativos em língua inglesa (contos, romances, entre outros, em versão original ou simplificada), como forma de valorizar o patrimônio cultural produzido em língua inglesa.
\BNCC{EF08LI07}
 %Explorar ambientes virtuais e/ou aplicativos para acessar e usufruir do patrimônio artístico literário em língua inglesa.
\BNCC{EF08LI08}
 %Analisar, criticamente, o conteúdo de textos, comparando diferentes perspectivas apresentadas sobre um mesmo assunto.
\BNCC{EF08LI09}
 %Avaliar a própria produção escrita e a de colegas, com base no contexto de comunicação (finalidade e adequação ao público, conteúdo a ser comunicado, organização textual, legibilidade, estrutura de frases).
\BNCC{EF08LI10}
 %Reconstruir o texto, com cortes, acréscimos, reformulações e correções, para aprimoramento, edição e publicação final.
\BNCC{EF08LI11}
 %Produzir textos (comentários em fóruns, relatos pessoais, mensagens instantâneas, tweets, reportagens, histórias de ficção, blogues, entre outros), com o uso de estratégias de escrita (planejamento, produção de rascunho, revisão e edição final), apontando sonhos e projetos para o futuro (pessoal, da família, da comunidade ou do planeta).
\BNCC{EF08LI12}
 %Construir repertório lexical relativo a planos, previsões e expectativas para o futuro.
\BNCC{EF08LI13}
 %Reconhecer sufixos e prefixos comuns utilizados na formação de palavras em língua inglesa.
\BNCC{EF08LI14}
 %Utilizar formas verbais do futuro para descrever planos e expectativas e fazer previsões.
\BNCC{EF08LI15}
 %Utilizar, de modo inteligível, as formas comparativas e superlativas de adjetivos para comparar qualidades e quantidades.
\BNCC{EF08LI16}
 %Utilizar, de modo inteligível, corretamente, some, any, many, much.
\BNCC{EF08LI17}
 %Empregar, de modo inteligível, os pronomes relativos (who, which, that, whose) para construir períodos compostos por subordinação.
\BNCC{EF08LI18}
 %Construir repertório cultural por meio do contato com manifestações artístico-culturais vinculadas à língua inglesa (artes plásticas e visuais, literatura, música, cinema, dança, festividades, entre outros), valorizando a diversidade entre culturas.
\BNCC{EF08LI19}
 %Investigar de que forma expressões, gestos e comportamentos são interpretados em função de aspectos culturais.
\BNCC{EF08LI20}
 %Examinar fatores que podem impedir o entendimento entre pessoas de culturas diferentes que falam a língua inglesa.
\BNCC{EF09LI01}
 %Fazer uso da língua inglesa para expor pontos de vista, argumentos e contra-argumentos, considerando o contexto e os recursos linguísticos voltados para a eficácia da comunicação.
\BNCC{EF09LI02}
 %Compilar as ideias-chave de textos por meio de tomada de notas.
\BNCC{EF09LI03}
 %Analisar posicionamentos defendidos e refutados em textos orais sobre temas de interesse social e coletivo.
\BNCC{EF09LI04}
 %Expor resultados de pesquisa ou estudo com o apoio de recursos, tais como notas, gráficos, tabelas, entre outros, adequando as estratégias de construção do texto oral aos objetivos de comunicação e ao contexto.
\BNCC{EF09LI05}
 %Identificar recursos de persuasão (escolha e jogo de palavras, uso de cores e imagens, tamanho de letras), utilizados nos textos publicitários e de propaganda, como elementos de convencimento.
\BNCC{EF09LI06}
 %Distinguir fatos de opiniões em textos argumentativos da esfera jornalística.
\BNCC{EF09LI07}
 %Identificar argumentos principais e as evidências/exemplos que os sustentam.
\BNCC{EF09LI08}
 %Explorar ambientes virtuais de informação e socialização, analisando a qualidade e a validade das informações veiculadas.
\BNCC{EF09LI09}
 %Compartilhar, com os colegas, a leitura dos textos escritos pelo grupo, valorizando os diferentes pontos de vista defendidos, com ética e respeito.
\BNCC{EF09LI10}
 %Propor potenciais argumentos para expor e defender ponto de vista em texto escrito, refletindo sobre o tema proposto e pesquisando dados, evidências e exemplos para sustentar os argumentos, organizando-os em sequência lógica.
\BNCC{EF09LI11}
 %Utilizar recursos verbais e não verbais para construção da persuasão em textos da esfera publicitária, de forma adequada ao contexto de circulação (produção e compreensão).
\BNCC{EF09LI12}
 %Produzir textos (infográficos, fóruns de discussão on-line, fotorreportagens, campanhas publicitárias, memes, entre outros) sobre temas de interesse coletivo local ou global, que revelem posicionamento crítico.
\BNCC{EF09LI13}
 %Reconhecer, nos novos gêneros digitais (blogues, mensagens instantâneas, tweets, entre outros), novas formas de escrita (abreviação de palavras, palavras com combinação de letras e números, pictogramas, símbolos gráficos, entre outros) na constituição das mensagens.
\BNCC{EF09LI14}
 %Utilizar conectores indicadores de adição, condição, oposição, contraste, conclusão e síntese como auxiliares na construção da argumentação e intencionalidade discursiva.
\BNCC{EF09LI15}
 %Empregar, de modo inteligível, as formas verbais em orações condicionais dos tipos 1 e 2 (If-clauses).
\BNCC{EF09LI16}
 %Empregar, de modo inteligível, os verbos should, must, have to, may e might para indicar recomendação, necessidade ou obrigação e probabilidade.
\BNCC{EF09LI17}
 %Debater sobre a expansão da língua inglesa pelo mundo, em função do processo de colonização nas Américas, África, Ásia e Oceania.
\BNCC{EF09LI18}
 %Analisar a importância da língua inglesa para o desenvolvimento das ciências (produção, divulgação e discussão de novos conhecimentos), da economia e da política no cenário mundial.
\BNCC{EF09LI19}
 %Discutir a comunicação intercultural por meio da língua inglesa como mecanismo de valorização pessoal e de construção de identidades no mundo globalizado.
\BNCC{EF06MA01}
 %Comparar, ordenar, ler e escrever números naturais e números racionais cuja representação decimal é finita, fazendo uso da reta numérica.
\BNCC{EF06MA02}
 %Reconhecer o sistema de numeração decimal, como o que prevaleceu no mundo ocidental, e destacar semelhanças e diferenças com outros sistemas, de modo a sistematizar suas principais características (base, valor posicional e função do zero), utilizando, inclusive, a composição e decomposição de números naturais e números racionais em sua representação decimal.
\BNCC{EF06MA03}
 %Resolver e elaborar problemas que envolvam cálculos (mentais ou escritos, exatos ou aproximados) com números naturais, por meio de estratégias variadas, com compreensão dos processos neles envolvidos com e sem uso de calculadora.
\BNCC{EF06MA04}
 %Construir algoritmo em linguagem natural e representá-lo por fluxograma que indique a resolução de um problema simples (por exemplo, se um número natural qualquer é par).
\BNCC{EF06MA05}
 %Classificar números naturais em primos e compostos, estabelecer relações entre números, expressas pelos termos “é múltiplo de”, “é divisor de”, “é fator de”, e estabelecer, por meio de investigações, critérios de divisibilidade por 2, 3, 4, 5, 6, 8, 9, 10, 100 e 1000.
\BNCC{EF06MA06}
 %Resolver e elaborar problemas que envolvam as ideias de múltiplo e de divisor.
\BNCC{EF06MA07}
 %Compreender, comparar e ordenar frações associadas às ideias de partes de inteiros e resultado de divisão, identificando frações equivalentes.
\BNCC{EF06MA08}
 %Reconhecer que os números racionais positivos podem ser expressos nas formas fracionária e decimal, estabelecer relações entre essas representações, passando de uma representação para outra, e relacioná-los a pontos na reta numérica.
\BNCC{EF06MA09}
 %Resolver e elaborar problemas que envolvam o cálculo da fração de uma quantidade e cujo resultado seja um número natural, com e sem uso de calculadora.
\BNCC{EF06MA10}
 %Resolver e elaborar problemas que envolvam adição ou subtração com números racionais positivos na representação fracionária.
\BNCC{EF06MA11}
 %Resolver e elaborar problemas com números racionais positivos na representação decimal, envolvendo as quatro operações fundamentais e a potenciação, por meio de estratégias diversas, utilizando estimativas e arredondamentos para verificar a razoabilidade de respostas, com e sem uso de calculadora.
\BNCC{EF06MA12}
 %Fazer estimativas de quantidades e aproximar números para múltiplos da potência de 10 mais próxima.
\BNCC{EF06MA13}
 %Resolver e elaborar problemas que envolvam porcentagens, com base na ideia de proporcionalidade, sem fazer uso da “regra de três”, utilizando estratégias pessoais, cálculo mental e calculadora, em contextos de educação financeira, entre outros.
\BNCC{EF06MA14}
 %Reconhecer que a relação de igualdade matemática não se altera ao adicionar, subtrair, multiplicar ou dividir os seus dois membros por um mesmo número e utilizar essa noção para determinar valores desconhecidos na resolução de problemas.
\BNCC{EF06MA15}
 %Resolver e elaborar problemas que envolvam a partilha de uma quantidade em duas partes desiguais, envolvendo relações aditivas e multiplicativas, bem como a razão entre as partes e entre uma das partes e o todo.
\BNCC{EF06MA16}
 %Associar pares ordenados de números a pontos do plano cartesiano do 1º quadrante, em situações como a localização dos vértices de um polígono.
\BNCC{EF06MA17}
 %Quantificar e estabelecer relações entre o número de vértices, faces e arestas de prismas e pirâmides, em função do seu polígono da base, para resolver problemas e desenvolver a percepção espacial.
\BNCC{EF06MA18}
 %Reconhecer, nomear e comparar polígonos, considerando lados, vértices e ângulos, e classificá-los em regulares e não regulares, tanto em suas representações no plano como em faces de poliedros.
\BNCC{EF06MA19}
 %Identificar características dos triângulos e classificá-los em relação às medidas dos lados e dos ângulos.
\BNCC{EF06MA20}
 %Identificar características dos quadriláteros, classificá-los em relação a lados e a ângulos e reconhecer a inclusão e a intersecção de classes entre eles.
\BNCC{EF06MA21}
 %Construir figuras planas semelhantes em situações de ampliação e de redução, com o uso de malhas quadriculadas, plano cartesiano ou tecnologias digitais.
\BNCC{EF06MA22}
 %Utilizar instrumentos, como réguas e esquadros, ou softwares para representações de retas paralelas e perpendiculares e construção de quadriláteros, entre outros.
\BNCC{EF06MA23}
 %Construir algoritmo para resolver situações passo a passo (como na construção de dobraduras ou na indicação de deslocamento de um objeto no plano segundo pontos de referência e distâncias fornecidas etc.).
\BNCC{EF06MA24}
 %Resolver e elaborar problemas que envolvam as grandezas comprimento, massa, tempo, temperatura, área (triângulos e retângulos), capacidade e volume (sólidos formados por blocos retangulares), sem uso de fórmulas, inseridos, sempre que possível, em contextos oriundos de situações reais e/ou relacionadas às outras áreas do conhecimento.
\BNCC{EF06MA25}
 %Reconhecer a abertura do ângulo como grandeza associada às figuras geométricas.
\BNCC{EF06MA26}
 %Resolver problemas que envolvam a noção de ângulo em diferentes contextos e em situações reais, como ângulo de visão.
\BNCC{EF06MA27}
 %Determinar medidas da abertura de ângulos, por meio de transferidor e/ou tecnologias digitais.
\BNCC{EF06MA28}
 %Interpretar, descrever e desenhar plantas baixas simples de residências e vistas aéreas.
\BNCC{EF06MA29}
 %Analisar e descrever mudanças que ocorrem no perímetro e na área de um quadrado ao se ampliarem ou reduzirem, igualmente, as medidas de seus lados, para compreender que o perímetro é proporcional à medida do lado, o que não ocorre com a área.
\BNCC{EF06MA30}
 %Calcular a probabilidade de um evento aleatório, expressando-a por número racional (forma fracionária, decimal e percentual) e comparar esse número com a probabilidade obtida por meio de experimentos sucessivos.
\BNCC{EF06MA31}
 %Identificar as variáveis e suas frequências e os elementos constitutivos (título, eixos, legendas, fontes e datas) em diferentes tipos de gráfico.
\BNCC{EF06MA32}
 %Interpretar e resolver situações que envolvam dados de pesquisas sobre contextos ambientais, sustentabilidade, trânsito, consumo responsável, entre outros, apresentadas pela mídia em tabelas e em diferentes tipos de gráficos e redigir textos escritos com o objetivo de sintetizar conclusões.
\BNCC{EF06MA33}
 %Planejar e coletar dados de pesquisa referente a práticas sociais escolhidas pelos alunos e fazer uso de planilhas eletrônicas para registro, representação e interpretação das informações, em tabelas, vários tipos de gráficos e texto.
\BNCC{EF06MA34}
 %Interpretar e desenvolver fluxogramas simples, identificando as relações entre os objetos representados (por exemplo, posição de cidades considerando as estradas que as unem, hierarquia dos funcionários de uma empresa etc.).
\BNCC{EF07MA01}
 %Resolver e elaborar problemas com números naturais, envolvendo as noções de divisor e de múltiplo, podendo incluir máximo divisor comum ou mínimo múltiplo comum, por meio de estratégias diversas, sem a aplicação de algoritmos.
\BNCC{EF07MA02}
 %Resolver e elaborar problemas que envolvam porcentagens, como os que lidam com acréscimos e decréscimos simples, utilizando estratégias pessoais, cálculo mental e calculadora, no contexto de educação financeira, entre outros.
\BNCC{EF07MA03}
 %Comparar e ordenar números inteiros em diferentes contextos, incluindo o histórico, associá-los a pontos da reta numérica e utilizá-los em situações que envolvam adição e subtração.
\BNCC{EF07MA04}
 %Resolver e elaborar problemas que envolvam operações com números inteiros.
\BNCC{EF07MA05}
 %Resolver um mesmo problema utilizando diferentes algoritmos.
\BNCC{EF07MA06}
 %Reconhecer que as resoluções de um grupo de problemas que têm a mesma estrutura podem ser obtidas utilizando os mesmos procedimentos.
\BNCC{EF07MA07}
 %Representar por meio de um fluxograma os passos utilizados para resolver um grupo de problemas.
\BNCC{EF07MA08}
 %Comparar e ordenar frações associadas às ideias de partes de inteiros, resultado da divisão, razão e operador.
\BNCC{EF07MA09}
 %Utilizar, na resolução de problemas, a associação entre razão e fração, como a fração 2/3 para expressar a razão de duas partes de uma grandeza para três partes da mesma ou três partes de outra grandeza.
\BNCC{EF07MA10}
 %Comparar e ordenar números racionais em diferentes contextos e associá-los a pontos da reta numérica.
\BNCC{EF07MA11}
 %Compreender e utilizar a multiplicação e a divisão de números racionais, a relação entre elas e suas propriedades operatórias.
\BNCC{EF07MA12}
 %Resolver e elaborar problemas que envolvam as operações com números racionais.
\BNCC{EF07MA13}
 %Compreender a ideia de variável, representada por letra ou símbolo, para expressar relação entre duas grandezas, diferenciando-a da ideia de incógnita.
\BNCC{EF07MA14}
 %Classificar sequências em recursivas e não recursivas, reconhecendo que o conceito de recursão está presente não apenas na matemática, mas também nas artes e na literatura.
\BNCC{EF07MA15}
 %Utilizar a simbologia algébrica para expressar regularidades encontradas em sequências numéricas.
\BNCC{EF07MA16}
 %Reconhecer se duas expressões algébricas obtidas para descrever a regularidade de uma mesma sequência numérica são ou não equivalentes.
\BNCC{EF07MA17}
 %Resolver e elaborar problemas que envolvam variação de proporcionalidade direta e de proporcionalidade inversa entre duas grandezas, utilizando sentença algébrica para expressar a relação entre elas.
\BNCC{EF07MA18}
 %Resolver e elaborar problemas que possam ser representados por equações polinomiais de 1º grau, redutíveis à forma ax + b = c, fazendo uso das propriedades da igualdade.
\BNCC{EF07MA19}
 %Realizar transformações de polígonos representados no plano cartesiano, decorrentes da multiplicação das coordenadas de seus vértices por um número inteiro.
\BNCC{EF07MA20}
 %Reconhecer e representar, no plano cartesiano, o simétrico de figuras em relação aos eixos e à origem.
\BNCC{EF07MA21}
 %Reconhecer e construir figuras obtidas por simetrias de translação, rotação e reflexão, usando instrumentos de desenho ou softwares de geometria dinâmica e vincular esse estudo a representações planas de obras de arte, elementos arquitetônicos, entre outros.
\BNCC{EF07MA22}
 %Construir circunferências, utilizando compasso, reconhecê-las como lugar geométrico e utilizá-las para fazer composições artísticas e resolver problemas que envolvam objetos equidistantes.
\BNCC{EF07MA23}
 %Verificar relações entre os ângulos formados por retas paralelas cortadas por uma transversal, com e sem uso de softwares de geometria dinâmica.
\BNCC{EF07MA24}
 %Construir triângulos, usando régua e compasso, reconhecer a condição de existência do triângulo quanto à medida dos lados e verificar que a soma das medidas dos ângulos internos de um triângulo é 180°.
\BNCC{EF07MA25}
 %Reconhecer a rigidez geométrica dos triângulos e suas aplicações, como na construção de estruturas arquitetônicas (telhados, estruturas metálicas e outras) ou nas artes plásticas.
\BNCC{EF07MA26}
 %Descrever, por escrito e por meio de um fluxograma, um algoritmo para a construção de um triângulo qualquer, conhecidas as medidas dos três lados.
\BNCC{EF07MA27}
 %Calcular medidas de ângulos internos de polígonos regulares, sem o uso de fórmulas, e estabelecer relações entre ângulos internos e externos de polígonos, preferencialmente vinculadas à construção de mosaicos e de ladrilhamentos.
\BNCC{EF07MA28}
 %Descrever, por escrito e por meio de um fluxograma, um algoritmo para a construção de um polígono regular (como quadrado e triângulo equilátero), conhecida a medida de seu lado.
\BNCC{EF07MA29}
 %Resolver e elaborar problemas que envolvam medidas de grandezas inseridos em contextos oriundos de situações cotidianas ou de outras áreas do conhecimento, reconhecendo que toda medida empírica é aproximada.
\BNCC{EF07MA30}
 %Resolver e elaborar problemas de cálculo de medida do volume de blocos retangulares, envolvendo as unidades usuais (metro cúbico, decímetro cúbico e centímetro cúbico).
\BNCC{EF07MA31}
 %Estabelecer expressões de cálculo de área de triângulos e de quadriláteros.
\BNCC{EF07MA32}
 %Resolver e elaborar problemas de cálculo de medida de área de figuras planas que podem ser decompostas por quadrados, retângulos e/ou triângulos, utilizando a equivalência entre áreas.
\BNCC{EF07MA33}
 %Estabelecer o número π como a razão entre a medida de uma circunferência e seu diâmetro, para compreender e resolver problemas, inclusive os de natureza histórica.
\BNCC{EF07MA34}
 %Planejar e realizar experimentos aleatórios ou simulações que envolvem cálculo de probabilidades ou estimativas por meio de frequência de ocorrências.
\BNCC{EF07MA35}
 %Compreender, em contextos significativos, o significado de média estatística como indicador da tendência de uma pesquisa, calcular seu valor e relacioná-lo, intuitivamente, com a amplitude do conjunto de dados.
\BNCC{EF07MA36}
 %Planejar e realizar pesquisa envolvendo tema da realidade social, identificando a necessidade de ser censitária ou de usar amostra, e interpretar os dados para comunicá-los por meio de relatório escrito, tabelas e gráficos, com o apoio de planilhas eletrônicas.
\BNCC{EF07MA37}
 %Interpretar e analisar dados apresentados em gráfico de setores divulgados pela mídia e compreender quando é possível ou conveniente sua utilização.
\BNCC{EF08MA01}
 %Efetuar cálculos com potências de expoentes inteiros e aplicar esse conhecimento na representação de números em notação científica.
\BNCC{EF08MA02}
 %Resolver e elaborar problemas usando a relação entre potenciação e radiciação, para representar uma raiz como potência de expoente fracionário.
\BNCC{EF08MA03}
 %Resolver e elaborar problemas de contagem cuja resolução envolva a aplicação do princípio multiplicativo.
\BNCC{EF08MA04}
 %Resolver e elaborar problemas, envolvendo cálculo de porcentagens, incluindo o uso de tecnologias digitais.
\BNCC{EF08MA05}
 %Reconhecer e utilizar procedimentos para a obtenção de uma fração geratriz para uma dízima periódica.
\BNCC{EF08MA06}
 %Resolver e elaborar problemas que envolvam cálculo do valor numérico de expressões algébricas, utilizando as propriedades das operações.
\BNCC{EF08MA07}
 %Associar uma equação linear de 1º grau com duas incógnitas a uma reta no plano cartesiano.
\BNCC{EF08MA08}
 %Resolver e elaborar problemas relacionados ao seu contexto próximo, que possam ser representados por sistemas de equações de 1º grau com duas incógnitas e interpretá-los, utilizando, inclusive, o plano cartesiano como recurso.
\BNCC{EF08MA09}
 %Resolver e elaborar, com e sem uso de tecnologias, problemas que possam ser representados por equações polinomiais de 2º grau do tipo ax2 = b.
\BNCC{EF08MA10}
 %Identificar a regularidade de uma sequência numérica ou figural não recursiva e construir um algoritmo por meio de um fluxograma que permita indicar os números ou as figuras seguintes.
\BNCC{EF08MA11}
 %Identificar a regularidade de uma sequência numérica recursiva e construir um algoritmo por meio de um fluxograma que permita indicar os números seguintes.
\BNCC{EF08MA12}
 %Identificar a natureza da variação de duas grandezas, diretamente, inversamente proporcionais ou não proporcionais, expressando a relação existente por meio de sentença algébrica e representá-la no plano cartesiano.
\BNCC{EF08MA13}
 %Resolver e elaborar problemas que envolvam grandezas diretamente ou inversamente proporcionais, por meio de estratégias variadas.
\BNCC{EF08MA14}
 %Demonstrar propriedades de quadriláteros por meio da identificação da congruência de triângulos.
\BNCC{EF08MA15}
 %Construir, utilizando instrumentos de desenho ou softwares de geometria dinâmica, mediatriz, bissetriz, ângulos de 90°, 60°, 45° e 30° e polígonos regulares.
\BNCC{EF08MA16}
 %Descrever, por escrito e por meio de um fluxograma, um algoritmo para a construção de um hexágono regular de qualquer área, a partir da medida do ângulo central e da utilização de esquadros e compasso.
\BNCC{EF08MA17}
 %Aplicar os conceitos de mediatriz e bissetriz como lugares geométricos na resolução de problemas.
\BNCC{EF08MA18}
 %Reconhecer e construir figuras obtidas por composições de transformações geométricas (translação, reflexão e rotação), com o uso de instrumentos de desenho ou de softwares de geometria dinâmica.
\BNCC{EF08MA19}
 %Resolver e elaborar problemas que envolvam medidas de área de figuras geométricas, utilizando expressões de cálculo de área (quadriláteros, triângulos e círculos), em situações como determinar medida de terrenos.
\BNCC{EF08MA20}
 %Reconhecer a relação entre um litro e um decímetro cúbico e a relação entre litro e metro cúbico, para resolver problemas de cálculo de capacidade de recipientes.
\BNCC{EF08MA21}
 %Resolver e elaborar problemas que envolvam o cálculo do volume de recipiente cujo formato é o de um bloco retangular.
\BNCC{EF08MA22}
 %Calcular a probabilidade de eventos, com base na construção do espaço amostral, utilizando o princípio multiplicativo, e reconhecer que a soma das probabilidades de todos os elementos do espaço amostral é igual a 1.
\BNCC{EF08MA23}
 %Avaliar a adequação de diferentes tipos de gráficos para representar um conjunto de dados de uma pesquisa.
\BNCC{EF08MA24}
 %Classificar as frequências de uma variável contínua de uma pesquisa em classes, de modo que resumam os dados de maneira adequada para a tomada de decisões.
\BNCC{EF08MA25}
 %Obter os valores de medidas de tendência central de uma pesquisa estatística (média, moda e mediana) com a compreensão de seus significados e relacioná-los com a dispersão de dados, indicada pela amplitude.
\BNCC{EF08MA26}
 %Selecionar razões, de diferentes naturezas (física, ética ou econômica), que justificam a realização de pesquisas amostrais e não censitárias, e reconhecer que a seleção da amostra pode ser feita de diferentes maneiras (amostra casual simples, sistemática e estratificada).
\BNCC{EF08MA27}
 %Planejar e executar pesquisa amostral, selecionando uma técnica de amostragem adequada, e escrever relatório que contenha os gráficos apropriados para representar os conjuntos de dados, destacando aspectos como as medidas de tendência central, a amplitude e as conclusões.
\BNCC{EF09MA01}
 %Reconhecer que, uma vez fixada uma unidade de comprimento, existem segmentos de reta cujo comprimento não é expresso por número racional (como as medidas de diagonais de um polígono e alturas de um triângulo, quando se toma a medida de cada lado como unidade).
\BNCC{EF09MA02}
 %Reconhecer um número irracional como um número real cuja representação decimal é infinita e não periódica, e estimar a localização de alguns deles na reta numérica.
\BNCC{EF09MA03}
 %Efetuar cálculos com números reais, inclusive potências com expoentes fracionários.
\BNCC{EF09MA04}
 %Resolver e elaborar problemas com números reais, inclusive em notação científica, envolvendo diferentes operações.
\BNCC{EF09MA05}
 %Resolver e elaborar problemas que envolvam porcentagens, com a ideia de aplicação de percentuais sucessivos e a determinação das taxas percentuais, preferencialmente com o uso de tecnologias digitais, no contexto da educação financeira.
\BNCC{EF09MA06}
 %Compreender as funções como relações de dependência unívoca entre duas variáveis e suas representações numérica, algébrica e gráfica e utilizar esse conceito para analisar situações que envolvam relações funcionais entre duas variáveis.
\BNCC{EF09MA07}
 %Resolver problemas que envolvam a razão entre duas grandezas de espécies diferentes, como velocidade e densidade demográfica.
\BNCC{EF09MA08}
 %Resolver e elaborar problemas que envolvam relações de proporcionalidade direta e inversa entre duas ou mais grandezas, inclusive escalas, divisão em partes proporcionais e taxa de variação, em contextos socioculturais, ambientais e de outras áreas.
\BNCC{EF09MA09}
 %Compreender os processos de fatoração de expressões algébricas, com base em suas relações com os produtos notáveis, para resolver e elaborar problemas que possam ser representados por equações polinomiais do 2º grau.
\BNCC{EF09MA10}
 %Demonstrar relações simples entre os ângulos formados por retas paralelas cortadas por uma transversal.
\BNCC{EF09MA11}
 %Resolver problemas por meio do estabelecimento de relações entre arcos, ângulos centrais e ângulos inscritos na circunferência, fazendo uso, inclusive, de softwares de geometria dinâmica.
\BNCC{EF09MA12}
 %Reconhecer as condições necessárias e suficientes para que dois triângulos sejam semelhantes.
\BNCC{EF09MA13}
 %Demonstrar relações métricas do triângulo retângulo, entre elas o teorema de Pitágoras, utilizando, inclusive, a semelhança de triângulos.
\BNCC{EF09MA14}
 %Resolver e elaborar problemas de aplicação do teorema de Pitágoras ou das relações de proporcionalidade envolvendo retas paralelas cortadas por secantes.
\BNCC{EF09MA15}
 %Descrever, por escrito e por meio de um fluxograma, um algoritmo para a construção de um polígono regular cuja medida do lado é conhecida, utilizando régua e compasso, como também softwares.
\BNCC{EF09MA16}
 %Determinar o ponto médio de um segmento de reta e a distância entre dois pontos quaisquer, dadas as coordenadas desses pontos no plano cartesiano, sem o uso de fórmulas, e utilizar esse conhecimento para calcular, por exemplo, medidas de perímetros e áreas de figuras planas construídas no plano.
\BNCC{EF09MA17}
 %Reconhecer vistas ortogonais de figuras espaciais e aplicar esse conhecimento para desenhar objetos em perspectiva.
\BNCC{EF09MA18}
 %Reconhecer e empregar unidades usadas para expressar medidas muito grandes ou muito pequenas, tais como distância entre planetas e sistemas solares, tamanho de vírus ou de células, capacidade de armazenamento de computadores, entre outros.
\BNCC{EF09MA19}
 %Resolver e elaborar problemas que envolvam medidas de volumes de prismas e de cilindros retos, inclusive com uso de expressões de cálculo, em situações cotidianas.
\BNCC{EF09MA20}
 %Reconhecer, em experimentos aleatórios, eventos independentes e dependentes e calcular a probabilidade de sua ocorrência, nos dois casos.
\BNCC{EF09MA21}
 %Analisar e identificar, em gráficos divulgados pela mídia, os elementos que podem induzir, às vezes propositadamente, erros de leitura, como escalas inapropriadas, legendas não explicitadas corretamente, omissão de informações importantes (fontes e datas), entre outros.
\BNCC{EF09MA22}
 %Escolher e construir o gráfico mais adequado (colunas, setores, linhas), com ou sem uso de planilhas eletrônicas, para apresentar um determinado conjunto de dados, destacando aspectos como as medidas de tendência central.
\BNCC{EF09MA23}
 %Planejar e executar pesquisa amostral envolvendo tema da realidade social e comunicar os resultados por meio de relatório contendo avaliação de medidas de tendência central e da amplitude, tabelas e gráficos adequados, construídos com o apoio de planilhas eletrônicas.
\BNCC{EF06CI01}
 %Classificar como homogênea ou heterogênea a mistura de dois ou mais materiais (água e sal, água e óleo, água e areia etc.).
\BNCC{EF06CI02}
 %Identificar evidências de transformações químicas a partir do resultado de misturas de materiais que originam produtos diferentes dos que foram misturados (mistura de ingredientes para fazer um bolo, mistura de vinagre com bicarbonato de sódio etc.).
\BNCC{EF06CI03}
 %Selecionar métodos mais adequados para a separação de diferentes sistemas heterogêneos a partir da identificação de processos de separação de materiais (como a produção de sal de cozinha, a destilação de petróleo, entre outros).
\BNCC{EF06CI04}
 %Associar a produção de medicamentos e outros materiais sintéticos ao desenvolvimento científico e tecnológico, reconhecendo benefícios e avaliando impactos socioambientais.
\BNCC{EF06CI05}
 %Explicar a organização básica das células e seu papel como unidade estrutural e funcional dos seres vivos.
\BNCC{EF06CI06}
 %Concluir, com base na análise de ilustrações e/ou modelos (físicos ou digitais), que os organismos são um complexo arranjo de sistemas com diferentes níveis de organização.
\BNCC{EF06CI07}
 %Justificar o papel do sistema nervoso na coordenação das ações motoras e sensoriais do corpo, com base na análise de suas estruturas básicas e respectivas funções.
\BNCC{EF06CI08}
 %Explicar a importância da visão (captação e interpretação das imagens) na interação do organismo com o meio e, com base no funcionamento do olho humano, selecionar lentes adequadas para a correção de diferentes defeitos da visão.
\BNCC{EF06CI09}
 %Deduzir que a estrutura, a sustentação e a movimentação dos animais resultam da interação entre os sistemas muscular, ósseo e nervoso.
\BNCC{EF06CI10}
 %Explicar como o funcionamento do sistema nervoso pode ser afetado por substâncias psicoativas.
\BNCC{EF06CI11}
 %Identificar as diferentes camadas que estruturam o planeta Terra (da estrutura interna à atmosfera) e suas principais características.
\BNCC{EF06CI12}
 %Identificar diferentes tipos de rocha, relacionando a formação de fósseis a rochas sedimentares em diferentes períodos geológicos.
\BNCC{EF06CI13}
 %Selecionar argumentos e evidências que demonstrem a esfericidade da Terra.
\BNCC{EF06CI14}
 %Inferir que as mudanças na sombra de uma vara (gnômon) ao longo do dia em diferentes períodos do ano são uma evidência dos movimentos relativos entre a Terra e o Sol, que podem ser explicados por meio dos movimentos de rotação e translação da Terra e da inclinação de seu eixo de rotação em relação ao plano de sua órbita em torno do Sol.
\BNCC{EF07CI01}
 %Discutir a aplicação, ao longo da história, das máquinas simples e propor soluções e invenções para a realização de tarefas mecânicas cotidianas.
\BNCC{EF07CI02}
 %Diferenciar temperatura, calor e sensação térmica nas diferentes situações de equilíbrio termodinâmico cotidianas.
\BNCC{EF07CI03}
 %Utilizar o conhecimento das formas de propagação do calor para justificar a utilização de determinados materiais (condutores e isolantes) na vida cotidiana, explicar o princípio de funcionamento de alguns equipamentos (garrafa térmica, coletor solar etc.) e/ou construir soluções tecnológicas a partir desse conhecimento.
\BNCC{EF07CI04}
 %Avaliar o papel do equilíbrio termodinâmico para a manutenção da vida na Terra, para o funcionamento de máquinas térmicas e em outras situações cotidianas.
\BNCC{EF07CI05}
 %Discutir o uso de diferentes tipos de combustível e máquinas térmicas ao longo do tempo, para avaliar avanços, questões econômicas e problemas socioambientais causados pela produção e uso desses materiais e máquinas.
\BNCC{EF07CI06}
 %Discutir e avaliar mudanças econômicas, culturais e sociais, tanto na vida cotidiana quanto no mundo do trabalho, decorrentes do desenvolvimento de novos materiais e tecnologias (como automação e informatização).
\BNCC{EF07CI07}
 %Caracterizar os principais ecossistemas brasileiros quanto à paisagem, à quantidade de água, ao tipo de solo, à disponibilidade de luz solar, à temperatura etc., correlacionando essas características à flora e fauna específicas.
\BNCC{EF07CI08}
 %Avaliar como os impactos provocados por catástrofes naturais ou mudanças nos componentes físicos, biológicos ou sociais de um ecossistema afetam suas populações, podendo ameaçar ou provocar a extinção de espécies, alteração de hábitos, migração etc.
\BNCC{EF07CI09}
 %Interpretar as condições de saúde da comunidade, cidade ou estado, com base na análise e comparação de indicadores de saúde (como taxa de mortalidade infantil, cobertura de saneamento básico e incidência de doenças de veiculação hídrica, atmosférica entre outras) e dos resultados de políticas públicas destinadas à saúde.
\BNCC{EF07CI10}
 %Argumentar sobre a importância da vacinação para a saúde pública, com base em informações sobre a maneira como a vacina atua no organismo e o papel histórico da vacinação para a manutenção da saúde individual e coletiva e para a erradicação de doenças.
\BNCC{EF07CI11}
 %Analisar historicamente o uso da tecnologia, incluindo a digital, nas diferentes dimensões da vida humana, considerando indicadores ambientais e de qualidade de vida.
\BNCC{EF07CI12}
 %Demonstrar que o ar é uma mistura de gases, identificando sua composição, e discutir fenômenos naturais ou antrópicos que podem alterar essa composição.
\BNCC{EF07CI13}
 %Descrever o mecanismo natural do efeito estufa, seu papel fundamental para o desenvolvimento da vida na Terra, discutir as ações humanas responsáveis pelo seu aumento artificial (queima dos combustíveis fósseis, desmatamento, queimadas etc.) e selecionar e implementar propostas para a reversão ou controle desse quadro.
\BNCC{EF07CI14}
 %Justificar a importância da camada de ozônio para a vida na Terra, identificando os fatores que aumentam ou diminuem sua presença na atmosfera, e discutir propostas individuais e coletivas para sua preservação.
\BNCC{EF07CI15}
 %Interpretar fenômenos naturais (como vulcões, terremotos e tsunamis) e justificar a rara ocorrência desses fenômenos no Brasil, com base no modelo das placas tectônicas.
\BNCC{EF07CI16}
 %Justificar o formato das costas brasileira e africana com base na teoria da deriva dos continentes.
\BNCC{EF08CI01}
 %Identificar e classificar diferentes fontes (renováveis e não renováveis) e tipos de energia utilizados em residências, comunidades ou cidades.
\BNCC{EF08CI02}
 %Construir circuitos elétricos com pilha/bateria, fios e lâmpada ou outros dispositivos e compará-los a circuitos elétricos residenciais.
\BNCC{EF08CI03}
 %Classificar equipamentos elétricos residenciais (chuveiro, ferro, lâmpadas, TV, rádio, geladeira etc.) de acordo com o tipo de transformação de energia (da energia elétrica para a térmica, luminosa, sonora e mecânica, por exemplo).
\BNCC{EF08CI04}
 %Calcular o consumo de eletrodomésticos a partir dos dados de potência (descritos no próprio equipamento) e tempo médio de uso para avaliar o impacto de cada equipamento no consumo doméstico mensal.
\BNCC{EF08CI05}
 %Propor ações coletivas para otimizar o uso de energia elétrica em sua escola e/ou comunidade, com base na seleção de equipamentos segundo critérios de sustentabilidade (consumo de energia e eficiência energética) e hábitos de consumo responsável.
\BNCC{EF08CI06}
 %Discutir e avaliar usinas de geração de energia elétrica (termelétricas, hidrelétricas, eólicas etc.), suas semelhanças e diferenças, seus impactos socioambientais, e como essa energia chega e é usada em sua cidade, comunidade, casa ou escola.
\BNCC{EF08CI07}
 %Comparar diferentes processos reprodutivos em plantas e animais em relação aos mecanismos adaptativos e evolutivos.
\BNCC{EF08CI08}
 %Analisar e explicar as transformações que ocorrem na puberdade considerando a atuação dos hormônios sexuais e do sistema nervoso.
\BNCC{EF08CI09}
 %Comparar o modo de ação e a eficácia dos diversos métodos contraceptivos e justificar a necessidade de compartilhar a responsabilidade na escolha e na utilização do método mais adequado à prevenção da gravidez precoce e indesejada e de Doenças Sexualmente Transmissíveis (DST).
\BNCC{EF08CI10}
 %Identificar os principais sintomas, modos de transmissão e tratamento de algumas DST (com ênfase na AIDS), e discutir estratégias e métodos de prevenção.
\BNCC{EF08CI11}
 %Selecionar argumentos que evidenciem as múltiplas dimensões da sexualidade humana (biológica, sociocultural, afetiva e ética).
\BNCC{EF08CI12}
 %Justificar, por meio da construção de modelos e da observação da Lua no céu, a ocorrência das fases da Lua e dos eclipses, com base nas posições relativas entre Sol, Terra e Lua.
\BNCC{EF08CI13}
 %Representar os movimentos de rotação e translação da Terra e analisar o papel da inclinação do eixo de rotação da Terra em relação à sua órbita na ocorrência das estações do ano, com a utilização de modelos tridimensionais.
\BNCC{EF08CI14}
 %Relacionar climas regionais aos padrões de circulação atmosférica e oceânica e ao aquecimento desigual causado pela forma e pelos movimentos da Terra.
\BNCC{EF08CI15}
 %Identificar as principais variáveis envolvidas na previsão do tempo e simular situações nas quais elas possam ser medidas.
\BNCC{EF08CI16}
 %Discutir iniciativas que contribuam para restabelecer o equilíbrio ambiental a partir da identificação de alterações climáticas regionais e globais provocadas pela intervenção humana.
\BNCC{EF09CI01}
 %Investigar as mudanças de estado físico da matéria e explicar essas transformações com base no modelo de constituição submicroscópica.
\BNCC{EF09CI02}
 %Comparar quantidades de reagentes e produtos envolvidos em transformações químicas, estabelecendo a proporção entre as suas massas.
\BNCC{EF09CI03}
 %Identificar modelos que descrevem a estrutura da matéria (constituição do átomo e composição de moléculas simples) e reconhecer sua evolução histórica.
\BNCC{EF09CI04}
 %Planejar e executar experimentos que evidenciem que todas as cores de luz podem ser formadas pela composição das três cores primárias da luz e que a cor de um objeto está relacionada também à cor da luz que o ilumina.
\BNCC{EF09CI05}
 %Investigar os principais mecanismos envolvidos na transmissão e recepção de imagem e som que revolucionaram os sistemas de comunicação humana.
\BNCC{EF09CI06}
 %Classificar as radiações eletromagnéticas por suas frequências, fontes e aplicações, discutindo e avaliando as implicações de seu uso em controle remoto, telefone celular, raio X, forno de micro-ondas, fotocélulas etc.
\BNCC{EF09CI07}
 %Discutir o papel do avanço tecnológico na aplicação das radiações na medicina diagnóstica (raio X, ultrassom, ressonância nuclear magnética) e no tratamento de doenças (radioterapia, cirurgia ótica a laser, infravermelho, ultravioleta etc.).
\BNCC{EF09CI08}
 %Associar os gametas à transmissão das características hereditárias, estabelecendo relações entre ancestrais e descendentes.
\BNCC{EF09CI09}
 %Discutir as ideias de Mendel sobre hereditariedade (fatores hereditários, segregação, gametas, fecundação), considerando-as para resolver problemas envolvendo a transmissão de características hereditárias em diferentes organismos.
\BNCC{EF09CI10}
 %Comparar as ideias evolucionistas de Lamarck e Darwin apresentadas em textos científicos e históricos, identificando semelhanças e diferenças entre essas ideias e sua importância para explicar a diversidade biológica.
\BNCC{EF09CI11}
 %Discutir a evolução e a diversidade das espécies com base na atuação da seleção natural sobre as variantes de uma mesma espécie, resultantes de processo reprodutivo.
\BNCC{EF09CI12}
 %Justificar a importância das unidades de conservação para a preservação da biodiversidade e do patrimônio nacional, considerando os diferentes tipos de unidades (parques, reservas e florestas nacionais), as populações humanas e as atividades a eles relacionados.
\BNCC{EF09CI13}
 %Propor iniciativas individuais e coletivas para a solução de problemas ambientais da cidade ou da comunidade, com base na análise de ações de consumo consciente e de sustentabilidade bem-sucedidas.
\BNCC{EF09CI14}
 %Descrever a composição e a estrutura do Sistema Solar (Sol, planetas rochosos, planetas gigantes gasosos e corpos menores), assim como a localização do Sistema Solar na nossa Galáxia (a Via Láctea) e dela no Universo (apenas uma galáxia dentre bilhões).
\BNCC{EF09CI15}
 %Relacionar diferentes leituras do céu e explicações sobre a origem da Terra, do Sol ou do Sistema Solar às necessidades de distintas culturas (agricultura, caça, mito, orientação espacial e temporal etc.).
\BNCC{EF09CI16}
 %Selecionar argumentos sobre a viabilidade da sobrevivência humana fora da Terra, com base nas condições necessárias à vida, nas características dos planetas e nas distâncias e nos tempos envolvidos em viagens interplanetárias e interestelares.
\BNCC{EF09CI17}
 %Analisar o ciclo evolutivo do Sol (nascimento, vida e morte) baseado no conhecimento das etapas de evolução de estrelas de diferentes dimensões e os efeitos desse processo no nosso planeta.
\BNCC{EF06GE01}
 %Comparar modificações das paisagens nos lugares de vivência e os usos desses lugares em diferentes tempos.
\BNCC{EF06GE02}
 %Analisar modificações de paisagens por diferentes tipos de sociedade, com destaque para os povos originários.
\BNCC{EF06GE03}
 %Descrever os movimentos do planeta e sua relação com a circulação geral da atmosfera, o tempo atmosférico e os padrões climáticos.
\BNCC{EF06GE04}
 %Descrever o ciclo da água, comparando o escoamento superficial no ambiente urbano e rural, reconhecendo os principais componentes da morfologia das bacias e das redes hidrográficas e a sua localização no modelado da superfície terrestre e da cobertura vegetal.
\BNCC{EF06GE05}
 %Relacionar padrões climáticos, tipos de solo, relevo e formações vegetais.
\BNCC{EF06GE06}
 %Identificar as características das paisagens transformadas pelo trabalho humano a partir do desenvolvimento da agropecuária e do processo de industrialização.
\BNCC{EF06GE07}
 %Explicar as mudanças na interação humana com a natureza a partir do surgimento das cidades.
\BNCC{EF06GE08}
 %Medir distâncias na superfície pelas escalas gráficas e numéricas dos mapas.
\BNCC{EF06GE09}
 %Elaborar modelos tridimensionais, blocos-diagramas e perfis topográficos e de vegetação, visando à representação de elementos e estruturas da superfície terrestre.
\BNCC{EF06GE10}
 %Explicar as diferentes formas de uso do solo (rotação de terras, terraceamento, aterros etc.) e de apropriação dos recursos hídricos (sistema de irrigação, tratamento e redes de distribuição), bem como suas vantagens e desvantagens em diferentes épocas e lugares.
\BNCC{EF06GE11}
 %Analisar distintas interações das sociedades com a natureza, com base na distribuição dos componentes físico-naturais, incluindo as transformações da biodiversidade local e do mundo.
\BNCC{EF06GE12}
 %Identificar o consumo dos recursos hídricos e o uso das principais bacias hidrográficas no Brasil e no mundo, enfatizando as transformações nos ambientes urbanos.
\BNCC{EF06GE13}
 %Analisar consequências, vantagens e desvantagens das práticas humanas na dinâmica climática (ilha de calor etc.).
\BNCC{EF07GE01}
 %Avaliar, por meio de exemplos extraídos dos meios de comunicação, ideias e estereótipos acerca das paisagens e da formação territorial do Brasil.
\BNCC{EF07GE02}
 %Analisar a influência dos fluxos econômicos e populacionais na formação socioeconômica e territorial do Brasil, compreendendo os conflitos e as tensões históricas e contemporâneas.
\BNCC{EF07GE03}
 %Selecionar argumentos que reconheçam as territorialidades dos povos indígenas originários, das comunidades remanescentes de quilombos, de povos das florestas e do cerrado, de ribeirinhos e caiçaras, entre outros grupos sociais do campo e da cidade, como direitos legais dessas comunidades.
\BNCC{EF07GE04}
 %Analisar a distribuição territorial da população brasileira, considerando a diversidade étnico-cultural (indígena, africana, europeia e asiática), assim como aspectos de renda, sexo e idade nas regiões brasileiras.
\BNCC{EF07GE05}
 %Analisar fatos e situações representativas das alterações ocorridas entre o período mercantilista e o advento do capitalismo.
\BNCC{EF07GE06}
 %Discutir em que medida a produção, a circulação e o consumo de mercadorias provocam impactos ambientais, assim como influem na distribuição de riquezas, em diferentes lugares.
\BNCC{EF07GE07}
 %Analisar a influência e o papel das redes de transporte e comunicação na configuração do território brasileiro.
\BNCC{EF07GE08}
 %Estabelecer relações entre os processos de industrialização e inovação tecnológica com as transformações socioeconômicas do território brasileiro.
\BNCC{EF07GE09}
 %Interpretar e elaborar mapas temáticos e históricos, inclusive utilizando tecnologias digitais, com informações demográficas e econômicas do Brasil (cartogramas), identificando padrões espaciais, regionalizações e analogias espaciais.
\BNCC{EF07GE10}
 %Elaborar e interpretar gráficos de barras, gráficos de setores e histogramas, com base em dados socioeconômicos das regiões brasileiras.
\BNCC{EF07GE11}
 %Caracterizar dinâmicas dos componentes físico-naturais no território nacional, bem como sua distribuição e biodiversidade (Florestas Tropicais, Cerrados, Caatingas, Campos Sulinos e Matas de Araucária).
\BNCC{EF07GE12}
 %Comparar unidades de conservação existentes no Município de residência e em outras localidades brasileiras, com base na organização do Sistema Nacional de Unidades de Conservação (SNUC).
\BNCC{EF08GE01}
 %Descrever as rotas de dispersão da população humana pelo planeta e os principais fluxos migratórios em diferentes períodos da história, discutindo os fatores históricos e condicionantes físico-naturais associados à distribuição da população humana pelos continentes.
\BNCC{EF08GE02}
 %Relacionar fatos e situações representativas da história das famílias do Município em que se localiza a escola, considerando a diversidade e os fluxos migratórios da população mundial.
\BNCC{EF08GE03}
 %Analisar aspectos representativos da dinâmica demográfica, considerando características da população (perfil etário, crescimento vegetativo e mobilidade espacial).
\BNCC{EF08GE04}
 %Compreender os fluxos de migração na América Latina (movimentos voluntários e forçados, assim como fatores e áreas de expulsão e atração) e as principais políticas migratórias da região.
\BNCC{EF08GE05}
 %Aplicar os conceitos de Estado, nação, território, governo e país para o entendimento de conflitos e tensões na contemporaneidade, com destaque para as situações geopolíticas na América e na África e suas múltiplas regionalizações a partir do pós-guerra.
\BNCC{EF08GE06}
 %Analisar a atuação das organizações mundiais nos processos de integração cultural e econômica nos contextos americano e africano, reconhecendo, em seus lugares de vivência, marcas desses processos.
\BNCC{EF08GE07}
 %Analisar os impactos geoeconômicos, geoestratégicos e geopolíticos da ascensão dos Estados Unidos da América no cenário internacional em sua posição de liderança global e na relação com a China e o Brasil.
\BNCC{EF08GE08}
 %Analisar a situação do Brasil e de outros países da América Latina e da África, assim como da potência estadunidense na ordem mundial do pós-guerra.
\BNCC{EF08GE09}
 %Analisar os padrões econômicos mundiais de produção, distribuição e intercâmbio dos produtos agrícolas e industrializados, tendo como referência os Estados Unidos da América e os países denominados de Brics (Brasil, Rússia, Índia, China e África do Sul).
\BNCC{EF08GE10}
 %Distinguir e analisar conflitos e ações dos movimentos sociais brasileiros, no campo e na cidade, comparando com outros movimentos sociais existentes nos países latino-americanos.
\BNCC{EF08GE11}
 %Analisar áreas de conflito e tensões nas regiões de fronteira do continente latino-americano e o papel de organismos internacionais e regionais de cooperação nesses cenários.
\BNCC{EF08GE12}
 %Compreender os objetivos e analisar a importância dos organismos de integração do território americano (Mercosul, OEA, OEI, Nafta, Unasul, Alba, Comunidade Andina, Aladi, entre outros).
\BNCC{EF08GE13}
 %Analisar a influência do desenvolvimento científico e tecnológico na caracterização dos tipos de trabalho e na economia dos espaços urbanos e rurais da América e da África.
\BNCC{EF08GE14}
 %Analisar os processos de desconcentração, descentralização e recentralização das atividades econômicas a partir do capital estadunidense e chinês em diferentes regiões no mundo, com destaque para o Brasil.
\BNCC{EF08GE15}
 %Analisar a importância dos principais recursos hídricos da America Latina (Aquífero Guarani, Bacias do rio da Prata, do Amazonas e do Orinoco, sistemas de nuvens na Amazônia e nos Andes, entre outros) e discutir os desafios relacionados à gestão e comercialização da água.
\BNCC{EF08GE16}
 %Analisar as principais problemáticas comuns às grandes cidades latino-americanas, particularmente aquelas relacionadas à distribuição, estrutura e dinâmica da população e às condições de vida e trabalho.
\BNCC{EF08GE17}
 %Analisar a segregação socioespacial em ambientes urbanos da América Latina, com atenção especial ao estudo de favelas, alagados e zona de riscos.
\BNCC{EF08GE18}
 %Elaborar mapas ou outras formas de representação cartográfica para analisar as redes e as dinâmicas urbanas e rurais, ordenamento territorial, contextos culturais, modo de vida e usos e ocupação de solos da África e América.
\BNCC{EF08GE19}
 %Interpretar cartogramas, mapas esquemáticos (croquis) e anamorfoses geográficas com informações geográficas acerca da África e América.
\BNCC{EF08GE20}
 %Analisar características de países e grupos de países da América e da África no que se refere aos aspectos populacionais, urbanos, políticos e econômicos, e discutir as desigualdades sociais e econômicas e as pressões sobre a natureza e suas riquezas (sua apropriação e valoração na produção e circulação), o que resulta na espoliação desses povos.
\BNCC{EF08GE21}
 %Analisar o papel ambiental e territorial da Antártica no contexto geopolítico, sua relevância para os países da América do Sul e seu valor como área destinada à pesquisa e à compreensão do ambiente global.
\BNCC{EF08GE22}
 %Identificar os principais recursos naturais dos países da América Latina, analisando seu uso para a produção de matéria-prima e energia e sua relevância para a cooperação entre os países do Mercosul.
\BNCC{EF08GE23}
 %Identificar paisagens da América Latina e associá-las, por meio da cartografia, aos diferentes povos da região, com base em aspectos da geomorfologia, da biogeografia e da climatologia.
\BNCC{EF08GE24}
 %Analisar as principais características produtivas dos países latino-americanos (como exploração mineral na Venezuela; agricultura de alta especialização e exploração mineira no Chile; circuito da carne nos pampas argentinos e no Brasil; circuito da cana-de-açúcar em Cuba; polígono industrial do sudeste brasileiro e plantações de soja no centro-oeste; maquiladoras mexicanas, entre outros).
\BNCC{EF09GE01}
 %Analisar criticamente de que forma a hegemonia europeia foi exercida em várias regiões do planeta, notadamente em situações de conflito, intervenções militares e/ou influência cultural em diferentes tempos e lugares.
\BNCC{EF09GE02}
 %Analisar a atuação das corporações internacionais e das organizações econômicas mundiais na vida da população em relação ao consumo, à cultura e à mobilidade.
\BNCC{EF09GE03}
 %Identificar diferentes manifestações culturais de minorias étnicas como forma de compreender a multiplicidade cultural na escala mundial, defendendo o princípio do respeito às diferenças.
\BNCC{EF09GE04}
 %Relacionar diferenças de paisagens aos modos de viver de diferentes povos na Europa, Ásia e Oceania, valorizando identidades e interculturalidades regionais.
\BNCC{EF09GE05}
 %Analisar fatos e situações para compreender a integração mundial (econômica, política e cultural), comparando as diferentes interpretações: globalização e mundialização.
\BNCC{EF09GE06}
 %Associar o critério de divisão do mundo em Ocidente e Oriente com o Sistema Colonial implantado pelas potências europeias.
\BNCC{EF09GE07}
 %Analisar os componentes físico-naturais da Eurásia e os determinantes histórico-geográficos de sua divisão em Europa e Ásia.
\BNCC{EF09GE08}
 %Analisar transformações territoriais, considerando o movimento de fronteiras, tensões, conflitos e múltiplas regionalidades na Europa, na Ásia e na Oceania.
\BNCC{EF09GE09}
 %Analisar características de países e grupos de países europeus, asiáticos e da Oceania em seus aspectos populacionais, urbanos, políticos e econômicos, e discutir suas desigualdades sociais e econômicas e pressões sobre seus ambientes físico-naturais.
\BNCC{EF09GE10}
 %Analisar os impactos do processo de industrialização na produção e circulação de produtos e culturas na Europa, na Ásia e na Oceania.
\BNCC{EF09GE11}
 %Relacionar as mudanças técnicas e científicas decorrentes do processo de industrialização com as transformações no trabalho em diferentes regiões do mundo e suas consequências no Brasil.
\BNCC{EF09GE12}
 %Relacionar o processo de urbanização às transformações da produção agropecuária, à expansão do desemprego estrutural e ao papel crescente do capital financeiro em diferentes países, com destaque para o Brasil.
\BNCC{EF09GE13}
 %Analisar a importância da produção agropecuária na sociedade urbano-industrial ante o problema da desigualdade mundial de acesso aos recursos alimentares e à matéria-prima.
\BNCC{EF09GE14}
 %Elaborar e interpretar gráficos de barras e de setores, mapas temáticos e esquemáticos (croquis) e anamorfoses geográficas para analisar, sintetizar e apresentar dados e informações sobre diversidade, diferenças e desigualdades sociopolíticas e geopolíticas mundiais.
\BNCC{EF09GE15}
 %Comparar e classificar diferentes regiões do mundo com base em informações populacionais, econômicas e socioambientais representadas em mapas temáticos e com diferentes projeções cartográficas.
\BNCC{EF09GE16}
 %Identificar e comparar diferentes domínios morfoclimáticos da Europa, da Ásia e da Oceania.
\BNCC{EF09GE17}
 %Explicar as características físico-naturais e a forma de ocupação e usos da terra em diferentes regiões da Europa, da Ásia e da Oceania.
\BNCC{EF09GE18}
 %Identificar e analisar as cadeias industriais e de inovação e as consequências dos usos de recursos naturais e das diferentes fontes de energia (tais como termoelétrica, hidrelétrica, eólica e nuclear) em diferentes países.
\BNCC{EF06HI01}
 %Identificar diferentes formas de compreensão da noção de tempo e de periodização dos processos históricos (continuidades e rupturas).
\BNCC{EF06HI02}
 %Identificar a gênese da produção do saber histórico e analisar o significado das fontes que originaram determinadas formas de registro em sociedades e épocas distintas.
\BNCC{EF06HI03}
 %Identificar as hipóteses científicas sobre o surgimento da espécie humana e sua historicidade e analisar os significados dos mitos de fundação.
\BNCC{EF06HI04}
 %Conhecer as teorias sobre a origem do homem americano.
\BNCC{EF06HI05}
 %Descrever modificações da natureza e da paisagem realizadas por diferentes tipos de sociedade, com destaque para os povos indígenas originários e povos africanos, e discutir a natureza e a lógica das transformações ocorridas.
\BNCC{EF06HI06}
 %Identificar geograficamente as rotas de povoamento no território americano.
\BNCC{EF06HI07}
 %Identificar aspectos e formas de registro das sociedades antigas na África, no Oriente Médio e nas Américas, distinguindo alguns significados presentes na cultura material e na tradição oral dessas sociedades.
\BNCC{EF06HI08}
 %Identificar os espaços territoriais ocupados e os aportes culturais, científicos, sociais e econômicos dos astecas, maias e incas e dos povos indígenas de diversas regiões brasileiras.
\BNCC{EF06HI09}
 %Discutir o conceito de Antiguidade Clássica, seu alcance e limite na tradição ocidental, assim como os impactos sobre outras sociedades e culturas.
\BNCC{EF06HI10}
 %Explicar a formação da Grécia Antiga, com ênfase na formação da pólis e nas transformações políticas, sociais e culturais.
\BNCC{EF06HI11}
 %Caracterizar o processo de formação da Roma Antiga e suas configurações sociais e políticas nos períodos monárquico e republicano.
\BNCC{EF06HI12}
 %Associar o conceito de cidadania a dinâmicas de inclusão e exclusão na Grécia e Roma antigas.
\BNCC{EF06HI13}
 %Conceituar “império” no mundo antigo, com vistas à análise das diferentes formas de equilíbrio e desequilíbrio entre as partes envolvidas.
\BNCC{EF06HI14}
 %Identificar e analisar diferentes formas de contato, adaptação ou exclusão entre populações em diferentes tempos e espaços.
\BNCC{EF06HI15}
 %Descrever as dinâmicas de circulação de pessoas, produtos e culturas no Mediterrâneo e seu significado.
\BNCC{EF06HI16}
 %Caracterizar e comparar as dinâmicas de abastecimento e as formas de organização do trabalho e da vida social em diferentes sociedades e períodos, com destaque para as relações entre senhores e servos.
\BNCC{EF06HI17}
 %Diferenciar escravidão, servidão e trabalho livre no mundo antigo.
\BNCC{EF06HI18}
 %Analisar o papel da religião cristã na cultura e nos modos de organização social no período medieval.
\BNCC{EF06HI19}
 %Descrever e analisar os diferentes papéis sociais das mulheres no mundo antigo e nas sociedades medievais.
\BNCC{EF07HI01}
 %Explicar o significado de “modernidade” e suas lógicas de inclusão e exclusão, com base em uma concepção europeia.
\BNCC{EF07HI02}
 %Identificar conexões e interações entre as sociedades do Novo Mundo, da Europa, da África e da Ásia no contexto das navegações e indicar a complexidade e as interações que ocorrem nos Oceanos Atlântico, Índico e Pacífico.
\BNCC{EF07HI03}
 %Identificar aspectos e processos específicos das sociedades africanas e americanas antes da chegada dos europeus, com destaque para as formas de organização social e o desenvolvimento de saberes e técnicas.
\BNCC{EF07HI04}
 %Identificar as principais características dos Humanismos e dos Renascimentos e analisar seus significados.
\BNCC{EF07HI05}
 %Identificar e relacionar as vinculações entre as reformas religiosas e os processos culturais e sociais do período moderno na Europa e na América.
\BNCC{EF07HI06}
 %Comparar as navegações no Atlântico e no Pacífico entre os séculos XIV e XVI.
\BNCC{EF07HI07}
 %Descrever os processos de formação e consolidação das monarquias e suas principais características com vistas à compreensão das razões da centralização política.
\BNCC{EF07HI08}
 %Descrever as formas de organização das sociedades americanas no tempo da conquista com vistas à compreensão dos mecanismos de alianças, confrontos e resistências.
\BNCC{EF07HI09}
 %Analisar os diferentes impactos da conquista europeia da América para as populações ameríndias e identificar as formas de resistência.
\BNCC{EF07HI10}
 %Analisar, com base em documentos históricos, diferentes interpretações sobre as dinâmicas das sociedades americanas no período colonial.
\BNCC{EF07HI11}
 %Analisar a formação histórico-geográfica do território da América portuguesa por meio de mapas históricos.
\BNCC{EF07HI12}
 %Identificar a distribuição territorial da população brasileira em diferentes épocas, considerando a diversidade étnico-racial e étnico-cultural (indígena, africana, europeia e asiática).
\BNCC{EF07HI13}
 %Caracterizar a ação dos europeus e suas lógicas mercantis visando ao domínio no mundo atlântico.
\BNCC{EF07HI14}
 %Descrever as dinâmicas comerciais das sociedades americanas e africanas e analisar suas interações com outras sociedades do Ocidente e do Oriente.
\BNCC{EF07HI15}
 %Discutir o conceito de escravidão moderna e suas distinções em relação ao escravismo antigo e à servidão medieval.
\BNCC{EF07HI16}
 %Analisar os mecanismos e as dinâmicas de comércio de escravizados em suas diferentes fases, identificando os agentes responsáveis pelo tráfico e as regiões e zonas africanas de procedência dos escravizados.
\BNCC{EF07HI17}
 %Discutir as razões da passagem do mercantilismo para o capitalismo.
\BNCC{EF08HI01}
 %Identificar os principais aspectos conceituais do iluminismo e do liberalismo e discutir a relação entre eles e a organização do mundo contemporâneo.
\BNCC{EF08HI02}
 %Identificar as particularidades político-sociais da Inglaterra do século XVII e analisar os desdobramentos posteriores à Revolução Gloriosa.
\BNCC{EF08HI03}
 %Analisar os impactos da Revolução Industrial na produção e circulação de povos, produtos e culturas.
\BNCC{EF08HI04}
 %Identificar e relacionar os processos da Revolução Francesa e seus desdobramentos na Europa e no mundo.
\BNCC{EF08HI05}
 %Explicar os movimentos e as rebeliões da América portuguesa, articulando as temáticas locais e suas interfaces com processos ocorridos na Europa e nas Américas.
\BNCC{EF08HI06}
 %Aplicar os conceitos de Estado, nação, território, governo e país para o entendimento de conflitos e tensões.
\BNCC{EF08HI07}
 %Identificar e contextualizar as especificidades dos diversos processos de independência nas Américas, seus aspectos populacionais e suas conformações territoriais.
\BNCC{EF08HI08}
 %Conhecer o ideário dos líderes dos movimentos independentistas e seu papel nas revoluções que levaram à independência das colônias hispano-americanas.
\BNCC{EF08HI09}
 %Conhecer as características e os principais pensadores do Pan-americanismo.
\BNCC{EF08HI10}
 %Identificar a Revolução de São Domingo como evento singular e desdobramento da Revolução Francesa e avaliar suas implicações.
\BNCC{EF08HI11}
 %Identificar e explicar os protagonismos e a atuação de diferentes grupos sociais e étnicos nas lutas de independência no Brasil, na América espanhola e no Haiti.
\BNCC{EF08HI12}
 %Caracterizar a organização política e social no Brasil desde a chegada da Corte portuguesa, em 1808, até 1822 e seus desdobramentos para a história política brasileira.
\BNCC{EF08HI13}
 %Analisar o processo de independência em diferentes países latino-americanos e comparar as formas de governo neles adotadas.
\BNCC{EF08HI14}
 %Discutir a noção da tutela dos grupos indígenas e a participação dos negros na sociedade brasileira do final do período colonial, identificando permanências na forma de preconceitos, estereótipos e violências sobre as populações indígenas e negras no Brasil e nas Américas.
\BNCC{EF08HI15}
 %Identificar e analisar o equilíbrio das forças e os sujeitos envolvidos nas disputas políticas durante o Primeiro e o Segundo Reinado.
\BNCC{EF08HI16}
 %Identificar, comparar e analisar a diversidade política, social e regional nas rebeliões e nos movimentos contestatórios ao poder centralizado.
\BNCC{EF08HI17}
 %Relacionar as transformações territoriais, em razão de questões de fronteiras, com as tensões e conflitos durante o Império.
\BNCC{EF08HI18}
 %Identificar as questões internas e externas sobre a atuação do Brasil na Guerra do Paraguai e discutir diferentes versões sobre o conflito.
\BNCC{EF08HI19}
 %Formular questionamentos sobre o legado da escravidão nas Américas, com base na seleção e consulta de fontes de diferentes naturezas.
\BNCC{EF08HI20}
 %Identificar e relacionar aspectos das estruturas sociais da atualidade com os legados da escravidão no Brasil e discutir a importância de ações afirmativas.
\BNCC{EF08HI21}
 %Identificar e analisar as políticas oficiais com relação ao indígena durante o Império.
\BNCC{EF08HI22}
 %Discutir o papel das culturas letradas, não letradas e das artes na produção das identidades no Brasil do século XIX.
\BNCC{EF08HI23}
 %Estabelecer relações causais entre as ideologias raciais e o determinismo no contexto do imperialismo europeu e seus impactos na África e na Ásia.
\BNCC{EF08HI24}
 %Reconhecer os principais produtos, utilizados pelos europeus, procedentes do continente africano durante o imperialismo e analisar os impactos sobre as comunidades locais na forma de organização e exploração econômica.
\BNCC{EF08HI25}
 %Caracterizar e contextualizar aspectos das relações entre os Estados Unidos da América e a América Latina no século XIX.
\BNCC{EF08HI26}
 %Identificar e contextualizar o protagonismo das populações locais na resistência ao imperialismo na África e Ásia.
\BNCC{EF08HI27}
 %Identificar as tensões e os significados dos discursos civilizatórios, avaliando seus impactos negativos para os povos indígenas originários e as populações negras nas Américas.
\BNCC{EF09HI01}
 %Descrever e contextualizar os principais aspectos sociais, culturais, econômicos e políticos da emergência da República no Brasil.
\BNCC{EF09HI02}
 %Caracterizar e compreender os ciclos da história republicana, identificando particularidades da história local e regional até 1954.
\BNCC{EF09HI03}
 %Identificar os mecanismos de inserção dos negros na sociedade brasileira pós-abolição e avaliar os seus resultados.
\BNCC{EF09HI04}
 %Discutir a importância da participação da população negra na formação econômica, política e social do Brasil.
\BNCC{EF09HI05}
 %Identificar os processos de urbanização e modernização da sociedade brasileira e avaliar suas contradições e impactos na região em que vive.
\BNCC{EF09HI06}
 %Identificar e discutir o papel do trabalhismo como força política, social e cultural no Brasil, em diferentes escalas (nacional, regional, cidade, comunidade).
\BNCC{EF09HI07}
 %Identificar e explicar, em meio a lógicas de inclusão e exclusão, as pautas dos povos indígenas, no contexto republicano (até 1964), e das populações afrodescendentes.
\BNCC{EF09HI08}
 %Identificar as transformações ocorridas no debate sobre as questões da diversidade no Brasil durante o século XX e compreender o significado das mudanças de abordagem em relação ao tema.
\BNCC{EF09HI09}
 %Relacionar as conquistas de direitos políticos, sociais e civis à atuação de movimentos sociais.
\BNCC{EF09HI10}
 %Identificar e relacionar as dinâmicas do capitalismo e suas crises, os grandes conflitos mundiais e os conflitos vivenciados na Europa.
\BNCC{EF09HI11}
 %Identificar as especificidades e os desdobramentos mundiais da Revolução Russa e seu significado histórico.
\BNCC{EF09HI12}
 %Analisar a crise capitalista de 1929 e seus desdobramentos em relação à economia global.
\BNCC{EF09HI13}
 %Descrever e contextualizar os processos da emergência do fascismo e do nazismo, a consolidação dos estados totalitários e as práticas de extermínio (como o holocausto).
\BNCC{EF09HI14}
 %Caracterizar e discutir as dinâmicas do colonialismo no continente africano e asiático e as lógicas de resistência das populações locais diante das questões internacionais.
\BNCC{EF09HI15}
 %Discutir as motivações que levaram à criação da Organização das Nações Unidas (ONU) no contexto do pós-guerra e os propósitos dessa organização.
\BNCC{EF09HI16}
 %Relacionar a Carta dos Direitos Humanos ao processo de afirmação dos direitos fundamentais e de defesa da dignidade humana, valorizando as instituições voltadas para a defesa desses direitos e para a identificação dos agentes responsáveis por sua violação.
\BNCC{EF09HI17}
 %Identificar e analisar processos sociais, econômicos, culturais e políticos do Brasil a partir de 1946.
\BNCC{EF09HI18}
 %Descrever e analisar as relações entre as transformações urbanas e seus impactos na cultura brasileira entre 1946 e 1964 e na produção das desigualdades regionais e sociais.
\BNCC{EF09HI19}
 %Identificar e compreender o processo que resultou na ditadura civil-militar no Brasil e discutir a emergência de questões relacionadas à memória e à justiça sobre os casos de violação dos direitos humanos.
\BNCC{EF09HI20}
 %Discutir os processos de resistência e as propostas de reorganização da sociedade brasileira durante a ditadura civil-militar.
\BNCC{EF09HI21}
 %Identificar e relacionar as demandas indígenas e quilombolas como forma de contestação ao modelo desenvolvimentista da ditadura.
\BNCC{EF09HI22}
 %Discutir o papel da mobilização da sociedade brasileira do final do período ditatorial até a Constituição de 1988.
\BNCC{EF09HI23}
 %Identificar direitos civis, políticos e sociais expressos na Constituição de 1988 e relacioná-los à noção de cidadania e ao pacto da sociedade brasileira de combate a diversas formas de preconceito, como o racismo.
\BNCC{EF09HI24}
 %Analisar as transformações políticas, econômicas, sociais e culturais de 1989 aos dias atuais, identificando questões prioritárias para a promoção da cidadania e dos valores democráticos.
\BNCC{EF09HI25}
 %Relacionar as transformações da sociedade brasileira aos protagonismos da sociedade civil após 1989.
\BNCC{EF09HI26}
 %Discutir e analisar as causas da violência contra populações marginalizadas (negros, indígenas, mulheres, homossexuais, camponeses, pobres etc.) com vistas à tomada de consciência e à construção de uma cultura de paz, empatia e respeito às pessoas.
\BNCC{EF09HI27}
 %Relacionar aspectos das mudanças econômicas, culturais e sociais ocorridas no Brasil a partir da década de 1990 ao papel do País no cenário internacional na era da globalização.
\BNCC{EF09HI28}
 %Identificar e analisar aspectos da Guerra Fria, seus principais conflitos e as tensões geopolíticas no interior dos blocos liderados por soviéticos e estadunidenses.
\BNCC{EF09HI29}
 %Descrever e analisar as experiências ditatoriais na América Latina, seus procedimentos e vínculos com o poder, em nível nacional e internacional, e a atuação de movimentos de contestação às ditaduras.
\BNCC{EF09HI30}
 %Comparar as características dos regimes ditatoriais latino-americanos, com especial atenção para a censura política, a opressão e o uso da força, bem como para as reformas econômicas e sociais e seus impactos.
\BNCC{EF09HI31}
 %Descrever e avaliar os processos de descolonização na África e na Ásia.
\BNCC{EF09HI32}
 %Analisar mudanças e permanências associadas ao processo de globalização, considerando os argumentos dos movimentos críticos às políticas globais.
\BNCC{EF09HI33}
 %Analisar as transformações nas relações políticas locais e globais geradas pelo desenvolvimento das tecnologias digitais de informação e comunicação.
\BNCC{EF09HI34}
 %Discutir as motivações da adoção de diferentes políticas econômicas na América Latina, assim como seus impactos sociais nos países da região.
\BNCC{EF09HI35}
 %Analisar os aspectos relacionados ao fenômeno do terrorismo na contemporaneidade, incluindo os movimentos migratórios e os choques entre diferentes grupos e culturas.
\BNCC{EF09HI36}
 %Identificar e discutir as diversidades identitárias e seus significados históricos no início do século XXI, combatendo qualquer forma de preconceito e violência.
\BNCC{EF06ER01}
 %Reconhecer o papel da tradição escrita na preservação de memórias, acontecimentos e ensinamentos religiosos.
\BNCC{EF06ER02}
 %Reconhecer e valorizar a diversidade de textos religiosos escritos (textos do Budismo, Cristianismo, Espiritismo, Hinduísmo, Islamismo, Judaísmo, entre outros).
\BNCC{EF06ER03}
 %Reconhecer, em textos escritos, ensinamentos relacionados a modos de ser e viver.
\BNCC{EF06ER04}
 %Reconhecer que os textos escritos são utilizados pelas tradições religiosas de maneiras diversas.
\BNCC{EF06ER05}
 %Discutir como o estudo e a interpretação dos textos religiosos influenciam os adeptos a vivenciarem os ensinamentos das tradições religiosas.
\BNCC{EF06ER06}
 %Reconhecer a importância dos mitos, ritos, símbolos e textos na estruturação das diferentes crenças, tradições e movimentos religiosos.
\BNCC{EF06ER07}
 %Exemplificar a relação entre mito, rito e símbolo nas práticas celebrativas de diferentes tradições religiosas.
\BNCC{EF07ER01}
 %Reconhecer e respeitar as práticas de comunicação com as divindades em distintas manifestações e tradições religiosas.
\BNCC{EF07ER02}
 %Identificar práticas de espiritualidade utilizadas pelas pessoas em determinadas situações (acidentes, doenças, fenômenos climáticos).
\BNCC{EF07ER03}
 %Reconhecer os papéis atribuídos às lideranças de diferentes tradições religiosas.
\BNCC{EF07ER04}
 %Exemplificar líderes religiosos que se destacaram por suas contribuições à sociedade.
\BNCC{EF07ER05}
 %Discutir estratégias que promovam a convivência ética e respeitosa entre as religiões.
\BNCC{EF07ER06}
 %Identificar princípios éticos em diferentes tradições religiosas e filosofias de vida, discutindo como podem influenciar condutas pessoais e práticas sociais.
\BNCC{EF07ER07}
 %Identificar e discutir o papel das lideranças religiosas e seculares na defesa e promoção dos direitos humanos.
\BNCC{EF07ER08}
 %Reconhecer o direito à liberdade de consciência, crença ou convicção, questionando concepções e práticas sociais que a violam.
\BNCC{EF08ER01}
 %Discutir como as crenças e convicções podem influenciar escolhas e atitudes pessoais e coletivas.
\BNCC{EF08ER02}
 %Analisar filosofias de vida, manifestações e tradições religiosas destacando seus princípios éticos.
\BNCC{EF08ER03}
 %Analisar doutrinas das diferentes tradições religiosas e suas concepções de mundo, vida e morte.
\BNCC{EF08ER04}
 %Discutir como filosofias de vida, tradições e instituições religiosas podem influenciar diferentes campos da esfera pública (política, saúde, educação, economia).
\BNCC{EF08ER05}
 %Debater sobre as possibilidades e os limites da interferência das tradições religiosas na esfera pública.
\BNCC{EF08ER06}
 %Analisar práticas, projetos e políticas públicas que contribuem para a promoção da liberdade de pensamento, crenças e convicções.
\BNCC{EF08ER07}
 %Analisar as formas de uso das mídias e tecnologias pelas diferentes denominações religiosas.
\BNCC{EF09ER01}
 %Analisar princípios e orientações para o cuidado da vida e nas diversas tradições religiosas e filosofias de vida.
\BNCC{EF09ER02}
 %Discutir as diferentes expressões de valorização e de desrespeito à vida, por meio da análise de matérias nas diferentes mídias.
\BNCC{EF09ER03}
 %Identificar sentidos do viver e do morrer em diferentes tradições religiosas, através do estudo de mitos fundantes.
\BNCC{EF09ER04}
 %Identificar concepções de vida e morte em diferentes tradições religiosas e filosofias de vida, por meio da análise de diferentes ritos fúnebres.
\BNCC{EF09ER05}
 %Analisar as diferentes ideias de imortalidade elaboradas pelas tradições religiosas (ancestralidade, reencarnação, transmigração e ressurreição).
\BNCC{EF09ER06}
 %Reconhecer a coexistência como uma atitude ética de respeito à vida e à dignidade humana.
\BNCC{EF09ER07}
 %Identificar princípios éticos (familiares, religiosos e culturais) que possam alicerçar a construção de projetos de vida.
\BNCC{EF09ER08}
 %Construir projetos de vida assentados em princípios e valores éticos.