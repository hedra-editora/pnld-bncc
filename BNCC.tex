
\BNCC{EF01LP01} Identificar: Espacialidade; Letramento;
%Reconhecer que textos são lidos e escritos da esquerda para a direita e de cima para baixo da página.

\BNCC{EF01LP02} Escrever: Exercícios: Ditado;
%Escrever, espontaneamente ou por ditado, palavras e frases de forma alfabética – usando letras/grafemas que representem fonemas.

\BNCC{EF01LP03} Identificar: Exercícios: Comparação, Leitura;
%Observar escritas convencionais, comparando-as às suas produções escritas, percebendo semelhanças e diferenças.

\BNCC{EF01LP04} Identificar: Letramento: sinais;
%Distinguir as letras do alfabeto de outros sinais gráficos.

\BNCC{EF01LP05} Identificar: Letramento: sons;
%Reconhecer o sistema de escrita alfabética como representação dos sons da fala.

\BNCC{EF01LP06} Falar: Letramento: fonemas na fala; Oral;
%Segmentar oralmente palavras em sílabas.

\BNCC{EF01LP07} Identificar: Identificar; Letramento: representação de fonemas;
%Identificar fonemas e sua representação por letras.

\BNCC{EF01LP08} Identificar +Relacionar: Letramento: sílabas; 
%Relacionar elementos sonoros (sílabas, fonemas, partes de palavras) com sua representação escrita.

\BNCC{EF01LP09} Identificar +Relacionar: Exercícios: Comparação;
%Comparar palavras, identificando semelhanças e diferenças entre sons de sílabas iniciais.

\BNCC{EF01LP10} Identificar: Letramento: nomes das letras;
%Nomear as letras do alfabeto e recitá-lo na ordem das letras.

\BNCC{EF01LP11} Identificar: Letramento: Letra cursiva, Maiúscula, Minúscula; 
%Conhecer, diferenciar e relacionar letras em formato imprensa e cursiva, maiúsculas e minúsculas.

\BNCC{EF01LP12} Identificar: Letramento: Espaços; 
%Reconhecer a separação das palavras, na escrita, por espaços em branco.

\BNCC{EF01LP13} Identificar: Redação: Comparação;
%Comparar palavras, identificando semelhanças e diferenças entre sons de sílabas mediais e finais.

\BNCC{EF01LP14} Identificar: Letramento: Pontuação, Voz alta;
%Identificar outros sinais no texto além das letras, como pontos finais, de interrogação e exclamação e seus efeitos na entonação.

\BNCC{EF01LP15} Identificar: Exercícios: Oposição e semelhança, agrupar palavras;
%Agrupar palavras pelo critério de aproximação de significado (sinonímia) e separar palavras pelo critério de oposição de significado (antonímia).

\BNCC{EF01LP16} Ler: Gêneros: texto popular; Exercícios: Trava-língua; Cultura popular: cantiga, quadra, quadrinhas;
%Ler e compreender, em colaboração com os colegas e com a ajuda do professor, quadras, quadrinhas, parlendas, trava-línguas, dentre outros gêneros do campo da vida cotidiana, considerando a situação comunicativa e o tema/assunto do texto e relacionando sua forma de organização à sua finalidade.

\BNCC{EF01LP17} Produzir: Produção: listas, agendas, calendário, avisos, contives, receitas, material digital; Vida cotidiana.
%Planejar e produzir, em colaboração com os colegas e com a ajuda do professor, listas, agendas, calendários, avisos, convites, receitas, instruções de montagem e legendas para álbuns, fotos ou ilustrações (digitais ou impressos), dentre outros gêneros do campo da vida cotidiana, considerando a situação comunicativa e o tema/assunto/ finalidade do texto.

\BNCC{EF01LP18} Produzir: Gêneros: texto popular; Exercícios: Trava-língua; Cultura popular: cantiga, quadra, quadrinhas;
%Registrar, em colaboração com os colegas e com a ajuda do professor, cantigas, quadras, quadrinhas, parlendas, trava-línguas, dentre outros gêneros do campo da vida cotidiana, considerando a situação comunicativa e o tema/assunto/finalidade do texto.

\BNCC{EF01LP19} Falar: Recitar; Cultura popular: cantiga, quadra, quadrinhas;
%Recitar parlendas, quadras, quadrinhas, trava-línguas, com entonação adequada e observando as rimas.

\BNCC{EF01LP20} Identificar: listas, agendas, calendários, regras, avisos, convites, receitas...; Vida Cotidiana;
%Identificar e reproduzir, em listas, agendas, calendários, regras, avisos, convites, receitas, instruções de montagem e legendas para álbuns, fotos ou ilustrações (digitais ou impressos), a formatação e diagramação específica de cada um desses gêneros.

\BNCC{EF01LP21} Produzir: rlistas, agendas, calendários, regras, avisos, convites, receitas...; Vida Cotidiana;
%Escrever, em colaboração com os colegas e com a ajuda do professor, listas de regras e regulamentos que organizam a vida na comunidade escolar, dentre outros gêneros do campo da atuação cidadã, considerando a situação comunicativa e o tema/assunto do texto.

\BNCC{EF01LP22} Produzir: diagramas, entrevistas, curiosidades; Digital; +professor;
%Planejar e produzir, em colaboração com os colegas e com a ajuda do professor, diagramas, entrevistas, curiosidades, dentre outros gêneros do campo investigativo, digitais ou impressos, considerando a situação comunicativa e o tema/assunto/finalidade do texto.

\BNCC{EF01LP23} Falar: +grupo, +professor; entrevistas, curiosidades; Vídeo;
%Planejar e produzir, em colaboração com os colegas e com a ajuda do professor, entrevistas, curiosidades, dentre outros gêneros do campo investigativo, que possam ser repassados oralmente por meio de ferramentas digitais, em áudio ou vídeo, considerando a situação comunicativa e o tema/assunto/finalidade do texto.

\BNCC{EF01LP24} Identificar: Interpretar enunciados de tarefas; diagramas, entrevistas, curiosidades, digitais ou impressos... 
%Identificar e reproduzir, em enunciados de tarefas escolares, diagramas, entrevistas, curiosidades, digitais ou impressos, a formatação e diagramação específica de cada um desses gêneros, inclusive em suas versões orais.

\BNCC{EF01LP25} Produzir: texto; +professor; Gêneros: Reportagem, personagens, enredo, tempo e espaço; Livros de imagem; Jornais;
%Produzir, tendo o professor como escriba, recontagens de histórias lidas pelo professor, histórias imaginadas ou baseadas em livros de imagens, observando a forma de composição de textos narrativos (personagens, enredo, tempo e espaço).

\BNCC{EF01LP26} Identificar: personagens, enredo, tempo e espaço.
%Identificar elementos de uma narrativa lida ou escutada, incluindo personagens, enredo, tempo e espaço.

\BNCC{EF12LP01} Ler: palavras novas; Frequência; Vocabulário;
%Ler palavras novas com precisão na decodificação, no caso de palavras de uso frequente, ler globalmente, por memorização.

\BNCC{EF12LP02} Ler: +professor; +grupo; Textos digitais;
%Buscar, selecionar e ler, com a mediação do professor (leitura compartilhada), textos que circulam em meios impressos ou digitais, de acordo com as necessidades e interesses.

\BNCC{EF12LP03} Produzir: Textos breves; Redação; Correção; Pontuação; Espaço;
%Copiar textos breves, mantendo suas características e voltando para o texto sempre que tiver dúvidas sobre sua distribuição gráfica, espaçamento entre as palavras, escrita das palavras e pontuação.

\BNCC{EF12LP04} Ler: +grupo, +professor, +sozinho; listas, agendas, calendários, avisos, convites, receitas; Vida cotidiana;
%Ler e compreender, em colaboração com os colegas e com a ajuda do professor ou já com certa autonomia, listas, agendas, calendários, avisos, convites, receitas, instruções de montagem (digitais ou impressos), dentre outros gêneros do campo da vida cotidiana, considerando a situação comunicativa e o tema/assunto do texto e relacionando sua forma de organização à sua finalidade.

\BNCC{EF12LP05} Produzir: +grupo, +professor; Gênero: letras de canção, cordel; Artístico-literário; Poemas;
%Planejar e produzir, em colaboração com os colegas e com a ajuda do professor, (re)contagens de histórias, poemas e outros textos versificados (letras de canção, quadrinhas, cordel), poemas visuais, tiras e histórias em quadrinhos, dentre outros gêneros do campo artístico-literário, considerando a situação comunicativa e a finalidade do texto.

\BNCC{EF12LP06} Produzir: recados, avisos, convites, receitas, instruções de montagem; +grupo, +professor;
%Planejar e produzir, em colaboração com os colegas e com a ajuda do professor, recados, avisos, convites, receitas, instruções de montagem, dentre outros gêneros do campo da vida cotidiana, que possam ser repassados oralmente por meio de ferramentas digitais, em áudio ou vídeo, considerando a situação comunicativa e o tema/assunto/finalidade do texto.

\BNCC{EF12LP07} Identificar: rimas, aliterações, assonâncias, o ritmo; cantiga, quadras, quadrinhas; Poesia;
%Identificar e (re)produzir, em cantiga, quadras, quadrinhas, parlendas, trava-línguas e canções, rimas, aliterações, assonâncias, o ritmo de fala relacionado ao ritmo e à melodia das músicas e seus efeitos de sentido.

\BNCC{EF12LP08} Ler: fotolegendas em notícias, manchetes e lides em notícias; +grupo, +professor; Jornais;
%Ler e compreender, em colaboração com os colegas e com a ajuda do professor, fotolegendas em notícias, manchetes e lides em notícias, álbum de fotos digital noticioso e notícias curtas para público infantil, dentre outros gêneros do campo jornalístico, considerando a situação comunicativa e o tema/assunto do texto.

\BNCC{EF12LP09} Ler: Publicidade; +grupo, +professor; Entender o assunto.
%Ler e compreender, em colaboração com os colegas e com a ajuda do professor, slogans, anúncios publicitários e textos de campanhas de conscientização destinados ao público infantil, dentre outros gêneros do campo publicitário, considerando a situação comunicativa e o tema/assunto do texto.

\BNCC{EF12LP10} Ler: Cartaz; +grupo, +professor; Cidadania;
%Ler e compreender, em colaboração com os colegas e com a ajuda do professor, cartazes, avisos, folhetos, regras e regulamentos que organizam a vida na comunidade escolar, dentre outros gêneros do campo da atuação cidadã, considerando a situação comunicativa e o tema/assunto do texto.

\BNCC{EF12LP11} Escrever: Jornais; +grupo, +professor; Cidadania;
%Escrever, em colaboração com os colegas e com a ajuda do professor, fotolegendas em notícias, manchetes e lides em notícias, álbum de fotos digital noticioso e notícias curtas para público infantil, digitais ou impressos, dentre outros gêneros do campo jornalístico, considerando a situação comunicativa e o tema/assunto do texto.

\BNCC{EF12LP12} Produzir: Escrvere publicidade; 
%Escrever, em colaboração com os colegas e com a ajuda do professor, slogans, anúncios publicitários e textos de campanhas de conscientização destinados ao público infantil, dentre outros gêneros do campo publicitário, considerando a situação comunicativa e o tema/ assunto/finalidade do texto.

\BNCC{EF12LP13} Produzir: Campanha publicitária; Cidadania;
%Planejar, em colaboração com os colegas e com a ajuda do professor, slogans e peça de campanha de conscientização destinada ao público infantil que possam ser repassados oralmente por meio de ferramentas digitais, em áudio ou vídeo, considerando a situação comunicativa e o tema/assunto/finalidade do texto.

\BNCC{EF12LP14} Identificar: Jornais; Opinião;
%Identificar e reproduzir, em fotolegendas de notícias, álbum de fotos digital noticioso, cartas de leitor (revista infantil), digitais ou impressos, a formatação e diagramação específica de cada um desses gêneros, inclusive em suas versões orais.

\BNCC{EF12LP15} Identificar: Publicidade;
%Identificar a forma de composição de slogans publicitários.

\BNCC{EF12LP16} Identificar: Campanhas de conscientização;
%Identificar e reproduzir, em anúncios publicitários e textos de campanhas de conscientização destinados ao público infantil (orais e escritos, digitais ou impressos), a formatação e diagramação específica de cada um desses gêneros, inclusive o uso de imagens.

\BNCC{EF12LP17} Ler: Enunciados; +grupo, +prof; experimentos, entrevistas, verbetes; Compreensão de texto;
%Ler e compreender, em colaboração com os colegas e com a ajuda do professor, enunciados de tarefas escolares, diagramas, curiosidades, pequenos relatos de experimentos, entrevistas, verbetes de enciclopédia infantil, entre outros gêneros do campo investigativo, considerando a situação comunicativa e o tema/assunto do texto.

\BNCC{EF12LP18} Ler: Poemas, rimas; Mundo imaginário;
%Apreciar poemas e outros textos versificados, observando rimas, sonoridades, jogos de palavras, reconhecendo seu pertencimento ao mundo imaginário e sua dimensão de encantamento, jogo e fruição.

\BNCC{EF12LP19} Identificar: Poemas, rimas; Associações;
%Reconhecer, em textos versificados, rimas, sonoridades, jogos de palavras, palavras, expressões, comparações, relacionando-as com sensações e associações.

\BNCC{EF15LP01} Identificar: Jornais; Contextos de circulação; Vida cotidiana;
%Identificar a função social de textos que circulam em campos da vida social dos quais participa cotidianamente (a casa, a rua, a comunidade, a escola) e nas mídias impressa, de massa e digital, reconhecendo para que foram produzidos, onde circulam, quem os produziu e a quem se destinam.

\BNCC{EF15LP02} Identificar: "Interpretação de texto", recepção, gênero, tema, imagens, e "partes da obra": índice, prefácio...;
%Estabelecer expectativas em relação ao texto que vai ler (pressuposições antecipadoras dos sentidos, da forma e da função social do texto), apoiando-se em seus conhecimentos prévios sobre as condições de produção e recepção desse texto, o gênero, o suporte e o universo temático, bem como sobre saliências textuais, recursos gráficos, imagens, dados da própria obra (índice, prefácio etc.), confirmando antecipações e inferências realizadas antes e durante a leitura de textos, checando a adequação das hipóteses realizadas.

\BNCC{EF15LP03} Identificar: Informações explícitas;
%Localizar informações explícitas em textos.

\BNCC{EF15LP04} Identificar: "interpretação de imagem", recursos gráfico-visuais, multissemióticos;
%Identificar o efeito de sentido produzido pelo uso de recursos expressivos gráfico-visuais em textos multissemióticos.

\BNCC{EF15LP05} Produzir: planejar "partes do texto", para quem escreve, tema, "adequação";
%Planejar, com a ajuda do professor, o texto que será produzido, considerando a situação comunicativa, os interlocutores (quem escreve/para quem escreve); a finalidade ou o propósito (escrever para quê); a circulação (onde o texto vai circular); o suporte (qual é o portador do texto); a linguagem, organização e forma do texto e seu tema, pesquisando em meios impressos ou digitais, sempre que for preciso, informações necessárias à produção do texto, organizando em tópicos os dados e as fontes pesquisadas.

\BNCC{EF15LP06} Produzir: revisar: cortes, reformulações, ortografia, "edição";
%Reler e revisar o texto produzido com a ajuda do professor e a colaboração dos colegas, para corrigi-lo e aprimorá-lo, fazendo cortes, acréscimos, reformulações, correções de ortografia e pontuação.

\BNCC{EF15LP07} Produzir: editar em grupo; +grupo, +professor;
%Editar a versão final do texto, em colaboração com os colegas e com a ajuda do professor, ilustrando, quando for o caso, em suporte adequado, manual ou digital.

\BNCC{EF15LP08} Produzir: programas de edição de texto;
%Utilizar software, inclusive programas de edição de texto, para editar e publicar os textos produzidos, explorando os recursos multissemióticos disponíveis.

\BNCC{EF15LP09} Falar: tom de voz audível, boa articulação; "seminário"; retórica: "actio, ação"; 
%Expressar-se em situações de intercâmbio oral com clareza, preocupando-se em ser compreendido pelo interlocutor e usando a palavra com tom de voz audível, boa articulação e ritmo adequado.

\BNCC{EF15LP10} Falar: formular perguntas; +grupo, +professor;
%Escutar, com atenção, falas de professores e colegas, formulando perguntas pertinentes ao tema e solicitando esclarecimentos sempre que necessário.

\BNCC{EF15LP11} Identificar +Falar: conversação; "roda de conversa", "discussão";
%Reconhecer características da conversação espontânea presencial, respeitando os turnos de fala, selecionando e utilizando, durante a conversação, formas de tratamento adequadas, de acordo com a situação e a posição do interlocutor.

\BNCC{EF15LP12} Identificar +Falar: "retórica: actio, ação";
%Atribuir significado a aspectos não linguísticos (paralinguísticos) observados na fala, como direção do olhar, riso, gestos, movimentos da cabeça (de concordância ou discordância), expressão corporal, tom de voz.

\BNCC{EF15LP13} Identificar: diferentes opiniões; Opinião; 
%Identificar finalidades da interação oral em diferentes contextos comunicativos (solicitar informações, apresentar opiniões, informar, relatar experiências etc.).

\BNCC{EF15LP14} Identificar: "fale com suas próprias palavras", quadrinhos, tirinhas;
%Construir o sentido de histórias em quadrinhos e tirinhas, relacionando imagens e palavras e interpretando recursos gráficos (tipos de balões, de letras, onomatopeias).

\BNCC{EF15LP15} Identificar: patrimônio, dimensão lúdica; Mundo imaginário;
%Reconhecer que os textos literários fazem parte do mundo do imaginário e apresentam uma dimensão lúdica, de encantamento, valorizando-os, em sua diversidade cultural, como patrimônio artístico da humanidade.

\BNCC{EF15LP16} Ler: narrativas, contos, crônicoas; +grupo, +professor e +sozinho; Mundo imaginário;	
%Ler e compreender, em colaboração com os colegas e com a ajuda do professor e, mais tarde, de maneira autônoma, textos narrativos de maior porte como contos (populares, de fadas, acumulativos, de assombração etc.) e crônicas.

\BNCC{EF15LP17} Ler: poemas, "poemas gráficos", "poesia concreta";
%Apreciar poemas visuais e concretos, observando efeitos de sentido criados pelo formato do texto na página, distribuição e diagramação das letras, pelas ilustrações e por outros efeitos visuais.

\BNCC{EF15LP18} Identificar: texto e ilustração;
%Relacionar texto com ilustrações e outros recursos gráficos.

\BNCC{EF15LP19} Falar: "com suas próprias palavras";
%Recontar oralmente, com e sem apoio de imagem, textos literários lidos pelo professor.

\BNCC{EF02LP01} Produzir: pontuação, ortografica correta; "correção de redação"; "Gramática";
%Utilizar, ao produzir o texto, grafia correta de palavras conhecidas ou com estruturas silábicas já dominadas, letras maiúsculas em início de frases e em substantivos próprios, segmentação entre as palavras, ponto final, ponto de interrogação e ponto de exclamação.

\BNCC{EF02LP02} Produzir: Segmentar palavras; "separação silábica"; "Gramática";
%Segmentar palavras em sílabas e remover e substituir sílabas iniciais, mediais ou finais para criar novas palavras.

\BNCC{EF02LP03} Produzir: por fonemas ("A bola bateu no Beto"); (f, v, t, d, p, b); "Gramática";
%Ler e escrever palavras com correspondências regulares diretas entre letras e fonemas (f, v, t, d, p, b) e correspondências regulares contextuais (c e q; e e o, em posição átona em final de palavra).

\BNCC{EF02LP04} Produzir: CV, V, CVC, CCV; consoante-vogal, vogal ... vogal-vogal e consoante-vogal-vogal; "Gramática";
%Ler e escrever corretamente palavras com sílabas CV, V, CVC, CCV, identificando que existem vogais em todas as sílabas.
% -- Este é um assunto que pertence à análise linguística e semiótica. As siglas CV, V, CVC, CCV, VC, VV e CVV significam, respectivamente: consoante-vogal, vogal, consoante-vogal-consoante, vogal-consoante, vogal-vogal e consoante-vogal-vogal.
% Levando em conta os conhecimentos explicados acima, podemos inferir um conjunto de palavras pertencentes à cada modalidade.
% CV = careca; peteca; colorido.
% V = ético; ideia; atleta
% CVC = partir, formatura; costas.
% CCV = profissional; cratera; trator.
% VV = averiguei; quais; paraguai.
% CVV = poeira; goela; poente.


\BNCC{EF02LP05} Ler +Produzir: uso do til, m e n finais; "Gramática";
%Ler e escrever corretamente palavras com marcas de nasalidade (til, m, n).

\BNCC{EF02LP06} Identificar: o princípio acrofônico; "começo de palavras"; "Gramática";
%Perceber o princípio acrofônico que opera nos nomes das letras do alfabeto.
% -- Acrofonia (em grego: acro – princípio, cabeça + phonos - som) consiste em dar às letras de um sistema de escrita (alfabeto) uma denominação de modo que o nome de cada letra começa com essa mesma letra. ... Por exemplo, A, "alpha", "amarelo" e "amor" são nomes acrofônicas da letra A.

\BNCC{EF02LP07} Produzir +Escrever: palavras, frases, textos curtos; "Redação";
%Escrever palavras, frases, textos curtos nas formas imprensa e cursiva.

\BNCC{EF02LP08} Produzir: Segmentar corretamente as palavras; "Redação";
%Segmentar corretamente as palavras ao escrever frases e textos.

\BNCC{EF02LP09} Produzir: ponto final, ponto de interrogação e ponto de exclamação; "Redação";
%Usar adequadamente ponto final, ponto de interrogação e ponto de exclamação.

\BNCC{EF02LP10} Identificar: sinônimos;
%Identificar sinônimos de palavras de texto lido, determinando a diferença de sentido entre eles, e formar antônimos de palavras encontradas em texto lido pelo acréscimo do prefixo de negação in-/im-.

\BNCC{EF02LP11} Produzir +Formar: o aumentativo e o diminutivo;
%Formar o aumentativo e o diminutivo de palavras com os sufixos -ão e -inho/-zinho.

\BNCC{EF02LP12} Ler: cantigas, letras de canção; "Poesia/Poema"; "Vida cotidiana";
%Ler e compreender com certa autonomia cantigas, letras de canção, dentre outros gêneros do campo da vida cotidiana, considerando a situação comunicativa e o tema/assunto do texto e relacionando sua forma de organização à sua finalidade.

\BNCC{EF02LP13} Produzir: bilhetes e cartas; "Vida cotidiana";
%Planejar e produzir bilhetes e cartas, em meio impresso e/ou digital, dentre outros gêneros do campo da vida cotidiana, considerando a situação comunicativa e o tema/assunto/finalidade do texto.

\BNCC{EF02LP14} Produzir: pequenos relatos, experiências pessoais; "Opinião";
%Planejar e produzir pequenos relatos de observação de processos, de fatos, de experiências pessoais, mantendo as características do gênero, considerando a situação comunicativa e o tema/assunto do texto.

\BNCC{EF02LP15} Falar +Cantar: cantigas e canções; "Música";
%Cantar cantigas e canções, obedecendo ao ritmo e à melodia.

\BNCC{EF02LP16} Identificar: em bilhetes, recados, avisos, cartas... a formatação do gênero; "Edição"; 
%Identificar e reproduzir, em bilhetes, recados, avisos, cartas, e-mails, receitas (modo de fazer), relatos (digitais ou impressos), a formatação e diagramação específica de cada um desses gêneros.

\BNCC{EF02LP17} Identificar: experiências pessoais, informalidade, "cagegorias temporais: antes, ontem, hoje, amanhã;
%Identificar e reproduzir, em relatos de experiências pessoais, a sequência dos fatos, utilizando expressões que marquem a passagem do tempo (“antes”, “depois”, “ontem”, “hoje”, “amanhã”, “outro dia”, “antigamente”, “há muito tempo” etc.), e o nível de informatividade necessário.

\BNCC{EF02LP18} Produzir: divulgar evento da escola; "Vida cotidiana";
%Planejar e produzir cartazes e folhetos para divulgar eventos da escola ou da comunidade, utilizando linguagem persuasiva e elementos textuais e visuais (tamanho da letra, leiaute, imagens) adequados ao gênero, considerando a situação comunicativa e o tema/assunto do texto.

\BNCC{EF02LP19} Produzir: notícia; +grupo, +professor; "Jornais";
%Planejar e produzir, em colaboração com os colegas e com a ajuda do professor, notícias curtas para público infantil, para compor jornal falado que possa ser repassado oralmente ou em meio digital, em áudio ou vídeo, dentre outros gêneros do campo jornalístico, considerando a situação comunicativa e o tema/assunto do texto.

\BNCC{EF02LP20} Identificar: identificar textos de atividades de pesquisa; "Interpretação de dados";
%Reconhecer a função de textos utilizados para apresentar informações coletadas em atividades de pesquisa (enquetes, pequenas entrevistas, registros de experimentações).

\BNCC{EF02LP21} Idenfificar +Explorar: textos digitais; +professor; "internet"; 
%Explorar, com a mediação do professor, textos informativos de diferentes ambientes digitais de pesquisa, conhecendo suas possibilidades.

\BNCC{EF02LP22} Produzir: experimentos, entrevistas, verbetes; +grupo, +professor
%Planejar e produzir, em colaboração com os colegas e com a ajuda do professor, pequenos relatos de experimentos, entrevistas, verbetes de enciclopédia infantil, dentre outros gêneros do campo investigativo, digitais ou impressos, considerando a situação comunicativa e o tema/assunto/finalidade do texto.

\BNCC{EF02LP23} Produzir: registros de observação; +sozinho; 
%Planejar e produzir, com certa autonomia, pequenos registros de observação de resultados de pesquisa, coerentes com um tema investigado.

\BNCC{EF02LP24} Falar: experimentos, entrevistas, verbetes; +grupo, +professor; "Internet"; Oral;
%Planejar e produzir, em colaboração com os colegas e com a ajuda do professor, relatos de experimentos, registros de observação, entrevistas, dentre outros gêneros do campo investigativo, que possam ser repassados oralmente por meio de ferramentas digitais, em áudio ou vídeo, considerando a situação comunicativa e o tema/assunto/ finalidade do texto.

\BNCC{EF02LP25} Identificar: formatação em experimentos, entrevistas, verbetes; 
%Identificar e reproduzir, em relatos de experimentos, entrevistas, verbetes de enciclopédia infantil, digitais ou impressos, a formatação e diagramação específica de cada um desses gêneros, inclusive em suas versões orais.

\BNCC{EF02LP26} Ler +Apreciar: textos literários, de gêneros variados; +sozinho; Gênero; 
%Ler e compreender, com certa autonomia, textos literários, de gêneros variados, desenvolvendo o gosto pela leitura.

\BNCC{EF02LP27} Produzir +Reescrever: textos narrativos "com suas próprias palavras";
%Reescrever textos narrativos literários lidos pelo professor.

\BNCC{EF02LP28} Identificar: identificar "plot", "trama", "conflito"; 
%Reconhecer o conflito gerador de uma narrativa ficcional e sua resolução, além de palavras, expressões e frases que caracterizam personagens e ambientes.

\BNCC{EF02LP29} Identificar: formatação em poemas visuais;
%Observar, em poemas visuais, o formato do texto na página, as ilustrações e outros efeitos visuais.

\BNCC{EF03LP01} Ler: c/qu; g/gu; r/rr; s/ss; o (e não u) e (e não i) e (til, m, n); "Gramática";
%Ler e escrever palavras com correspondências regulares contextuais entre grafemas e fonemas – c/qu; g/gu; r/rr; s/ss; o (e não u) e e (e não i) em sílaba átona em final de palavra – e com marcas de nasalidade (til, m, n).

\BNCC{EF03LP02} = EF02LP04 Produzir: CV, V, CVC, CCV +VC, +VV, +CVV; consoante-vogal, vogal ...; "Gramática";
%Ler e escrever corretamente palavras com sílabas CV, V, CVC, CCV, VC, VV, CVV, identificando que existem vogais em todas as sílabas.

\BNCC{EF03LP03} Produzir: dígrafos lh, nh, ch; "Gramática";
%Ler e escrever corretamente palavras com os dígrafos lh, nh, ch.

\BNCC{EF03LP04} Produzir: acentos em monossílabos terminados em a, e, o;  "Gramática";
%Usar acento gráfico (agudo ou circunflexo) em monossílabos tônicos terminados em a, e, o e em palavras oxítonas terminadas em a, e, o, seguidas ou não de s.

\BNCC{EF03LP05} Produzir: número de sílabas; "Gramática";
%Identificar o número de sílabas de palavras, classificando-as em monossílabas, dissílabas, trissílabas e polissílabas.

\BNCC{EF03LP06} Produzir: sílaba tônica; "Gramática";
%Identificar a sílaba tônica em palavras, classificando-as em oxítonas, paroxítonas e proparoxítonas.

\BNCC{EF03LP07} Identificar: pontuaçao; "Gramática";
%Identificar a função na leitura e usar na escrita ponto final, ponto de interrogação, ponto de exclamação e, em diálogos (discurso direto), dois-pontos e travessão.

\BNCC{EF03LP08} Identificar: substantivos e verbos; "Gramática";
%Identificar e diferenciar, em textos, substantivos e verbos e suas funções na oração: agente, ação, objeto da ação.

\BNCC{EF03LP09} Identificar: adjetivos, "Gramática";
%Identificar, em textos, adjetivos e sua função de atribuição de propriedades aos substantivos.

\BNCC{EF03LP10} Identificar: prefixos e sufixos; "Gramática";
%Reconhecer prefixos e sufixos produtivos na formação de palavras derivadas de substantivos, de adjetivos e de verbos, utilizando-os para compreender palavras e para formar novas palavras.

\BNCC{EF03LP11} Ler: texto e imagem; +sozinho; "Vida cotidiana"; 
%Ler e compreender, com autonomia, textos injuntivos instrucionais (receitas, instruções de montagem etc.), com a estrutura própria desses textos (verbos imperativos, indicação de passos a ser seguidos) e mesclando palavras, imagens e recursos gráfico- visuais, considerando a situação comunicativa e o tema/assunto do texto.

\BNCC{EF03LP12} Ler:  cartas pessoais e diários, opiniões, sentimentos; "Vida cotidiana";
%Ler e compreender, com autonomia, cartas pessoais e diários, com expressão de sentimentos e opiniões, dentre outros gêneros do campo da vida cotidiana, de acordo com as convenções do gênero carta e considerando a situação comunicativa e o tema/assunto do texto.

\BNCC{EF03LP13} Produzir: cartas pessoais e diários, opiniões, sentimentos; "Vida cotidiana";
%Planejar e produzir cartas pessoais e diários, com expressão de sentimentos e opiniões, dentre outros gêneros do campo da vida cotidiana, de acordo com as convenções dos gêneros carta e diário e considerando a situação comunicativa e o tema/assunto do texto.

\BNCC{EF03LP14} Produzir: textos instrucionais com imperativo; texto e imagem; 
%Planejar e produzir textos injuntivos instrucionais, com a estrutura própria desses textos (verbos imperativos, indicação de passos a ser seguidos) e mesclando palavras, imagens e recursos gráfico-visuais, considerando a situação comunicativa e o tema/ assunto do texto.

\BNCC{EF03LP15} Identificar: assistir programa de culinária infantil e produzir receita;
%Assistir, em vídeo digital, a programa de culinária infantil e, a partir dele, planejar e produzir receitas em áudio ou vídeo.

\BNCC{EF03LP16} Identificar: textos instrucionais com imperativo; texto e imagem; 
%Identificar e reproduzir, em textos injuntivos instrucionais (receitas, instruções de montagem, digitais ou impressos), a formatação própria desses textos (verbos imperativos, indicação de passos a ser seguidos) e a diagramação específica dos textos desses gêneros (lista de ingredientes ou materiais e instruções de execução – "modo de fazer").

\BNCC{EF03LP17} Identificar: formatação de carta e diário; data, saudação, corpo do texto;
%Identificar e reproduzir, em gêneros epistolares e diários, a formatação própria desses textos (relatos de acontecimentos, expressão de vivências, emoções, opiniões ou críticas) e a diagramação específica dos textos desses gêneros (data, saudação, corpo do texto, despedida, assinatura).

\BNCC{EF03LP18} Ler: "carta do leitor"; +sozinho; Jornais; Opinião;
%Ler e compreender, com autonomia, cartas dirigidas a veículos da mídia impressa ou digital (cartas de leitor e de reclamação a jornais, revistas) e notícias, dentre outros gêneros do campo jornalístico, de acordo com as convenções do gênero carta e considerando a situação comunicativa e o tema/assunto do texto.

\BNCC{EF03LP19} Identificar: retórica: "elocução"; Publicidade;
%Identificar e discutir o propósito do uso de recursos de persuasão (cores, imagens, escolha de palavras, jogo de palavras, tamanho de letras) em textos publicitários e de propaganda, como elementos de convencimento.

\BNCC{EF03LP20} Produzir: "carta do leitor"; +sozinho; Jornais; Opinião;
%Produzir cartas dirigidas a veículos da mídia impressa ou digital (cartas do leitor ou de reclamação a jornais ou revistas), dentre outros gêneros do campo político-cidadão, com opiniões e críticas, de acordo com as convenções do gênero carta e considerando a situação comunicativa e o tema/assunto do texto.

\BNCC{EF03LP21} Produzir: campanhas de conscientização; retórica: "elocução"; Publicidade;
%Produzir anúncios publicitários, textos de campanhas de conscientização destinados ao público infantil, observando os recursos de persuasão utilizados nos textos publicitários e de propaganda (cores, imagens, slogan, escolha de palavras, jogo de palavras, tamanho e tipo de letras, diagramação).

\BNCC{EF03LP22} Falar: telejornal para público infantil; +grupo; Jornais;
%Planejar e produzir, em colaboração com os colegas, telejornal para público infantil com algumas notícias e textos de campanhas que possam ser repassados oralmente ou em meio digital, em áudio ou vídeo, considerando a situação comunicativa, a organização específica da fala nesses gêneros e o tema/assunto/ finalidade dos textos.

\BNCC{EF03LP23} Identificar: adjetivos em carta do leitor; Jornais;
%Analisar o uso de adjetivos em cartas dirigidas a veículos da mídia impressa ou digital (cartas do leitor ou de reclamação a jornais ou revistas), digitais ou impressas.

\BNCC{EF03LP24} Ler: relatos de observações e de pesquisas em fontes de informações; +sozinho; "Internet";
%Ler/ouvir e compreender, com autonomia, relatos de observações e de pesquisas em fontes de informações, considerando a situação comunicativa e o tema/assunto do texto.

\BNCC{EF03LP25} Produzir: resultados de observações e de pesquisas; tabelas; gráficos; texto e imagem;
%Planejar e produzir textos para apresentar resultados de observações e de pesquisas em fontes de informações, incluindo, quando pertinente, imagens, diagramas e gráficos ou tabelas simples, considerando a situação comunicativa e o tema/assunto do texto.

\BNCC{EF03LP26} Identificar: resultados de observações e de pesquisas; tabelas; gráficos; texto e imagem;
%Identificar e reproduzir, em relatórios de observação e pesquisa, a formatação e diagramação específica desses gêneros (passos ou listas de itens, tabelas, ilustrações, gráficos, resumo dos resultados), inclusive em suas versões orais.

\BNCC{EF03LP27} Falar: recitar cordel e cantar repentes;  
%Recitar cordel e cantar repentes e emboladas, observando as rimas e obedecendo ao ritmo e à melodia.

\BNCC{EF35LP01} Ler: Leitura silenciosa e em voz alta; +sozinho; 
%Ler e compreender, silenciosamente e, em seguida, em voz alta, com autonomia e fluência, textos curtos com nível de textualidade adequado.

\BNCC{EF35LP02} Identificar: visita à biblioteca; Opinião;
%Selecionar livros da biblioteca e/ou do cantinho de leitura da sala de aula e/ou disponíveis em meios digitais para leitura individual, justificando a escolha e compartilhando com os colegas sua opinião, após a leitura.

\BNCC{EF35LP03} Identificar: a ideia central do texto; Compreensão de texto;
%Identificar a ideia central do texto, demonstrando compreensão global.

\BNCC{EF35LP04} Identificar: informações implícitas; Compreensão de texto;
%Inferir informações implícitas nos textos lidos.

\BNCC{EF35LP05} Identificar: contexto e sentido da frase; Compreensão de texto;
%Inferir o sentido de palavras ou expressões desconhecidas em textos, com base no contexto da frase ou do texto.

\BNCC{EF35LP06} Identificar: relações entre partes de um texto; Compreensão de texto;
%Recuperar relações entre partes de um texto, identificando substituições lexicais (de substantivos por sinônimos) ou pronominais (uso de pronomes anafóricos – pessoais, possessivos, demonstrativos) que contribuem para a continuidade do texto.

\BNCC{EF35LP07} Produzir: ortografia, regras básicas de concordância nominal e verbal;
%Utilizar, ao produzir um texto, conhecimentos linguísticos e gramaticais, tais como ortografia, regras básicas de concordância nominal e verbal, pontuação (ponto final, ponto de exclamação, ponto de interrogação, vírgulas em enumerações) e pontuação do discurso direto, quando for o caso.

\BNCC{EF35LP08} Produzir: recursos de referenciação; Gramática;
%Utilizar, ao produzir um texto, recursos de referenciação (por substituição lexical ou por pronomes pessoais, possessivos e demonstrativos), vocabulário apropriado ao gênero, recursos de coesão pronominal (pronomes anafóricos) e articuladores de relações de sentido (tempo, causa, oposição, conclusão, comparação), com nível suficiente de informatividade.

\BNCC{EF35LP09} Produzir: "edição"; Gênero; divisão e partes do texto;
%Organizar o texto em unidades de sentido, dividindo-o em parágrafos segundo as normas gráficas e de acordo com as características do gênero textual.

\BNCC{EF35LP10} Identificar: gêneros do discurso oral e gêneros;
%Identificar gêneros do discurso oral, utilizados em diferentes situações e contextos comunicativos, e suas características linguístico-expressivas e composicionais 
%(conversação espontânea, conversação telefônica, entrevistas pessoais, entrevistas no rádio ou na TV, debate, noticiário de rádio e TV, narração de jogos esportivos no rádio e TV, aula, debate etc.).

\BNCC{EF35LP11} Identificar: ouvir gravações, canções, textos falados;
%Ouvir gravações, canções, textos falados em diferentes variedades linguísticas, identificando características regionais, urbanas e rurais da fala e respeitando as diversas variedades linguísticas como características do uso da língua por diferentes grupos regionais ou diferentes culturas locais, rejeitando preconceitos linguísticos.

\BNCC{EF35LP12} Identificar: verbete; dicionário; ortografia; 
%Recorrer ao dicionário para esclarecer dúvida sobre a escrita de palavras, especialmente no caso de palavras com relações irregulares fonema-grafema.

\BNCC{EF35LP13} Identificar: memorizar palavras de uso frequente; 
%Memorizar a grafia de palavras de uso frequente nas quais as relações fonema-grafema são irregulares e com h inicial que não representa fonema.

\BNCC{EF35LP14} Identificar: pronomes, possessivos e demonstrativos; Gramática;
%Identificar em textos e usar na produção textual pronomes pessoais, possessivos e demonstrativos, como recurso coesivo anafórico.

\BNCC{EF35LP15} Falar: Debate; opinar e defender ponto de vista; Vida cotidiana; 
%Opinar e defender ponto de vista sobre tema polêmico relacionado a situações vivenciadas na escola e/ou na comunidade, utilizando registro formal e estrutura adequada à argumentação, considerando a situação comunicativa e o tema/assunto do texto.

\BNCC{EF35LP16} Identificar: formatação; Jornais; cartas de reclamação; 
%Identificar e reproduzir, em notícias, manchetes, lides e corpo de notícias simples para público infantil e cartas de reclamação (revista infantil), digitais ou impressos, a formatação e diagramação específica de cada um desses gêneros, inclusive em suas versões orais.

\BNCC{EF35LP17} Idenfificar: fenômenos sociais e naturais; +professor; 
%Buscar e selecionar, com o apoio do professor, informações de interesse sobre fenômenos sociais e naturais, em textos que circulam em meios impressos ou digitais.

\BNCC{EF35LP18} Identificar: ouvir apresentações realizadas por colegas; "seminário";
%Escutar, com atenção, apresentações de trabalhos realizadas por colegas, formulando perguntas pertinentes ao tema e solicitando esclarecimentos sempre que necessário.

\BNCC{EF35LP19} Identificar: as ideias principais formais; Compreensão de texto;
%Recuperar as ideias principais em situações formais de escuta de exposições, apresentações e palestras.

\BNCC{EF35LP20} Produzir: trabalhos ou pesquisas escolares; "seminário com imagem";
%Expor trabalhos ou pesquisas escolares, em sala de aula, com apoio de recursos multissemióticos (imagens, diagrama, tabelas etc.), orientando-se por roteiro escrito, planejando o tempo de fala e adequando a linguagem à situação comunicativa.

\BNCC{EF35LP21} Ler: +sozinho; Gênero; preferências por gêneros; Opinião;
%Ler e compreender, de forma autônoma, textos literários de diferentes gêneros e extensões, inclusive aqueles sem ilustrações, estabelecendo preferências por gêneros, temas, autores.

\BNCC{EF35LP22} Identificar: diálogos em textos narrativos; "nível de discurso";
%Perceber diálogos em textos narrativos, observando o efeito de sentido de verbos de enunciação e, se for o caso, o uso de variedades linguísticas no discurso direto.

\BNCC{EF35LP23} Ler +Apreciar: poema, rimas, aliterações; estroves, refrões;
%Apreciar poemas e outros textos versificados, observando rimas, aliterações e diferentes modos de divisão dos versos, estrofes e refrões e seu efeito de sentido.

\BNCC{EF35LP24} Identificar: texto dramático, personagens e cena; Teatro;
%Identificar funções do texto dramático (escrito para ser encenado) e sua organização por meio de diálogos entre personagens e marcadores das falas das personagens e de cena.

\BNCC{EF35LP25} Produzir: +sozinho; "redação"; narrativas ficcionais;  
%Criar narrativas ficcionais, com certa autonomia, utilizando detalhes descritivos, sequências de eventos e imagens apropriadas para sustentar o sentido do texto, e marcadores de tempo, espaço e de fala de personagens.

\BNCC{EF35LP26} Ler: +sozinho; narrativa com personagem; Compreensão de texto
%Ler e compreender, com certa autonomia, narrativas ficcionais que apresentem cenários e personagens, observando os elementos da estrutura narrativa: enredo, tempo, espaço, personagens, narrador e a construção do discurso indireto e discurso direto.

\BNCC{EF35LP27} Ler: versos, poemas, rimas; Compreensão de texto
%Ler e compreender, com certa autonomia, textos em versos, explorando rimas, sons e jogos de palavras, imagens poéticas (sentidos figurados) e recursos visuais e sonoros.

\BNCC{EF35LP28} Falar: Declamar poemas;
%Declamar poemas, com entonação, postura e interpretação adequadas.

\BNCC{EF35LP29} Identificar: personagem, cenário, narrativa; "trama/plot";
%Identificar, em narrativas, cenário, personagem central, conflito gerador, resolução e o ponto de vista com base no qual histórias são narradas, diferenciando narrativas em primeira e terceira pessoas.

\BNCC{EF35LP30} Identificar: discurso indireto e discurso direto; Gramática;
%Diferenciar discurso indireto e discurso direto, determinando o efeito de sentido de verbos de enunciação e explicando o uso de variedades linguísticas no discurso direto, quando for o caso.

\BNCC{EF35LP31} Identificar: versos, poemas, recursos rítmicos; metáfora; "música"
%Identificar, em textos versificados, efeitos de sentido decorrentes do uso de recursos rítmicos e sonoros e de metáforas.

\BNCC{EF04LP01} Produzir: correspondência fonema--grafema: Gramática;
%Grafar palavras utilizando regras de correspondência fonema--grafema regulares diretas e contextuais.

\BNCC{EF04LP02} Ler:  palavras com sílabas VV e CVV...; Gramática;
%Ler e escrever, corretamente, palavras com sílabas VV e CVV em casos nos quais a combinação VV (ditongo) é reduzida na língua oral (ai, ei, ou).

\BNCC{EF04LP03} Identificar: Localizar palavras, dicionário, verbetes;
%Localizar palavras no dicionário para esclarecer significados, reconhecendo o significado mais plausível para o contexto que deu origem à consulta.

\BNCC{EF04LP04} Produzir: Usar acento gráfico (agudo ou circunflexo); "rouxinol";
%Usar acento gráfico (agudo ou circunflexo) em paroxítonas terminadas em -i(s), -l, -r, -ão(s).

\BNCC{EF04LP05} Identificar: ponto final, "!,?,:,—"; Gramática
%Identificar a função na leitura e usar, adequadamente, na escrita ponto final, de interrogação, de exclamação, dois-pontos e travessão em diálogos (discurso direto), vírgula em enumerações e em separação de vocativo e de aposto.

\BNCC{EF04LP06} Identificar: concordância; Gramática;
%Identificar em textos e usar na produção textual a concordância entre substantivo ou pronome pessoal e verbo (concordância verbal).

\BNCC{EF04LP07} Identificar: concordância artigo, substantivo e adjetivo; Gramática;
%Identificar em textos e usar na produção textual a concordância entre artigo, substantivo e adjetivo (concordância no grupo nominal).

\BNCC{EF04LP08} Identificar: sufixos -agem, -oso, -eza, -izar/-isar: Gramática;
%Reconhecer e grafar, corretamente, palavras derivadas com os sufixos -agem, -oso, -eza, -izar/-isar (regulares morfológicas).

\BNCC{EF04LP09} Ler: +sozinho; boletos, faturas e carnês; Vida cotidiana;
%Ler e compreender, com autonomia, boletos, faturas e carnês, dentre outros gêneros do campo da vida cotidiana, de acordo com as convenções do gênero (campos, itens elencados, medidas de consumo, código de barras) e considerando a situação comunicativa e a finalidade do texto.

\BNCC{EF04LP10} Ler: +sozinho; cartas de reclamação; Vida cotidiana; 
%Ler e compreender, com autonomia, cartas pessoais de reclamação, dentre outros gêneros do campo da vida cotidiana, de acordo com as convenções do gênero carta e considerando a situação comunicativa e o tema/assunto/finalidade do texto.

\BNCC{EF04LP11} Produzir: cartas de reclamação;
%Planejar e produzir, com autonomia, cartas pessoais de reclamação, dentre outros gêneros do campo da vida cotidiana, de acordo com as convenções do gênero carta e com a estrutura própria desses textos (problema, opinião, argumentos), considerando a situação comunicativa e o tema/assunto/finalidade do texto.

\BNCC{EF04LP12} Identificar +Assistir: "vídeo tutorial"; 
%Assistir, em vídeo digital, a programa infantil com instruções de montagem, de jogos e brincadeiras e, a partir dele, planejar e produzir tutoriais em áudio ou vídeo.

\BNCC{EF04LP13} Identificar: formatação em textos instrucionais; 
%Identificar e reproduzir, em textos injuntivos instrucionais (instruções de jogos digitais ou impressos), a formatação própria desses textos (verbos imperativos, indicação de passos a ser seguidos) e formato específico dos textos orais ou escritos desses gêneros (lista/ apresentação de materiais e instruções/passos de jogo).

\BNCC{EF04LP14} Identificar: fatos, participantes; Jornais;
%Identificar, em notícias, fatos, participantes, local e momento/tempo da ocorrência do fato noticiado.

\BNCC{EF04LP15} Identificar: fatos de opiniões/sugestões em textos; Jornais;
%Distinguir fatos de opiniões/sugestões em textos (informativos, jornalísticos, publicitários etc.).

\BNCC{EF04LP16} Produzir: notícias sobre fatos ocorridos no universo escolar; Jornais;
%Produzir notícias sobre fatos ocorridos no universo escolar, digitais ou impressas, para o jornal da escola, noticiando os fatos e seus atores e comentando decorrências, de acordo com as convenções do gênero notícia e considerando a situação comunicativa e o tema/assunto do texto.

\BNCC{EF04LP17} Produzir +Falar: jornais radiofônicos; Performance;  
%Produzir jornais radiofônicos ou televisivos e entrevistas veiculadas em rádio, TV e na internet, orientando-se por roteiro ou texto e demonstrando conhecimento dos gêneros jornal falado/televisivo e entrevista.

\BNCC{EF04LP18} Identificar: tons e expressões facial;
%Analisar o padrão entonacional e a expressão facial e corporal de âncoras de jornais radiofônicos ou televisivos e de entrevistadores/entrevistados.

\BNCC{EF04LP19} Ler: textos expositivos de divulgação científica para crianças; 
%Ler e compreender textos expositivos de divulgação científica para crianças, considerando a situação comunicativa e o tema/ assunto do texto.

\BNCC{EF04LP20} Identificar: função de gráficos, diagramas e tabelas em textos;
%Reconhecer a função de gráficos, diagramas e tabelas em textos, como forma de apresentação de dados e informações.

\BNCC{EF04LP21} Produzir: textos sobre temas baseados em fontes de info; Jornais;
%Planejar e produzir textos sobre temas de interesse, com base em resultados de observações e pesquisas em fontes de informações impressas ou eletrônicas, incluindo, quando pertinente, imagens e gráficos ou tabelas simples, considerando a situação comunicativa e o tema/assunto do texto.

\BNCC{EF04LP22} Produzir: +sozinho; verbetes/dicionário;
%Planejar e produzir, com certa autonomia, verbetes de enciclopédia infantil, digitais ou impressos, considerando a situação comunicativa e o tema/ assunto/finalidade do texto.

\BNCC{EF04LP23} Identificar: formatação em verbetes; 
%Identificar e reproduzir, em verbetes de enciclopédia infantil, digitais ou impressos, a formatação e diagramação específica desse gênero (título do verbete, definição, detalhamento, curiosidades), considerando a situação comunicativa e o tema/assunto/finalidade do texto.

\BNCC{EF04LP24} Identificar: formatação em relatórios;
%Identificar e reproduzir, em seu formato, tabelas, diagramas e gráficos em relatórios de observação e pesquisa, como forma de apresentação de dados e informações.

\BNCC{EF04LP25} Falar: montar cenas de texto dramático; Teatro; 
%Representar cenas de textos dramáticos, reproduzindo as falas das personagens, de acordo com as rubricas de interpretação e movimento indicadas pelo autor.

\BNCC{EF04LP26} Identificar: formatação em poemas; 
%Observar, em poemas concretos, o formato, a distribuição e a diagramação das letras do texto na página.

\BNCC{EF04LP27} Identificar: marcadores de fala em peças; Teatro;
%Identificar, em textos dramáticos, marcadores das falas das personagens e de cena.

\BNCC{EF05LP01} Identificar: grafar fonema-grafema regulares e irregulares; Gramática;
%Grafar palavras utilizando regras de correspondência fonema-grafema regulares, contextuais e morfológicas e palavras de uso frequente com correspondências irregulares.

\BNCC{EF05LP02} Identificar: polissemia das palavras; Compreensão de texto;
%Identificar o caráter polissêmico das palavras (uma mesma palavra com diferentes significados, de acordo com o contexto de uso), comparando o significado de determinados termos utilizados nas áreas científicas com esses mesmos termos utilizados na linguagem usual.

\BNCC{EF05LP03} Escrever: acentuar oxítonas, paroxítonas e proparoxítonas; Gramática;
%Acentuar corretamente palavras oxítonas, paroxítonas e proparoxítonas.

\BNCC{EF05LP04} Identificar: vírgula, ponto e vírgula, dois-pontos, ..., aspas, ();
%Diferenciar, na leitura de textos, vírgula, ponto e vírgula, dois-pontos e reconhecer, na leitura de textos, o efeito de sentido que decorre do uso de reticências, aspas, parênteses.

\BNCC{EF05LP05} Identificar: presente, passado e futuro em tempos verbais; Gramática;
%Identificar a expressão de presente, passado e futuro em tempos verbais do modo indicativo.

\BNCC{EF05LP06} Produzir +Escrever: flexionar verbos por escrito e oralmente; Gramática;
%Flexionar, adequadamente, na escrita e na oralidade, os verbos em concordância com pronomes pessoais/nomes sujeitos da oração.

\BNCC{EF05LP07} Identificar: adição, oposição, tempo, causa, condição, finalidade;
%Identificar, em textos, o uso de conjunções e a relação que estabelecem entre partes do texto: adição, oposição, tempo, causa, condição, finalidade.

\BNCC{EF05LP08} Identificar: sufixo, prefixo; palavras primitivas e compostar; Gramática;
%Diferenciar palavras primitivas, derivadas e compostas, e derivadas por adição de prefixo e de sufixo.

\BNCC{EF05LP09} Ler: textos instrucional de regras de jogo; Compreensão de texto;
%Ler e compreender, com autonomia, textos instrucional de regras de jogo, dentre outros gêneros do campo da vida cotidiana, de acordo com as convenções do gênero e considerando a situação comunicativa e a finalidade do texto.

\BNCC{EF05LP10} Ler: entender anedotas, piadas e cartuns;
%Ler e compreender, com autonomia, anedotas, piadas e cartuns, dentre outros gêneros do campo da vida cotidiana, de acordo com as convenções do gênero e considerando a situação comunicativa e a finalidade do texto.

\BNCC{EF05LP11} Produzir: Registrar anedotas, piadas e cartuns;
%Registrar, com autonomia, anedotas, piadas e cartuns, dentre outros gêneros do campo da vida cotidiana, de acordo com as convenções do gênero e considerando a situação comunicativa e a finalidade do texto.

\BNCC{EF05LP12} Produzir: textos instrucionais de regras de jogo;
%Planejar e produzir, com autonomia, textos instrucionais de regras de jogo, dentre outros gêneros do campo da vida cotidiana, de acordo com as convenções do gênero e considerando a situação comunicativa e a finalidade do texto.

\BNCC{EF05LP13} Identificar +Assistir:  vlog sobre brinquedos e livros infantis; Internet;
%Assistir, em vídeo digital, a postagem de vlog infantil de críticas de brinquedos e livros de literatura infantil e, a partir dele, planejar e produzir resenhas digitais em áudio ou vídeo.

\BNCC{EF05LP14} Identificar: formatação de resenha sobre brinquedo. 
%Identificar e reproduzir, em textos de resenha crítica de brinquedos ou livros de literatura infantil, a formatação própria desses textos (apresentação e avaliação do produto).

\BNCC{EF05LP15} Ler +Assistir: Notícias, reportagens; Jornais; Internet;
%Ler/assistir e compreender, com autonomia, notícias, reportagens, vídeos em vlogs argumentativos, dentre outros gêneros do campo político-cidadão, de acordo com as convenções dos gêneros e considerando a situação comunicativa e o tema/assunto do texto.

\BNCC{EF05LP16} Identificar: comparar fatos na mídia; fake-news; Jornais; Internet;
%Comparar informações sobre um mesmo fato veiculadas em diferentes mídias e concluir sobre qual é mais confiável e por quê.

\BNCC{EF05LP17} Produzir: roteiro de reportagem;
%Produzir roteiro para edição de uma reportagem digital sobre temas de interesse da turma, a partir de buscas de informações, imagens, áudios e vídeos na internet, de acordo com as convenções do gênero e considerando a situação comunicativa e o tema/assunto do texto.

\BNCC{EF05LP18} Produzir: vídeo para vlogs; "influencer"; 
%Roteirizar, produzir e editar vídeo para vlogs argumentativos sobre produtos de mídia para público infantil (filmes, desenhos animados, HQs, games etc.), com base em conhecimentos sobre os mesmos, de acordo com as convenções do gênero e considerando a situação comunicativa e o tema/ assunto/finalidade do texto.

\BNCC{EF05LP19} Falar: discussão; argumentar oralmente; "roda"; Opinião; Internet; 
%Argumentar oralmente sobre acontecimentos de interesse social, com base em conhecimentos sobre fatos divulgados em TV, rádio, mídia impressa e digital, respeitando pontos de vista diferentes.

\BNCC{EF05LP20} Identificar +Analisar: argumentos sobre produtos;
%Analisar a validade e força de argumentos em argumentações sobre produtos de mídia para público infantil (filmes, desenhos animados, HQs, games etc.), com base em conhecimentos sobre os mesmos.

\BNCC{EF05LP21} Identificar +Analisar: tom e expressão facial de influencers;
%Analisar o padrão entonacional, a expressão facial e corporal e as escolhas de variedade e registro linguísticos de vloggers de vlogs opinativos ou argumentativos.

\BNCC{EF05LP22} Ler: verbetes/dicionário; informações gramaticais; Gramática; 
%Ler e compreender verbetes de dicionário, identificando a estrutura, as informações gramaticais (significado de abreviaturas) e as informações semânticas.

\BNCC{EF05LP23} Identificar: comparar gráficos e tabelas;
%Comparar informações apresentadas em gráficos ou tabelas.

\BNCC{EF05LP24} Produzir: redação; resultados de pesquisa na internet; gráficos e tabelas;
%Planejar e produzir texto sobre tema de interesse, organizando resultados de pesquisa em fontes de informação impressas ou digitais, incluindo imagens e gráficos ou tabelas, considerando a situação comunicativa e o tema/assunto do texto.

\BNCC{EF05LP25} Produzir: verbetes/dicionário; finalidade do texto;
%Planejar e produzir, com certa autonomia, verbetes de dicionário, digitais ou impressos, considerando a situação comunicativa e o tema/assunto/finalidade do texto.

\BNCC{EF05LP26} Produzir: redação; regras sintáticas de concordância nominal e verbal; Gramática;
%Utilizar, ao produzir o texto, conhecimentos linguísticos e gramaticais: regras sintáticas de concordância nominal e verbal, convenções de escrita de citações, pontuação (ponto final, dois-pontos, vírgulas em enumerações) e regras ortográficas.

\BNCC{EF05LP27} Produzir; redação; pronomes anafóricos; relações de sentido; Gramática
%Utilizar, ao produzir o texto, recursos de coesão pronominal (pronomes anafóricos) e articuladores de relações de sentido (tempo, causa, oposição, conclusão, comparação), com nível adequado de informatividade.

\BNCC{EF05LP28} Identificar: poemas visuais digitais; minicontos; Internet;
%Observar, em ciberpoemas e minicontos infantis em mídia digital, os recursos multissemióticos presentes nesses textos digitais.


++++++++++++++++++++++++++++++++++++++++++++++++++++++++++++++++++++++++
++++++++++++++++++++++++++++++++++++++++++++++++++++++++++++++++++++++++
++++++++++++++++++++++++++++++++++++++++++++++++++++++++++++++++++++++++



\BNCC{EF15AR01} Ler +Apreciar: apreciar formas distintas das artes visuais;
%Identificar e apreciar formas distintas das artes visuais tradicionais e contemporâneas, cultivando a percepção, o imaginário, a capacidade de simbolizar e o repertório imagético.

\BNCC{EF15AR02} Identificar: ponto, linha, forma, cor, espaço, movimento etc;
%Explorar e reconhecer elementos constitutivos das artes visuais (ponto, linha, forma, cor, espaço, movimento etc.).

\BNCC{EF15AR03} Identificar: a influência de distintas matrizes estéticas e culturais;
%Reconhecer e analisar a influência de distintas matrizes estéticas e culturais das artes visuais nas manifestações artísticas das culturas locais, regionais e nacionais.

\BNCC{EF15AR04} Produzir: desenho, pintura, colagem, quadrinhos, dobradura, escultura, modelagem...; 
%Experimentar diferentes formas de expressão artística (desenho, pintura, colagem, quadrinhos, dobradura, escultura, modelagem, instalação, vídeo, fotografia etc.), fazendo uso sustentável de materiais, instrumentos, recursos e técnicas convencionais e não convencionais.

\BNCC{EF15AR05} Produzir: em +grupo +sozinho; diferentes espaços da escola e da comunidade;
%Experimentar a criação em artes visuais de modo individual, coletivo e colaborativo, explorando diferentes espaços da escola e da comunidade.

\BNCC{EF15AR06} Falar +Dialogar: sobre a sua criação e as dos colegas; "roda";
%Dialogar sobre a sua criação e as dos colegas, para alcançar sentidos plurais.

\BNCC{EF15AR07} Identificar: museus, galerias, instituições, artistas, artesãos, curadores etc;
%Reconhecer algumas categorias do sistema das artes visuais (museus, galerias, instituições, artistas, artesãos, curadores etc.).

\BNCC{EF15AR08} Produzir: dança;
%Experimentar e apreciar formas distintas de manifestações da dança presentes em diferentes contextos, cultivando a percepção, o imaginário, a capacidade de simbolizar e o repertório corporal.

\BNCC{EF15AR09} Identificar: parted do corpo na construção do movimento; 
%Estabelecer relações entre as partes do corpo e destas com o todo corporal na construção do movimento dançado.

\BNCC{EF15AR10} Produzir: orientação no espaço: deslocamentos, planos, direções, caminhos;
%Experimentar diferentes formas de orientação no espaço (deslocamentos, planos, direções, caminhos etc.) e ritmos de movimento (lento, moderado e rápido) na construção do movimento dançado.

\BNCC{EF15AR11} Produzir: +sozinho, +grupo; dança com base nos códigos de dança; 
%Criar e improvisar movimentos dançados de modo individual, coletivo e colaborativo, considerando os aspectos estruturais, dinâmicos e expressivos dos elementos constitutivos do movimento, com base nos códigos de dança.

\BNCC{EF15AR12} Falar +Discutir: as experiências pessoais e coletivas em dança vivenciadas; Opinião;
%Discutir, com respeito e sem preconceito, as experiências pessoais e coletivas em dança vivenciadas na escola, como fonte para a construção de vocabulários e repertórios próprios.

\BNCC{EF15AR13} Ler +Apreciar: "escutar músicas" de gêneros diferentes reconhecendo a função;
%Identificar e apreciar criticamente diversas formas e gêneros de expressão musical, reconhecendo e analisando os usos e as funções da música em diversos contextos de circulação, em especial, aqueles da vida cotidiana.

\BNCC{EF15AR14} Identificar: música: altura, intensidade, timbre, melodia, ritmo etc; 
%Perceber e explorar os elementos constitutivos da música (altura, intensidade, timbre, melodia, ritmo etc.), por meio de jogos, brincadeiras, canções e práticas diversas de composição/criação, execução e apreciação musical.

\BNCC{EF15AR15} Falar +Cantar: bater palma, voz, percussão; 
%Explorar fontes sonoras diversas, como as existentes no próprio corpo (palmas, voz, percussão corporal), na natureza e em objetos cotidianos, reconhecendo os elementos constitutivos da música e as características de instrumentos musicais variados.

\BNCC{EF15AR16} Identificar: música: representação gráfica de sons;
%Explorar diferentes formas de registro musical não convencional (representação gráfica de sons, partituras criativas etc.), bem como procedimentos e técnicas de registro em áudio e audiovisual, e reconhecer a notação musical convencional.

\BNCC{EF15AR17} Produzir: improvisações, composições e sonorização de histórias; 
%Experimentar improvisações, composições e sonorização de histórias, entre outros, utilizando vozes, sons corporais e/ou instrumentos musicais convencionais ou não convencionais, de modo individual, coletivo e colaborativo.

\BNCC{EF15AR18} Identificar +Reconhecer: Teatro presentes em diferentes contextos; "teatro de rua";
%Reconhecer e apreciar formas distintas de manifestações do teatro presentes em diferentes contextos, aprendendo a ver e a ouvir histórias dramatizadas e cultivando a percepção, o imaginário, a capacidade de simbolizar e o repertório ficcional.

\BNCC{EF15AR19} Identificar: Descobrir teatralidades na vida cotidiana;
%Descobrir teatralidades na vida cotidiana, identificando elementos teatrais (variadas entonações de voz, diferentes fisicalidades, diversidade de personagens e narrativas etc.).

\BNCC{EF15AR20} Produzir: improvisações teatrais;
%Experimentar o trabalho colaborativo, coletivo e autoral em improvisações teatrais e processos narrativos criativos em teatro, explorando desde a teatralidade dos gestos e das ações do cotidiano até elementos de diferentes matrizes estéticas e culturais.

\BNCC{EF15AR21} Produzir: imitação e o faz de conta; encenar com música, imagem; 
%Exercitar a imitação e o faz de conta, ressignificando objetos e fatos e experimentando-se no lugar do outro, ao compor e encenar acontecimentos cênicos, por meio de músicas, imagens, textos ou outros pontos de partida, de forma intencional e reflexiva.

\BNCC{EF15AR22} Produzir: diferentes movimentos e voz;
%Experimentar possibilidades criativas de movimento e de voz na criação de um personagem teatral, discutindo estereótipos.

\BNCC{EF15AR23} Identificar: relação entre diferentes/diversas linguagens artísticas;
%Reconhecer e experimentar, em projetos temáticos, as relações processuais entre diversas linguagens artísticas.

\BNCC{EF15AR24} Produzir: experimentar brinquedos, brincadeiras, jogos, danças, canções e histórias;
%Caracterizar e experimentar brinquedos, brincadeiras, jogos, danças, canções e histórias de diferentes matrizes estéticas e culturais.

\BNCC{EF15AR25} Ler +Valorizar: patrimônio cultural, material e imaterial, de culturas diversas; 
%Conhecer e valorizar o patrimônio cultural, material e imaterial, de culturas diversas, em especial a brasileira, incluindo-se suas matrizes indígenas, africanas e europeias, de diferentes épocas, favorecendo a construção de vocabulário e repertório relativos às diferentes linguagens artísticas.

\BNCC{EF15AR26} Identificar: diferentes tecnologias e recursos digitais; OER/REA;
%Explorar diferentes tecnologias e recursos digitais (multimeios, animações, jogos eletrônicos, gravações em áudio e vídeo, fotografia, softwares etc.) nos processos de criação artística.

\BNCC{EF12EF01} Produzir: diferentes brincadeiras e jogos da cultura popular;
%Experimentar, fruir e recriar diferentes brincadeiras e jogos da cultura popular presentes no contexto comunitário e regional, reconhecendo e respeitando as diferenças individuais de desempenho dos colegas.

\BNCC{EF12EF02} Identificar: brincadeiras e os jogos populares com corporal, visual, oral e escrita;
%Explicar, por meio de múltiplas linguagens (corporal, visual, oral e escrita), as brincadeiras e os jogos populares do contexto comunitário e regional, reconhecendo e valorizando a importância desses jogos e brincadeiras para suas culturas de origem.

\BNCC{EF12EF03} Produzir: estratégias para resolver desafios de brincadeiras e jogos populares;
%Planejar e utilizar estratégias para resolver desafios de brincadeiras e jogos populares do contexto comunitário e regional, com base no reconhecimento das características dessas práticas.

\BNCC{EF12EF04} Produzir: textos (orais, escritos, audiovisuais) para explicar brincadeiras;
%Colaborar na proposição e na produção de alternativas para a prática, em outros momentos e espaços, de brincadeiras e jogos e demais práticas corporais tematizadas na escola, produzindo textos (orais, escritos, audiovisuais) para divulgá-las na escola e na comunidade.

\BNCC{EF12EF05} Produzir: jogar e explicar diferentes tipos de esportes;
%Experimentar e fruir, prezando pelo trabalho coletivo e pelo protagonismo, a prática de esportes de marca e de precisão, identificando os elementos comuns a esses esportes.

\BNCC{EF12EF06} Falar +Discutir: normas e das regras dos esportes;
%Discutir a importância da observação das normas e das regras dos esportes de marca e de precisão para assegurar a integridade própria e as dos demais participantes.

\BNCC{EF12EF07} Produzir: equilíbrios, saltos, giros, rotações, acrobacias;
%Experimentar, fruir e identificar diferentes elementos básicos da ginástica (equilíbrios, saltos, giros, rotações, acrobacias, com e sem materiais) e da ginástica geral, de forma individual e em pequenos grupos, adotando procedimentos de segurança.

\BNCC{EF12EF08} Produzir: estratégias para a execução de ginásticas; 
%Planejar e utilizar estratégias para a execução de diferentes elementos básicos da ginástica e da ginástica geral.

\BNCC{EF12EF09} Produzir: fazer ginástica respeitando as diferenças individuais;
%Participar da ginástica geral, identificando as potencialidades e os limites do corpo, e respeitando as diferenças individuais e de desempenho corporal.

\BNCC{EF12EF10} Produzir +Descrever: as linguagens corporais; corporal, oral, escrita e audiovisual;
%Descrever, por meio de múltiplas linguagens (corporal, oral, escrita e audiovisual), as características dos elementos básicos da ginástica e da ginástica geral, identificando a presença desses elementos em distintas práticas corporais.

\BNCC{EF12EF11} Produzir: assistir danças comunitárias; 
%Experimentar e fruir diferentes danças do contexto comunitário e regional (rodas cantadas, brincadeiras rítmicas e expressivas), e recriá-las, respeitando as diferenças individuais e de desempenho corporal.

\BNCC{EF12EF12} Identificar: elementos: ritmo, espaço, gestos das danças comunitárias e reginais;
%Identificar os elementos constitutivos (ritmo, espaço, gestos) das danças do contexto comunitário e regional, valorizando e respeitando as manifestações de diferentes culturas.

\BNCC{EF35EF01} Produzir: brincadeiras e jogos populares do Brasil e do mundo; patrimônio cultural;
%Experimentar e fruir brincadeiras e jogos populares do Brasil e do mundo, incluindo aqueles de matriz indígena e africana, e recriá-los, valorizando a importância desse patrimônio histórico cultural.

\BNCC{EF35EF02} Produzir: brincadeiras de matriz indígena e africana;
%Planejar e utilizar estratégias para possibilitar a participação segura de todos os alunos em brincadeiras e jogos populares do Brasil e de matriz indígena e africana.

\BNCC{EF35EF03} Produzir: brincadeiras de matriz indígena e africana e a importância desse patrimônio;
%Descrever, por meio de múltiplas linguagens (corporal, oral, escrita, audiovisual), as brincadeiras e os jogos populares do Brasil e de matriz indígena e africana, explicando suas características e a importância desse patrimônio histórico cultural na preservação das diferentes culturas.

\BNCC{EF35EF04} Produzir: reinventar brincadeiras de matriz indígena e africana;
%Recriar, individual e coletivamente, e experimentar, na escola e fora dela, brincadeiras e jogos populares do Brasil e do mundo, incluindo aqueles de matriz indígena e africana, e demais práticas corporais tematizadas na escola, adequando-as aos espaços públicos disponíveis.

\BNCC{EF35EF05} Produzir: jogar diversos tipos de esporte de campo e taco;
%Experimentar e fruir diversos tipos de esportes de campo e taco, rede/parede e invasão, identificando seus elementos comuns e criando estratégias individuais e coletivas básicas para sua execução, prezando pelo trabalho coletivo e pelo protagonismo.

\BNCC{EF35EF06} Identificar: conceitos de jogo e esporte ontem e hoje;
%Diferenciar os conceitos de jogo e esporte, identificando as características que os constituem na contemporaneidade e suas manifestações (profissional e comunitária/lazer).

\BNCC{EF35EF07} Produzir: coreografias com equilíbrios, saltos, giros, rotações, acrobacias;
%Experimentar e fruir, de forma coletiva, combinações de diferentes elementos da ginástica geral (equilíbrios, saltos, giros, rotações, acrobacias, com e sem materiais), propondo coreografias com diferentes temas do cotidiano.

\BNCC{EF35EF08} Produzir: resolver desafios para fazer ginástica e vendo os limites de cada um;
%Planejar e utilizar estratégias para resolver desafios na execução de elementos básicos de apresentações coletivas de ginástica geral, reconhecendo as potencialidades e os limites do corpo e adotando procedimentos de segurança.

\BNCC{EF35EF09} Produzir: reinventar brincadeiras de matriz indígena e africana respeitando sentidos;
%Experimentar, recriar e fruir danças populares do Brasil e do mundo e danças de matriz indígena e africana, valorizando e respeitando os diferentes sentidos e significados dessas danças em suas culturas de origem.

\BNCC{EF35EF10} Identificar +Comparar: elementos ritmo, espaço, gestos em danças populares do Brasil;
%Comparar e identificar os elementos constitutivos comuns e diferentes (ritmo, espaço, gestos) em danças populares do Brasil e do mundo e danças de matriz indígena e africana.

\BNCC{EF35EF11} Produzir: utilizar estratégias para a execução de elementos das danças populares;
%Formular e utilizar estratégias para a execução de elementos constitutivos das danças populares do Brasil e do mundo, e das danças de matriz indígena e africana.

\BNCC{EF35EF12} Identificar: situações de injustiça e preconceito nas danças;
%Identificar situações de injustiça e preconceito geradas e/ou presentes no contexto das danças e demais práticas corporais e discutir alternativas para superá-las.

\BNCC{EF35EF13} Produzir: lutas comunitárias como tema nas danças de matriz indígena e africana;
%Experimentar, fruir e recriar diferentes lutas presentes no contexto comunitário e regional e lutas de matriz indígena e africana.

\BNCC{EF35EF14} Produzir: lutas do contexto comunitário respeitando o colega; "capoeira";
%Planejar e utilizar estratégias básicas das lutas do contexto comunitário e regional e lutas de matriz indígena e africana experimentadas, respeitando o colega como oponente e as normas de segurança.

\BNCC{EF35EF15} Identificar: características das lutas do contexto comunitário e regional;
%Identificar as características das lutas do contexto comunitário e regional e lutas de matriz indígena e africana, reconhecendo as diferenças entre lutas e brigas e entre lutas e as demais práticas corporais.
Matemática}

+++++++++++++++++++++++++++++++++++++++++++++++++++++++++++++++
+++++++++++++++++++++++++++++++++++++++++++++++++++++++++++++++
+++++++++++++++++++++++++++++++++++++++++++++++++++++++++++++++

\BNCC{EF01MA01} Ler: números naturais; reconhecer ordens de números ou códigos de identificação;
%Utilizar números naturais como indicador de quantidade ou de ordem em diferentes situações cotidianas e reconhecer situações em que os números não indicam contagem nem ordem, mas sim código de identificação.
Matemática}

\BNCC{EF01MA02} Ler: contar; agrupar;
%Contar de maneira exata ou aproximada, utilizando diferentes estratégias como o pareamento e outros agrupamentos.
Matemática}

\BNCC{EF01MA03} Ler: comparar quantidades de objetos;
%Estimar e comparar quantidades de objetos de dois conjuntos (em torno de 20 elementos), por estimativa e/ou por correspondência (um a um, dois a dois) para indicar “tem mais”, “tem menos” ou “tem a mesma quantidade”.
Matemática}

\BNCC{EF01MA04} Produzir: contar objetos e apresentar resultados; falar;
%Contar a quantidade de objetos de coleções até 100 unidades e apresentar o resultado por registros verbais e simbólicos, em situações de seu interesse, como jogos, brincadeiras, materiais da sala de aula, entre outros.
Matemática}

\BNCC{EF01MA05} Ler: comparar números em situações cotidianas;
%Comparar números naturais de até duas ordens em situações cotidianas, com e sem suporte da reta numérica.
Matemática}

\BNCC{EF01MA06} Ler: adição; cálculo;
%Construir fatos básicos da adição e utilizá-los em procedimentos de cálculo para resolver problemas.
Matemática}

\BNCC{EF01MA07} Ler: adição; compreensão da numeração decimal; cálculo;
%Compor e decompor número de até duas ordens, por meio de diferentes adições, com o suporte de material manipulável, contribuindo para a compreensão de características do sistema de numeração decimal e o desenvolvimento de estratégias de cálculo.
Matemática}

\BNCC{EF01MA08} Ler: adição; subtração; resolver problemas;
%Resolver e elaborar problemas de adição e de subtração, envolvendo números de até dois algarismos, com os significados de juntar, acrescentar, separar e retirar, com o suporte de imagens e/ou material manipulável, utilizando estratégias e formas de registro pessoais.
Matemática}

\BNCC{EF01MA09} Identificar: organizar; ordenar objetos; cor, formas;
%Organizar e ordenar objetos familiares ou representações por figuras, por meio de atributos, tais como cor, forma e medida.
Matemática}

\BNCC{EF01MA10} Identificar: descrever padrões de números;
%Descrever, após o reconhecimento e a explicitação de um padrão (ou regularidade), os elementos ausentes em sequências recursivas de números naturais, objetos ou figuras.
Matemática}

\BNCC{EF01MA11}  Identificar: descrever a localização de pessoas e objetos; 
%Descrever a localização de pessoas e de objetos no espaço em relação à sua própria posição, utilizando termos como à direita, à esquerda, em frente, atrás.
Matemática}

\BNCC{EF01MA12} Identificar: descrever a localização de pessoas e objetos; ponto de referência espacial;
%Descrever a localização de pessoas e de objetos no espaço segundo um dado ponto de referência, compreendendo que, para a utilização de termos que se referem à posição, como direita, esquerda, em cima, em baixo, é necessário explicitar-se o referencial.
Matemática}

\BNCC{EF01MA13} Identificar:  relacionar figuras geométricas;
%Relacionar figuras geométricas espaciais (cones, cilindros, esferas e blocos retangulares) a objetos familiares do mundo físico.
Matemática}

\BNCC{EF01MA14} Identificar: nomear figuras; sólidos geométricos;
%Identificar e nomear figuras planas (círculo, quadrado, retângulo e triângulo) em desenhos apresentados em diferentes disposições ou em contornos de faces de sólidos geométricos.
Matemática}

\BNCC{EF01MA15} Identificar: comparar comprimentos; ordenar objetos;
%Comparar comprimentos, capacidades ou massas, utilizando termos como mais alto, mais baixo, mais comprido, mais curto, mais grosso, mais fino, mais largo, mais pesado, mais leve, cabe mais, cabe menos, entre outros, para ordenar objetos de uso cotidiano.
Matemática}

\BNCC{EF01MA16} Produzir: falar; relatar; sequência de acontecimentos em horários;
%Relatar em linguagem verbal ou não verbal sequência de acontecimentos relativos a um dia, utilizando, quando possível, os horários dos eventos.
Matemática}

\BNCC{EF01MA17} Identificar: reconhecer períodos do dia; temporalidade;
%Reconhecer e relacionar períodos do dia, dias da semana e meses do ano, utilizando calendário, quando necessário.
Matemática}

\BNCC{EF01MA18} Produzir: escrever datas; consultar calendários;
%Produzir a escrita de uma data, apresentando o dia, o mês e o ano, e indicar o dia da semana de uma data, consultando calendários.
Matemática}

\BNCC{EF01MA19} Identificar: reconhecer valores de moedas; resolver situações monetárias simples;
%Reconhecer e relacionar valores de moedas e cédulas do sistema monetário brasileiro para resolver situações simples do cotidiano do estudante.
Matemática}

\BNCC{EF01MA20} Identificar: classificar eventos; probabilidade;
%Classificar eventos envolvendo o acaso, tais como “acontecerá com certeza”, “talvez aconteça” e “é impossível acontecer”, em situações do cotidiano.
Matemática}

\BNCC{EF01MA21} Ler: ler dados; tabelas; gráficos;
%Ler dados expressos em tabelas e em gráficos de colunas simples.
Matemática}

\BNCC{EF01MA22} Produzir: pesquisar; organizar dados;
%Realizar pesquisa, envolvendo até duas variáveis categóricas de seu interesse e universo de até 30 elementos, e organizar dados por meio de representações pessoais.
Matemática}

\BNCC{EF02MA01} Identificar: comparar e ordenar números;
%Comparar e ordenar números naturais (até a ordem de centenas) pela compreensão de características do sistema de numeração decimal (valor posicional e função do zero).
Matemática}

\BNCC{EF02MA02} Produzir: fazer estimativas; registrar resultado da contagem de objetos;
%Fazer estimativas por meio de estratégias diversas a respeito da quantidade de objetos de coleções e registrar o resultado da contagem desses objetos (até 1000 unidades).
Matemática}

\BNCC{EF02MA03} Identificar: comparar quantidades de objetos por conjuntos; estimativa;
%Comparar quantidades de objetos de dois conjuntos, por estimativa e/ou por correspondência (um a um, dois a dois, entre outros), para indicar “tem mais”, “tem menos” ou “tem a mesma quantidade”, indicando, quando for o caso, quantos a mais e quantos a menos.
Matemática}

\BNCC{EF02MA04} Ler: compor e decompor números; adição;
%Compor e decompor números naturais de até três ordens, com suporte de material manipulável, por meio de diferentes adições.
Matemática}

\BNCC{EF02MA05} Ler: adição; cálculo mental ou escrito;
%Construir fatos básicos da adição e subtração e utilizá-los no cálculo mental ou escrito.
Matemática}

\BNCC{EF02MA06} Produzir: resolver; elaborar problemas matemáticos; adição; subtração;
%Resolver e elaborar problemas de adição e de subtração, envolvendo números de até três ordens, com os significados de juntar, acrescentar, separar, retirar, utilizando estratégias pessoais ou convencionais.
Matemática}

\BNCC{EF02MA07} Produzir: resolver problemas de multiplicação;
%Resolver e elaborar problemas de multiplicação (por 2, 3, 4 e 5) com a ideia de adição de parcelas iguais por meio de estratégias e formas de registro pessoais, utilizando ou não suporte de imagens e/ou material manipulável.
Matemática}

\BNCC{EF02MA08} Produzir: resolver problemas matemáticos; dobro; metade;
%Resolver e elaborar problemas envolvendo dobro, metade, triplo e terça parte, com o suporte de imagens ou material manipulável, utilizando estratégias pessoais.
Matemática}

\BNCC{EF02MA09} Produzir: construir sequências de números em ordem crescente ou decrescente;
%Construir sequências de números naturais em ordem crescente ou decrescente a partir de um número qualquer, utilizando uma regularidade estabelecida.
Matemática}

\BNCC{EF02MA10} Produzir: descrever um padrão; sequências;
%Descrever um padrão (ou regularidade) de sequências repetitivas e de sequências recursivas, por meio de palavras, símbolos ou desenhos.
Matemática}

\BNCC{EF02MA11} Produzir: descrever elemementos ausentes em sequências;
%Descrever os elementos ausentes em sequências repetitivas e em sequências recursivas de números naturais, objetos ou figuras.
Matemática}

\BNCC{EF02MA12} Produzir: registrar a localização e deslocamentos de pessoas;
%Identificar e registrar, em linguagem verbal ou não verbal, a localização e os deslocamentos de pessoas e de objetos no espaço, considerando mais de um ponto de referência, e indicar as mudanças de direção e de sentido.
Matemática}

\BNCC{EF02MA13} Produzir: esboçar roteiros; desenhar plantas de ambientes;
%Esboçar roteiros a ser seguidos ou plantas de ambientes familiares, assinalando entradas, saídas e alguns pontos de referência.
Matemática}

\BNCC{EF02MA14} Identificar: nomear e comparar figuras geométricas; 
%Reconhecer, nomear e comparar figuras geométricas espaciais (cubo, bloco retangular, pirâmide, cone, cilindro e esfera), relacionando-as com objetos do mundo físico.
Matemática}

\BNCC{EF02MA15} Identificar: nomear e comparar figuras planas;
%Reconhecer, comparar e nomear figuras planas (círculo, quadrado, retângulo e triângulo), por meio de características comuns, em desenhos apresentados em diferentes disposições ou em sólidos geométricos.
Matemática}

\BNCC{EF02MA16} Identificar: medir; comparar comprimentos;
%Estimar, medir e comparar comprimentos de lados de salas (incluindo contorno) e de polígonos, utilizando unidades de medida não padronizadas e padronizadas (metro, centímetro e milímetro) e instrumentos adequados.
Matemática}

\BNCC{EF02MA17} Identificar: medir; comparar pesos;
%Estimar, medir e comparar capacidade e massa, utilizando estratégias pessoais e unidades de medida não padronizadas ou padronizadas (litro, mililitro, grama e quilograma).
Matemática}

\BNCC{EF02MA18} Identificar: indicar duração de intervalos de tempos;
%Indicar a duração de intervalos de tempo entre duas datas, como dias da semana e meses do ano, utilizando calendário, para planejamentos e organização de agenda.
Matemática}

\BNCC{EF02MA19} Produzir: medir a duração de intervalo de tempo; usar um relógio;
%Medir a duração de um intervalo de tempo por meio de relógio digital e registrar o horário do início e do fim do intervalo.
Matemática}

\BNCC{EF02MA20} Identificar: equivalência de valores; 
%Estabelecer a equivalência de valores entre moedas e cédulas do sistema monetário brasileiro para resolver situações cotidianas.
Matemática}

\BNCC{EF02MA21} Identificar: classificar resultados de eventos cotidianos; probabilidade;
%Classificar resultados de eventos cotidianos aleatórios como “pouco prováveis”, “muito prováveis”, “improváveis” e “impossíveis”.
Matemática}

\BNCC{EF02MA22} Identificar: comparar informações; gráficos; tabelas;
%Comparar informações de pesquisas apresentadas por meio de tabelas de dupla entrada e em gráficos de colunas simples ou barras, para melhor compreender aspectos da realidade próxima.
Matemática}

\BNCC{EF02MA23} Produzir: pesquisar; escolher variáveis categóricas;
%Realizar pesquisa em universo de até 30 elementos, escolhendo até três variáveis categóricas de seu interesse, organizando os dados coletados em listas, tabelas e gráficos de colunas simples.
Matemática}

\BNCC{EF03MA01} Ler: comparar números; unidade de milhar;
%Ler, escrever e comparar números naturais de até a ordem de unidade de milhar, estabelecendo relações entre os registros numéricos e em língua materna.
Matemática}

\BNCC{EF03MA02} Identificar: numeração decimal; 
%Identificar características do sistema de numeração decimal, utilizando a composição e a decomposição de número natural de até quatro ordens.
Matemática}

\BNCC{EF03MA03} Produzir: construir fatos de adição e multiplicação;
%Construir e utilizar fatos básicos da adição e da multiplicação para o cálculo mental ou escrito.
Matemática}

\BNCC{EF03MA04} Ler: relacionar números naturais e reta numérica;
%Estabelecer a relação entre números naturais e pontos da reta numérica para utilizá-la na ordenação dos números naturais e também na construção de fatos da adição e da subtração, relacionando-os com deslocamentos para a direita ou para a esquerda.
Matemática}

\BNCC{EF03MA05} Produzir: resolver problemas de adição e subtração; cálculo mental e escrito;
%Utilizar diferentes procedimentos de cálculo mental e escrito para resolver problemas significativos envolvendo adição e subtração com números naturais.
Matemática}

\BNCC{EF03MA06} Produzir: resolver problemas de adição e subtração; juntar; separar; comparar quantidades;
%Resolver e elaborar problemas de adição e subtração com os significados de juntar, acrescentar, separar, retirar, comparar e completar quantidades, utilizando diferentes estratégias de cálculo exato ou aproximado, incluindo cálculo mental.
Matemática}

\BNCC{EF03MA07} Produzir: resolver problema de multiplicação;
%Resolver e elaborar problemas de multiplicação (por 2, 3, 4, 5 e 10) com os significados de adição de parcelas iguais e elementos apresentados em disposição retangular, utilizando diferentes estratégias de cálculo e registros.
Matemática}

\BNCC{EF03MA08} Produzir; resolver problemas de divisão;
%Resolver e elaborar problemas de divisão de um número natural por outro (até 10), com resto zero e com resto diferente de zero, com os significados de repartição equitativa e de medida, por meio de estratégias e registros pessoais.
Matemática}

\BNCC{EF03MA09} Ler: associar o quociente de uma divisão;
%Associar o quociente de uma divisão com resto zero de um número natural por 2, 3, 4, 5 e 10 às ideias de metade, terça, quarta, quinta e décima partes.
Matemática}

\BNCC{EF03MA10} Identificar: regularidades de sequências ordenadas de números;
%Identificar regularidades em sequências ordenadas de números naturais, resultantes da realização de adições ou subtrações sucessivas, por um mesmo número, descrever uma regra de formação da sequência e determinar elementos faltantes ou seguintes.
Matemática}

\BNCC{EF03MA11} Ler: compreender a ideia de igualdade; soma; diferença;
%Compreender a ideia de igualdade para escrever diferentes sentenças de adições ou de subtrações de dois números naturais que resultem na mesma soma ou diferença.
Matemática}

\BNCC{EF03MA12} Produzir: esboços de trajetos; croquis; maquetes;
%Descrever e representar, por meio de esboços de trajetos ou utilizando croquis e maquetes, a movimentação de pessoas ou de objetos no espaço, incluindo mudanças de direção e sentido, com base em diferentes pontos de referência.
Matemática}

\BNCC{EF03MA13} Identificar: associar figuras geométricas a objetos do mundo físico;
%Associar figuras geométricas espaciais (cubo, bloco retangular, pirâmide, cone, cilindro e esfera) a objetos do mundo físico e nomear essas figuras.
Matemática}

\BNCC{EF03MA14} Produzir: descrever características de figuras geométricas;
%Descrever características de algumas figuras geométricas espaciais (prismas retos, pirâmides, cilindros, cones), relacionando-as com suas planificações.
Matemática}

\BNCC{EF03MA15} Identificar: classificar figuras planas;
%Classificar e comparar figuras planas (triângulo, quadrado, retângulo, trapézio e paralelogramo) em relação a seus lados (quantidade, posições relativas e comprimento) e vértices.
Matemática}

\BNCC{EF03MA16} Identificar: reconhecer figuras congruentes;
%Reconhecer figuras congruentes, usando sobreposição e desenhos em malhas quadriculadas ou triangulares, incluindo o uso de tecnologias digitais.
Matemática}

\BNCC{EF03MA17} Identificar: reconhecer unidade de medida;
%Reconhecer que o resultado de uma medida depende da unidade de medida utilizada.
Matemática}

\BNCC{EF03MA18} Identificar: escolher unidade de medida; instrumento de medição;
%Escolher a unidade de medida e o instrumento mais apropriado para medições de comprimento, tempo e capacidade.
Matemática}

\BNCC{EF03MA19} Identificar: comparar comprimentos; unidades de medida; 
%Estimar, medir e comparar comprimentos, utilizando unidades de medida não padronizadas e padronizadas mais usuais (metro, centímetro e milímetro) e diversos instrumentos de medida.
Matemática}

\BNCC{EF03MA20} Identificar: comparar pesos; unidades de medida;
%Estimar e medir capacidade e massa, utilizando unidades de medida não padronizadas e padronizadas mais usuais (litro, mililitro, quilograma, grama e miligrama), reconhecendo-as em leitura de rótulos e embalagens, entre outros.
Matemática}

\BNCC{EF03MA21} Identificar: comparar áreas de faces de objetos;
%Comparar, visualmente ou por superposição, áreas de faces de objetos, de figuras planas ou de desenhos.
Matemática}

\BNCC{EF03MA22} Ler: medidas e intervalos de tempo; relógios analógicos e digitais;
%Ler e registrar medidas e intervalos de tempo, utilizando relógios (analógico e digital) para informar os horários de início e término de realização de uma atividade e sua duração.
Matemática}

\BNCC{EF03MA23} Ler: relógios analógicos e digitais; reconhecer horas, minutos e segundos;
%Ler horas em relógios digitais e em relógios analógicos e reconhecer a relação entre hora e minutos e entre minuto e segundos.
Matemática}

\BNCC{EF03MA24} Produzir: resolver problemas com situações de compra, venda e troca;
%Resolver e elaborar problemas que envolvam a comparação e a equivalência de valores monetários do sistema brasileiro em situações de compra, venda e troca.
Matemática}

\BNCC{EF03MA25} Identificar: chances de ocorrência de eventos; probabilidades;
%Identificar, em eventos familiares aleatórios, todos os resultados possíveis, estimando os que têm maiores ou menores chances de ocorrência.
Matemática}

\BNCC{EF03MA26} Produzir: resolver problemas com dados apresentados em tabelas; gráficos;
%Resolver problemas cujos dados estão apresentados em tabelas de dupla entrada, gráficos de barras ou de colunas.
Matemática}

\BNCC{EF03MA27} Ler: interpretar dados em tabelas; 
%Ler, interpretar e comparar dados apresentados em tabelas de dupla entrada, gráficos de barras ou de colunas, envolvendo resultados de pesquisas significativas, utilizando termos como maior e menor frequência, apropriando-se desse tipo de linguagem para compreender aspectos da realidade sociocultural significativos.
Matemática}

\BNCC{EF03MA28} Produzir: pesquisar; organizar dados em listas; tabelas; gráficos;
%Realizar pesquisa envolvendo variáveis categóricas em um universo de até 50 elementos, organizar os dados coletados utilizando listas, tabelas simples ou de dupla entrada e representá-los em gráficos de colunas simples, com e sem uso de tecnologias digitais.
Matemática}

\BNCC{EF04MA01} Ler: ordenar números naturais;
%Ler, escrever e ordenar números naturais até a ordem de dezenas de milhar.
Matemática}

\BNCC{EF04MA02} Produzir: mostrar decomposição e composição de números;
%Mostrar, por decomposição e composição, que todo número natural pode ser escrito por meio de adições e multiplicações por potências de dez, para compreender o sistema de numeração decimal e desenvolver estratégias de cálculo.
Matemática}

\BNCC{EF04MA03} Produzir: resolver problemas matemáticos; adição e subtração; 
%Resolver e elaborar problemas com números naturais envolvendo adição e subtração, utilizando estratégias diversas, como cálculo, cálculo mental e algoritmos, além de fazer estimativas do resultado.
Matemática}

\BNCC{EF04MA04} Produzir: estratégias de cálculo, adição; subtração, multiplicação; divisão;
%Utilizar as relações entre adição e subtração, bem como entre multiplicação e divisão, para ampliar as estratégias de cálculo.
Matemática}

\BNCC{EF04MA05} Produzir: estratégias de cálculo
%Utilizar as propriedades das operações para desenvolver estratégias de cálculo.
Matemática}

\BNCC{EF04MA06} Produzir: resolver problemas matemáticos; proporcionalidade;
%Resolver e elaborar problemas envolvendo diferentes significados da multiplicação (adição de parcelas iguais, organização retangular e proporcionalidade), utilizando estratégias diversas, como cálculo por estimativa, cálculo mental e algoritmos.
Matemática}

\BNCC{EF04MA07} Produzir: elaborar problemas de divisão; 
%Resolver e elaborar problemas de divisão cujo divisor tenha no máximo dois algarismos, envolvendo os significados de repartição equitativa e de medida, utilizando estratégias diversas, como cálculo por estimativa, cálculo mental e algoritmos.
Matemática}


%parei aqui 1

\BNCC{EF04MA08} Produzir: resolver problemas simples de contagem; 
%Resolver, com o suporte de imagem e/ou material manipulável, problemas simples de contagem, como a determinação do número de agrupamentos possíveis ao se combinar cada elemento de uma coleção com todos os elementos de outra, utilizando estratégias e formas de registro pessoais.
Matemática}

\BNCC{EF04MA09} Identificar: reconhecer frações;
%Reconhecer as frações unitárias mais usuais (1/2, 1/3, 1/4, 1/5, 1/10 e 1/100) como unidades de medida menores do que uma unidade, utilizando a reta numérica como recurso.
Matemática}

\BNCC{EF04MA10} Identificar: reconhecer regras de numeração decimal;
%Reconhecer que as regras do sistema de numeração decimal podem ser estendidas para a representação decimal de um número racional e relacionar décimos e centésimos com a representação do sistema monetário brasileiro.
Matemática}

\BNCC{EF04MA11} Identificar: reconhecer regularidades em sequências numéricas;
%Identificar regularidades em sequências numéricas compostas por múltiplos de um número natural.
Matemática}

\BNCC{EF04MA12} Identificar: reconhecer grupos de números naturais;
%Reconhecer, por meio de investigações, que há grupos de números naturais para os quais as divisões por um determinado número resultam em restos iguais, identificando regularidades.
Matemática}

\BNCC{EF04MA13} Produzir: usar a calculadora para operações; adição; subtração; multiplicação; divisão;
%Reconhecer, por meio de investigações, utilizando a calculadora quando necessário, as relações inversas entre as operações de adição e de subtração e de multiplicação e de divisão, para aplicá-las na resolução de problemas.
Matemática}

\BNCC{EF04MA14} Identificar: reconhecer a relação de igualdade entre dois termos;
%Reconhecer e mostrar, por meio de exemplos, que a relação de igualdade existente entre dois termos permanece quando se adiciona ou se subtrai um mesmo número a cada um desses termos.
Matemática}

\BNCC{EF04MA15} Identificar: determinar o número que torna verdadeira uma igualdade;
%Determinar o número desconhecido que torna verdadeira uma igualdade que envolve as operações fundamentais com números naturais.
Matemática}

\BNCC{EF04MA16} Produzir: descrever deslocamentos e localização de pessoas e objetos;
%Descrever deslocamentos e localização de pessoas e de objetos no espaço, por meio de malhas quadriculadas e representações como desenhos, mapas, planta baixa e croquis, empregando termos como direita e esquerda, mudanças de direção e sentido, intersecção, transversais, paralelas e perpendiculares.
Matemática}

\BNCC{EF04MA17} Identificar: associar prismas e pirâmides;
%Associar prismas e pirâmides a suas planificações e analisar, nomear e comparar seus atributos, estabelecendo relações entre as representações planas e espaciais.
Matemática}

\BNCC{EF04MA18} Identificar: reconhecer ângulos retos e não retos; softwares de geometria;
%Reconhecer ângulos retos e não retos em figuras poligonais com o uso de dobraduras, esquadros ou softwares de geometria.
Matemática}

\BNCC{EF04MA19} Identificar: reconhecer simetria em figuras; softwares de geometria;
%Reconhecer simetria de reflexão em figuras e em pares de figuras geométricas planas e utilizá-la na construção de figuras congruentes, com o uso de malhas quadriculadas e de softwares de geometria.
Matemática}

\BNCC{EF04MA20} Identificar: medir comprimentos; unidades de medida;
%Medir e estimar comprimentos (incluindo perímetros), massas e capacidades, utilizando unidades de medida padronizadas mais usuais, valorizando e respeitando a cultura local.
Matemática}

\BNCC{EF04MA21} Identificar; medir áreas de figuras planas;
%Medir, comparar e estimar área de figuras planas desenhadas em malha quadriculada, pela contagem dos quadradinhos ou de metades de quadradinho, reconhecendo que duas figuras com formatos diferentes podem ter a mesma medida de área.
Matemática}

\BNCC{EF04MA22} Ler: medidas e intervalos de tempo; 
%Ler e registrar medidas e intervalos de tempo em horas, minutos e segundos em situações relacionadas ao seu cotidiano, como informar os horários de início e término de realização de uma tarefa e sua duração.
Matemática}

\BNCC{EF04MA23} Ler: reconhecer temperatura e grau Celsius; unidade de medida;
%Reconhecer temperatura como grandeza e o grau Celsius como unidade de medida a ela associada e utilizá-lo em comparações de temperaturas em diferentes regiões do Brasil ou no exterior ou, ainda, em discussões que envolvam problemas relacionados ao aquecimento global.
Matemática}

\BNCC{EF04MA24} Produzir: registrar temperaturas; elaborar gráficos;
%Registrar as temperaturas máxima e mínima diárias, em locais do seu cotidiano, e elaborar gráficos de colunas com as variações diárias da temperatura, utilizando, inclusive, planilhas eletrônicas.
Matemática}

\BNCC{EF04MA25} Produzir: resolver problemas de compra e venda; troco e desconto;
%Resolver e elaborar problemas que envolvam situações de compra e venda e formas de pagamento, utilizando termos como troco e desconto, enfatizando o consumo ético, consciente e responsável.

Matemática}

\BNCC{EF04MA26} Identificar: probabilidade de eventos;
%Identificar, entre eventos aleatórios cotidianos, aqueles que têm maior chance de ocorrência, reconhecendo características de resultados mais prováveis, sem utilizar frações.
Matemática}

\BNCC{EF04MA27} Ler: analisar dados; tabelas; gráficos;
%Analisar dados apresentados em tabelas simples ou de dupla entrada e em gráficos de colunas ou pictóricos, com base em informações das diferentes áreas do conhecimento, e produzir texto com a síntese de sua análise.
Matemática}

\BNCC{EF04MA28} Produzir: pesquisar; organizar dados; tabelas; gráficos;
%Realizar pesquisa envolvendo variáveis categóricas e numéricas e organizar dados coletados por meio de tabelas e gráficos de colunas simples ou agrupadas, com e sem uso de tecnologias digitais.
Matemática}

\BNCC{EF05MA01} Identificar: ordenar números naturais
%Ler, escrever e ordenar números naturais até a ordem das centenas de milhar com compreensão das principais características do sistema de numeração decimal.
Matemática}

\BNCC{EF05MA02} Identificar: ordenar números racionais
%Ler, escrever e ordenar números racionais na forma decimal com compreensão das principais características do sistema de numeração decimal, utilizando, como recursos, a composição e decomposição e a reta numérica.
Matemática}

\BNCC{EF05MA03} Identificar: representar frações distintas
%Identificar e representar frações (menores e maiores que a unidade), associando-as ao resultado de uma divisão ou à ideia de parte de um todo, utilizando a reta numérica como recurso.
Matemática}

\BNCC{EF05MA04} Identificar: representar frações equivalentes
%Identificar frações equivalentes.
Matemática}

\BNCC{EF05MA05} Identificar: comparar números racionais positivos; reta numérica;
%Comparar e ordenar números racionais positivos (representações fracionária e decimal), relacionando-os a pontos na reta numérica.
Matemática}

\BNCC{EF05MA06} Identificar: associar porcentagens;
%Associar as representações 10\%, 25\%, 50\%, 75\% e 100\% respectivamente à décima parte, quarta parte, metade, três quartos e um inteiro, para calcular porcentagens, utilizando estratégias pessoais, cálculo mental e calculadora, em contextos de educação financeira, entre outros.
Matemática}

\BNCC{EF05MA07} Produzir: resolver problemas de adição e subtração; representação decimal;
%Resolver e elaborar problemas de adição e subtração com números naturais e com números racionais, cuja representação decimal seja finita, utilizando estratégias diversas, como cálculo por estimativa, cálculo mental e algoritmos.
Matemática}

\BNCC{EF05MA08} Produzir: resolver problemas de multiplicação e divisão; representação decimal;
%Resolver e elaborar problemas de multiplicação e divisão com números naturais e com números racionais cuja representação decimal é finita (com multiplicador natural e divisor natural e diferente de zero), utilizando estratégias diversas, como cálculo por estimativa, cálculo mental e algoritmos.
Matemática}

\BNCC{EF05MA09} Produzir: resolver problemas de contagem; multiplicação; diagramas;
%Resolver e elaborar problemas simples de contagem envolvendo o princípio multiplicativo, como a determinação do número de agrupamentos possíveis ao se combinar cada elemento de uma coleção com todos os elementos de outra coleção, por meio de diagramas de árvore ou por tabelas.
Matemática}

\BNCC{EF05MA10} Produzir: investigar reações de igualdade; adição; subtração; 
%Concluir, por meio de investigações, que a relação de igualdade existente entre dois membros permanece ao adicionar, subtrair, multiplicar ou dividir cada um desses membros por um mesmo número, para construir a noção de equivalência.
Matemática}

\BNCC{EF05MA11} Produzir: resolver problemas matemáticos; conversão;
%Resolver e elaborar problemas cuja conversão em sentença matemática seja uma igualdade com uma operação em que um dos termos é desconhecido.
Matemática}

\BNCC{EF05MA12} Produzir: resolver problemas de proporcionalidade entre duas grandezas; escala em mapas;
%Resolver problemas que envolvam variação de proporcionalidade direta entre duas grandezas, para associar a quantidade de um produto ao valor a pagar, alterar as quantidades de ingredientes de receitas, ampliar ou reduzir escala em mapas, entre outros.
Matemática}

\BNCC{EF05MA13} Produzir: resolver problemas matemáticos; dividir quantidades em duas partes;
%Resolver problemas envolvendo a partilha de uma quantidade em duas partes desiguais, tais como dividir uma quantidade em duas partes, de modo que uma seja o dobro da outra, com compreensão da ideia de razão entre as partes e delas com o todo.
Matemática}

\BNCC{EF05MA14} Identificar: compreender diferentes representações; mapas; coordenadas geográficas;
%Utilizar e compreender diferentes representações para a localização de objetos no plano, como mapas, células em planilhas eletrônicas e coordenadas geográficas, a fim de desenvolver as primeiras noções de coordenadas cartesianas.
Matemática}

\BNCC{EF05MA15} Identificar: descrever a localização e movimentação de objetos; coordenadas
%Interpretar, descrever e representar a localização ou movimentação de objetos no plano cartesiano (1º quadrante), utilizando coordenadas cartesianas, indicando mudanças de direção e de sentido e giros.
Matemática}

\BNCC{EF05MA16} Identificar: figuras especiais; prismas; pirâmides
%Associar figuras espaciais a suas planificações (prismas, pirâmides, cilindros e cones) e analisar, nomear e comparar seus atributos.
Matemática}

\BNCC{EF05MA17} Produzir: comparar e desenhar polígonos;
%Reconhecer, nomear e comparar polígonos, considerando lados, vértices e ângulos, e desenhá-los, utilizando material de desenho ou tecnologias digitais.
Matemática}

\BNCC{EF05MA18} Identificar: reconhecer ângulos; proporcionalidade de figuras;
%Reconhecer a congruência dos ângulos e a proporcionalidade entre os lados correspondentes de figuras poligonais em situações de ampliação e de redução em malhas quadriculadas e usando tecnologias digitais.
Matemática}

\BNCC{EF05MA19} Produzir: resolver problemas envolvendo medidas; 
%Resolver e elaborar problemas envolvendo medidas das grandezas comprimento, área, massa, tempo, temperatura e capacidade, recorrendo a transformações entre as unidades mais usuais em contextos socioculturais.
Matemática}

\BNCC{EF05MA20} Identificar: perceber que figuras de perímetros iguais podem ter áreas diferentes;
%Concluir, por meio de investigações, que figuras de perímetros iguais podem ter áreas diferentes e que, também, figuras que têm a mesma área podem ter perímetros diferentes.
Matemática}

\BNCC{EF05MA21} Identificar: reconhecer volume como grandeza associada a sólidos geométricos;
%Reconhecer volume como grandeza associada a sólidos geométricos e medir volumes por meio de empilhamento de cubos, utilizando, preferencialmente, objetos concretos.
Matemática}

\BNCC{EF05MA22} Produzir: apresentar resultados de experimentos;
%Apresentar todos os possíveis resultados de um experimento aleatório, estimando se esses resultados são igualmente prováveis ou não.
Matemática}

\BNCC{EF05MA23} Produzir: determinar probabilidade de eventos;
%Determinar a probabilidade de ocorrência de um resultado em eventos aleatórios, quando todos os resultados possíveis têm a mesma chance de ocorrer (equiprováveis).
Matemática}

\BNCC{EF05MA24} Identificar: interpretar dados de tabelas e gráficos;
%Interpretar dados estatísticos apresentados em textos, tabelas e gráficos (colunas ou linhas), referentes a outras áreas do conhecimento ou a outros contextos, como saúde e trânsito, e produzir textos com o objetivo de sintetizar conclusões.
Matemática}

\BNCC{EF05MA25} Produzir: pesquisar; usar variáveis categóricas; tabelas; gráficos;
%Realizar pesquisa envolvendo variáveis categóricas e numéricas, organizar dados coletados por meio de tabelas, gráficos de colunas, pictóricos e de linhas, com e sem uso de tecnologias digitais, e apresentar texto escrito sobre a finalidade da pesquisa e a síntese dos resultados.

\BNCC{EF01CI01}
%Comparar características de diferentes materiais presentes em objetos de uso cotidiano, discutindo sua origem, os modos como são descartados e como podem ser usados de forma mais consciente.
Ciências}

\BNCC{EF01CI02}
%Localizar, nomear e representar graficamente (por meio de desenhos) partes do corpo humano e explicar suas funções.
Ciências}

\BNCC{EF01CI03}
%Discutir as razões pelas quais os hábitos de higiene do corpo (lavar as mãos antes de comer, escovar os dentes, limpar os olhos, o nariz e as orelhas etc.) são necessários para a manutenção da saúde.
Ciências}

\BNCC{EF01CI04}
%Comparar características físicas entre os colegas, reconhecendo a diversidade e a importância da valorização, do acolhimento e do respeito às diferenças.
Ciências}

\BNCC{EF01CI05}
%Identificar e nomear diferentes escalas de tempo: os períodos diários (manhã, tarde, noite) e a sucessão de dias, semanas, meses e anos.
Ciências}

\BNCC{EF01CI06}
%Selecionar exemplos de como a sucessão de dias e noites orienta o ritmo de atividades diárias de seres humanos e de outros seres vivos.
Ciências}

\BNCC{EF02CI01}
%Identificar de que materiais (metais, madeira, vidro etc.) são feitos os objetos que fazem parte da vida cotidiana, como esses objetos são utilizados e com quais materiais eram produzidos no passado.
Ciências}

\BNCC{EF02CI02}
%Propor o uso de diferentes materiais para a construção de objetos de uso cotidiano, tendo em vista algumas propriedades desses materiais (flexibilidade, dureza, transparência etc.).
Ciências}

\BNCC{EF02CI03}
%Discutir os cuidados necessários à prevenção de acidentes domésticos (objetos cortantes e inflamáveis, eletricidade, produtos de limpeza, medicamentos etc.).
Ciências}

\BNCC{EF02CI04}
%Descrever características de plantas e animais (tamanho, forma, cor, fase da vida, local onde se desenvolvem etc.) que fazem parte de seu cotidiano e relacioná-las ao ambiente em que eles vivem.
Ciências}

\BNCC{EF02CI05}
%Investigar a importância da água e da luz para a manutenção da vida de plantas em geral.
Ciências}

\BNCC{EF02CI06}
%Identificar as principais partes de uma planta (raiz, caule, folhas, flores e frutos) e a função desempenhada por cada uma delas, e analisar as relações entre as plantas, o ambiente e os demais seres vivos.
Ciências}

\BNCC{EF02CI07}
%Descrever as posições do Sol em diversos horários do dia e associá-las ao tamanho da sombra projetada.
Ciências}

\BNCC{EF02CI08}
%Comparar o efeito da radiação solar (aquecimento e reflexão) em diferentes tipos de superfície (água, areia, solo, superfícies escura, clara e metálica etc.).
Ciências}

\BNCC{EF03CI01}
%Produzir diferentes sons a partir da vibração de variados objetos e identificar variáveis que influem nesse fenômeno.
Ciências}

\BNCC{EF03CI02}
%Experimentar e relatar o que ocorre com a passagem da luz através de objetos transparentes (copos, janelas de vidro, lentes, prismas, água etc.), no contato com superfícies polidas (espelhos) e na intersecção com objetos opacos (paredes, pratos, pessoas e outros objetos de uso cotidiano).
Ciências}

\BNCC{EF03CI03}
%Discutir hábitos necessários para a manutenção da saúde auditiva e visual considerando as condições do ambiente em termos de som e luz.
Ciências}

\BNCC{EF03CI04}
%Identificar características sobre o modo de vida (o que comem, como se reproduzem, como se deslocam etc.) dos animais mais comuns no ambiente próximo.
Ciências}

\BNCC{EF03CI05}
%Descrever e comunicar as alterações que ocorrem desde o nascimento em animais de diferentes meios terrestres ou aquáticos, inclusive o homem.
Ciências}

\BNCC{EF03CI06}
%Comparar alguns animais e organizar grupos com base em características externas comuns (presença de penas, pelos, escamas, bico, garras, antenas, patas etc.).
Ciências}

\BNCC{EF03CI07}
%Identificar características da Terra (como seu formato esférico, a presença de água, solo etc.), com base na observação, manipulação e comparação de diferentes formas de representação do planeta (mapas, globos, fotografias etc.).
Ciências}

\BNCC{EF03CI08}
%Observar, identificar e registrar os períodos diários (dia e/ou noite) em que o Sol, demais estrelas, Lua e planetas estão visíveis no céu.
Ciências}

\BNCC{EF03CI09}
%Comparar diferentes amostras de solo do entorno da escola com base em características como cor, textura, cheiro, tamanho das partículas, permeabilidade etc.
Ciências}

\BNCC{EF03CI10}
%Identificar os diferentes usos do solo (plantação e extração de materiais, dentre outras possibilidades), reconhecendo a importância do solo para a agricultura e para a vida.
Ciências}

\BNCC{EF04CI01}
%Identificar misturas na vida diária, com base em suas propriedades físicas observáveis, reconhecendo sua composição.
Ciências}

\BNCC{EF04CI02}
%Testar e relatar transformações nos materiais do dia a dia quando expostos a diferentes condições (aquecimento, resfriamento, luz e umidade).
Ciências}

\BNCC{EF04CI03}
%Concluir que algumas mudanças causadas por aquecimento ou resfriamento são reversíveis (como as mudanças de estado físico da água) e outras não (como o cozimento do ovo, a queima do papel etc.).
Ciências}

\BNCC{EF04CI04}
%Analisar e construir cadeias alimentares simples, reconhecendo a posição ocupada pelos seres vivos nessas cadeias e o papel do Sol como fonte primária de energia na produção de alimentos.
Ciências}

\BNCC{EF04CI05}
%Descrever e destacar semelhanças e diferenças entre o ciclo da matéria e o fluxo de energia entre os componentes vivos e não vivos de um ecossistema.
Ciências}

\BNCC{EF04CI06}
%Relacionar a participação de fungos e bactérias no processo de decomposição, reconhecendo a importância ambiental desse processo.
Ciências}

\BNCC{EF04CI07}
%Verificar a participação de microrganismos na produção de alimentos, combustíveis, medicamentos, entre outros.
Ciências}

\BNCC{EF04CI08}
%Propor, a partir do conhecimento das formas de transmissão de alguns microrganismos (vírus, bactérias e protozoários), atitudes e medidas adequadas para prevenção de doenças a eles associadas.
Ciências}

\BNCC{EF04CI09}
%Identificar os pontos cardeais, com base no registro de diferentes posições relativas do Sol e da sombra de uma vara (gnômon).
Ciências}

\BNCC{EF04CI10}
%Comparar as indicações dos pontos cardeais resultantes da observação das sombras de uma vara (gnômon) com aquelas obtidas por meio de uma bússola.
Ciências}

\BNCC{EF04CI11}
%Associar os movimentos cíclicos da Lua e da Terra a períodos de tempo regulares e ao uso desse conhecimento para a construção de calendários em diferentes culturas.
Ciências}

\BNCC{EF05CI01}
%Explorar fenômenos da vida cotidiana que evidenciem propriedades físicas dos materiais – como densidade, condutibilidade térmica e elétrica, respostas a forças magnéticas, solubilidade, respostas a forças mecânicas (dureza, elasticidade etc.), entre outras.
Ciências}

\BNCC{EF05CI02}
%Aplicar os conhecimentos sobre as mudanças de estado físico da água para explicar o ciclo hidrológico e analisar suas implicações na agricultura, no clima, na geração de energia elétrica, no provimento de água potável e no equilíbrio dos ecossistemas regionais (ou locais).
Ciências}

\BNCC{EF05CI03}
%Selecionar argumentos que justifiquem a importância da cobertura vegetal para a manutenção do ciclo da água, a conservação dos solos, dos cursos de água e da qualidade do ar atmosférico.
Ciências}

\BNCC{EF05CI04}
%Identificar os principais usos da água e de outros materiais nas atividades cotidianas para discutir e propor formas sustentáveis de utilização desses recursos.
Ciências}

\BNCC{EF05CI05}
%Construir propostas coletivas para um consumo mais consciente e criar soluções tecnológicas para o descarte adequado e a reutilização ou reciclagem de materiais consumidos na escola e/ou na vida cotidiana.
Ciências}

\BNCC{EF05CI06}
%Selecionar argumentos que justifiquem por que os sistemas digestório e respiratório são considerados corresponsáveis pelo processo de nutrição do organismo, com base na identificação das funções desses sistemas.
Ciências}

\BNCC{EF05CI07}
%Justificar a relação entre o funcionamento do sistema circulatório, a distribuição dos nutrientes pelo organismo e a eliminação dos resíduos produzidos.
Ciências}

\BNCC{EF05CI08}
%Organizar um cardápio equilibrado com base nas características dos grupos alimentares (nutrientes e calorias) e nas necessidades individuais (atividades realizadas, idade, sexo etc.) para a manutenção da saúde do organismo.
Ciências}

\BNCC{EF05CI09}
%Discutir a ocorrência de distúrbios nutricionais (como obesidade, subnutrição etc.) entre crianças e jovens a partir da análise de seus hábitos (tipos e quantidade de alimento ingerido, prática de atividade física etc.).
Ciências}

\BNCC{EF05CI10}
%Identificar algumas constelações no céu, com o apoio de recursos (como mapas celestes e aplicativos digitais, entre outros), e os períodos do ano em que elas são visíveis no início da noite.
Ciências}

\BNCC{EF05CI11}
%Associar o movimento diário do Sol e das demais estrelas no céu ao movimento de rotação da Terra.
Ciências}

\BNCC{EF05CI12}
%Concluir sobre a periodicidade das fases da Lua, com base na observação e no registro das formas aparentes da Lua no céu ao longo de, pelo menos, dois meses.
Ciências}

\BNCC{EF05CI13}
%Projetar e construir dispositivos para observação à distância (luneta, periscópio etc.), para observação ampliada de objetos (lupas, microscópios) ou para registro de imagens (máquinas fotográficas) e discutir usos sociais desses dispositivos.
Geografia}

\BNCC{EF01GE01}
%Descrever características observadas de seus lugares de vivência (moradia, escola etc.) e identificar semelhanças e diferenças entre esses lugares.
Geografia}

\BNCC{EF01GE02}
%Identificar semelhanças e diferenças entre jogos e brincadeiras de diferentes épocas e lugares.
Geografia}

\BNCC{EF01GE03}
%Identificar e relatar semelhanças e diferenças de usos do espaço público (praças, parques) para o lazer e diferentes manifestações.
Geografia}

\BNCC{EF01GE04}
%Discutir e elaborar, coletivamente, regras de convívio em diferentes espaços (sala de aula, escola etc.).
Geografia}

\BNCC{EF01GE05}
%Observar e descrever ritmos naturais (dia e noite, variação de temperatura e umidade etc.) em diferentes escalas espaciais e temporais, comparando a sua realidade com outras.
Geografia}

\BNCC{EF01GE06}
%Descrever e comparar diferentes tipos de moradia ou objetos de uso cotidiano (brinquedos, roupas, mobiliários), considerando técnicas e materiais utilizados em sua produção.
Geografia}

\BNCC{EF01GE07}
%Descrever atividades de trabalho relacionadas com o dia a dia da sua comunidade.
Geografia}

\BNCC{EF01GE08}
%Criar mapas mentais e desenhos com base em itinerários, contos literários, histórias inventadas e brincadeiras.
Geografia}

\BNCC{EF01GE09}
%Elaborar e utilizar mapas simples para localizar elementos do local de vivência, considerando referenciais espaciais (frente e atrás, esquerda e direita, em cima e embaixo, dentro e fora) e tendo o corpo como referência.
Geografia}

\BNCC{EF01GE10}
%Descrever características de seus lugares de vivência relacionadas aos ritmos da natureza (chuva, vento, calor etc.).
Geografia}

\BNCC{EF01GE11}
%Associar mudanças de vestuário e hábitos alimentares em sua comunidade ao longo do ano, decorrentes da variação de temperatura e umidade no ambiente.
Geografia}

\BNCC{EF02GE01}
%Descrever a história das migrações no bairro ou comunidade em que vive.
Geografia}

\BNCC{EF02GE02}
%Comparar costumes e tradições de diferentes populações inseridas no bairro ou comunidade em que vive, reconhecendo a importância do respeito às diferenças.
Geografia}

\BNCC{EF02GE03}
%Comparar diferentes meios de transporte e de comunicação, indicando o seu papel na conexão entre lugares, e discutir os riscos para a vida e para o ambiente e seu uso responsável.
Geografia}

\BNCC{EF02GE04}
%Reconhecer semelhanças e diferenças nos hábitos, nas relações com a natureza e no modo de viver de pessoas em diferentes lugares.
Geografia}

\BNCC{EF02GE05}
%Analisar mudanças e permanências, comparando imagens de um mesmo lugar em diferentes tempos.
Geografia}

\BNCC{EF02GE06}
%Relacionar o dia e a noite a diferentes tipos de atividades sociais (horário escolar, comercial, sono etc.).
Geografia}

\BNCC{EF02GE07}
%Descrever as atividades extrativas (minerais, agropecuárias e industriais) de diferentes lugares, identificando os impactos ambientais.
Geografia}

\BNCC{EF02GE08}
%Identificar e elaborar diferentes formas de representação (desenhos, mapas mentais, maquetes) para representar componentes da paisagem dos lugares de vivência.
Geografia}

\BNCC{EF02GE09}
%Identificar objetos e lugares de vivência (escola e moradia) em imagens aéreas e mapas (visão vertical) e fotografias (visão oblíqua).
Geografia}

\BNCC{EF02GE10}
%Aplicar princípios de localização e posição de objetos (referenciais espaciais, como frente e atrás, esquerda e direita, em cima e embaixo, dentro e fora) por meio de representações espaciais da sala de aula e da escola.
Geografia}

\BNCC{EF02GE11}
%Reconhecer a importância do solo e da água para a vida, identificando seus diferentes usos (plantação e extração de materiais, entre outras possibilidades) e os impactos desses usos no cotidiano da cidade e do campo.
Geografia}

\BNCC{EF03GE01}
%Identificar e comparar aspectos culturais dos grupos sociais de seus lugares de vivência, seja na cidade, seja no campo.
Geografia}

\BNCC{EF03GE02}
%Identificar, em seus lugares de vivência, marcas de contribuição cultural e econômica de grupos de diferentes origens.
Geografia}

\BNCC{EF03GE03}
%Reconhecer os diferentes modos de vida de povos e comunidades tradicionais em distintos lugares.
Geografia}

\BNCC{EF03GE04}
%Explicar como os processos naturais e históricos atuam na produção e na mudança das paisagens naturais e antrópicas nos seus lugares de vivência, comparando-os a outros lugares.
Geografia}

\BNCC{EF03GE05}
%Identificar alimentos, minerais e outros produtos cultivados e extraídos da natureza, comparando as atividades de trabalho em diferentes lugares.
Geografia}

\BNCC{EF03GE06}
%Identificar e interpretar imagens bidimensionais e tridimensionais em diferentes tipos de representação cartográfica.
Geografia}

\BNCC{EF03GE07}
%Reconhecer e elaborar legendas com símbolos de diversos tipos de representações em diferentes escalas cartográficas.
Geografia}

\BNCC{EF03GE08}
%Relacionar a produção de lixo doméstico ou da escola aos problemas causados pelo consumo excessivo e construir propostas para o consumo consciente, considerando a ampliação de hábitos de redução, reúso e reciclagem/ descarte de materiais consumidos em casa, na escola e/ou no entorno.
Geografia}

\BNCC{EF03GE09}
%Investigar os usos dos recursos naturais, com destaque para os usos da água em atividades cotidianas (alimentação, higiene, cultivo de plantas etc.), e discutir os problemas ambientais provocados por esses usos.
Geografia}

\BNCC{EF03GE10}
%Identificar os cuidados necessários para utilização da água na agricultura e na geração de energia de modo a garantir a manutenção do provimento de água potável.
Geografia}

\BNCC{EF03GE11}
%Comparar impactos das atividades econômicas urbanas e rurais sobre o ambiente físico natural, assim como os riscos provenientes do uso de ferramentas e máquinas.
Geografia}

\BNCC{EF04GE01}
%Selecionar, em seus lugares de vivência e em suas histórias familiares e/ou da comunidade, elementos de distintas culturas (indígenas, afro-brasileiras, de outras regiões do país, latino-americanas, europeias, asiáticas etc.), valorizando o que é próprio em cada uma delas e sua contribuição para a formação da cultura local, regional e brasileira.
Geografia}

\BNCC{EF04GE02}
%Descrever processos migratórios e suas contribuições para a formação da sociedade brasileira.
Geografia}

\BNCC{EF04GE03}
%Distinguir funções e papéis dos órgãos do poder público municipal e canais de participação social na gestão do Município, incluindo a Câmara de Vereadores e Conselhos Municipais.
Geografia}

\BNCC{EF04GE04}
%Reconhecer especificidades e analisar a interdependência do campo e da cidade, considerando fluxos econômicos, de informações, de ideias e de pessoas.
Geografia}

\BNCC{EF04GE05}
%Distinguir unidades político-administrativas oficiais nacionais (Distrito, Município, Unidade da Federação e grande região), suas fronteiras e sua hierarquia, localizando seus lugares de vivência.
Geografia}

\BNCC{EF04GE06}
%Identificar e descrever territórios étnico-culturais existentes no Brasil, tais como terras indígenas e de comunidades remanescentes de quilombos, reconhecendo a legitimidade da demarcação desses territórios.
Geografia}

\BNCC{EF04GE07}
%Comparar as características do trabalho no campo e na cidade.
Geografia}

\BNCC{EF04GE08}
%Descrever e discutir o processo de produção (transformação de matérias-primas), circulação e consumo de diferentes produtos.
Geografia}

\BNCC{EF04GE09}
%Utilizar as direções cardeais na localização de componentes físicos e humanos nas paisagens rurais e urbanas.
Geografia}

\BNCC{EF04GE10}
%Comparar tipos variados de mapas, identificando suas características, elaboradores, finalidades, diferenças e semelhanças.
Geografia}

\BNCC{EF04GE11}
%Identificar as características das paisagens naturais e antrópicas (relevo, cobertura vegetal, rios etc.) no ambiente em que vive, bem como a ação humana na conservação ou degradação dessas áreas.
Geografia}

\BNCC{EF05GE01}
%Descrever e analisar dinâmicas populacionais na Unidade da Federação em que vive, estabelecendo relações entre migrações e condições de infraestrutura.
Geografia}

\BNCC{EF05GE02}
%Identificar diferenças étnico-raciais e étnico-culturais e desigualdades sociais entre grupos em diferentes territórios.
Geografia}

\BNCC{EF05GE03}
%Identificar as formas e funções das cidades e analisar as mudanças sociais, econômicas e ambientais provocadas pelo seu crescimento.
Geografia}

\BNCC{EF05GE04}
%Reconhecer as características da cidade e analisar as interações entre a cidade e o campo e entre cidades na rede urbana.
Geografia}

\BNCC{EF05GE05}
%Identificar e comparar as mudanças dos tipos de trabalho e desenvolvimento tecnológico na agropecuária, na indústria, no comércio e nos serviços.
Geografia}

\BNCC{EF05GE06}
%Identificar e comparar transformações dos meios de transporte e de comunicação.
Geografia}

\BNCC{EF05GE07}
%Identificar os diferentes tipos de energia utilizados na produção industrial, agrícola e extrativa e no cotidiano das populações.
Geografia}

\BNCC{EF05GE08}
%Analisar transformações de paisagens nas cidades, comparando sequência de fotografias, fotografias aéreas e imagens de satélite de épocas diferentes.
Geografia}

\BNCC{EF05GE09}
%Estabelecer conexões e hierarquias entre diferentes cidades, utilizando mapas temáticos e representações gráficas.
Geografia}

\BNCC{EF05GE10}
%Reconhecer e comparar atributos da qualidade ambiental e algumas formas de poluição dos cursos de água e dos oceanos (esgotos, efluentes industriais, marés negras etc.).
Geografia}

\BNCC{EF05GE11}
%Identificar e descrever problemas ambientais que ocorrem no entorno da escola e da residência (lixões, indústrias poluentes, destruição do patrimônio histórico etc.), propondo soluções (inclusive tecnológicas) para esses problemas.
Geografia}

\BNCC{EF05GE12}
%Identificar órgãos do poder público e canais de participação social responsáveis por buscar soluções para a melhoria da qualidade de vida (em áreas como meio ambiente, mobilidade, moradia e direito à cidade) e discutir as propostas implementadas por esses órgãos que afetam a comunidade em que vive.

\BNCC{EF01HI01}
%Identificar aspectos do seu crescimento por meio do registro das lembranças particulares ou de lembranças dos membros de sua família e/ou de sua comunidade.

\BNCC{EF01HI02}
%Identificar a relação entre as suas histórias e as histórias de sua família e de sua comunidade.

\BNCC{EF01HI03}
%Descrever e distinguir os seus papéis e responsabilidades relacionados à família, à escola e à comunidade.

\BNCC{EF01HI04}
%Identificar as diferenças entre os variados ambientes em que vive (doméstico, escolar e da comunidade), reconhecendo as especificidades dos hábitos e das regras que os regem.

\BNCC{EF01HI05}
%Identificar semelhanças e diferenças entre jogos e brincadeiras atuais e de outras épocas e lugares.

\BNCC{EF01HI06}
%Conhecer as histórias da família e da escola e identificar o papel desempenhado por diferentes sujeitos em diferentes espaços.

\BNCC{EF01HI07}
%Identificar mudanças e permanências nas formas de organização familiar.

\BNCC{EF01HI08}
%Reconhecer o significado das comemorações e festas escolares, diferenciando-as das datas festivas comemoradas no âmbito familiar ou da comunidade.

\BNCC{EF02HI01}
%Reconhecer espaços de sociabilidade e identificar os motivos que aproximam e separam as pessoas em diferentes grupos sociais ou de parentesco.

\BNCC{EF02HI02}
%Identificar e descrever práticas e papéis sociais que as pessoas exercem em diferentes comunidades.

\BNCC{EF02HI03}
%Selecionar situações cotidianas que remetam à percepção de mudança, pertencimento e memória.

\BNCC{EF02HI04}
%Selecionar e compreender o significado de objetos e documentos pessoais como fontes de memórias e histórias nos âmbitos pessoal, familiar, escolar e comunitário.

\BNCC{EF02HI05}
%Selecionar objetos e documentos pessoais e de grupos próximos ao seu convívio e compreender sua função, seu uso e seu significado.

\BNCC{EF02HI06}
%Identificar e organizar, temporalmente, fatos da vida cotidiana, usando noções relacionadas ao tempo (antes, durante, ao mesmo tempo e depois).

\BNCC{EF02HI07}
%Identificar e utilizar diferentes marcadores do tempo presentes na comunidade, como relógio e calendário.

\BNCC{EF02HI08}
%Compilar histórias da família e/ou da comunidade registradas em diferentes fontes.

\BNCC{EF02HI09}
%Identificar objetos e documentos pessoais que remetam à própria experiência no âmbito da família e/ou da comunidade, discutindo as razões pelas quais alguns objetos são preservados e outros são descartados.

\BNCC{EF02HI10}
%Identificar diferentes formas de trabalho existentes na comunidade em que vive, seus significados, suas especificidades e importância.

\BNCC{EF02HI11}
%Identificar impactos no ambiente causados pelas diferentes formas de trabalho existentes na comunidade em que vive.

\BNCC{EF03HI01}
%Identificar os grupos populacionais que formam a cidade, o município e a região, as relações estabelecidas entre eles e os eventos que marcam a formação da cidade, como fenômenos migratórios (vida rural/vida urbana), desmatamentos, estabelecimento de grandes empresas etc.

\BNCC{EF03HI02}
%Selecionar, por meio da consulta de fontes de diferentes naturezas, e registrar acontecimentos ocorridos ao longo do tempo na cidade ou região em que vive.

\BNCC{EF03HI03} Identificar +Comparar: "pontos de vista de eventos"
%Identificar e comparar pontos de vista em relação a eventos significativos do local em que vive, aspectos relacionados a condições sociais e à presença de diferentes grupos sociais e culturais, com especial destaque para as culturas africanas, indígenas e de migrantes.

\BNCC{EF03HI04} Identificar +Perceber: "razões de constituição dos patrimônios históricos"
%Identificar os patrimônios históricos e culturais de sua cidade ou região e discutir as razões culturais, sociais e políticas para que assim sejam considerados.

\BNCC{EF03HI05} Identificar +Perceber: "marcos históricos e seus significados"
%Identificar os marcos históricos do lugar em que vive e compreender seus significados.

\BNCC{EF03HI06} Identificar +Perceber: "critérios para escolha dos nomes dos registros de memória na cidade"
%Identificar os registros de memória na cidade (nomes de ruas, monumentos, edifícios etc.), discutindo os critérios que explicam a escolha desses nomes.

\BNCC{EF03HI07} Identificar +Diferenciar: "comunidades da cidade ou região e papel dos diferentes grupos sociais"
%Identificar semelhanças e diferenças existentes entre comunidades de sua cidade ou região, e descrever o papel dos diferentes grupos sociais que as formam.

\BNCC{EF03HI08} Identificar +Comparar: "modos de vida na cidade e campo"; "comparar com passado"
%Identificar modos de vida na cidade e no campo no presente, comparando-os com os do passado.

\BNCC{EF03HI09} Produzir +Mapear: "espaços públicos e suas funções"; "ruas, praças, escolas, etc"
%Mapear os espaços públicos no lugar em que vive (ruas, praças, escolas, hospitais, prédios da Prefeitura e da Câmara de Vereadores etc.) e identificar suas funções.

\BNCC{EF03HI10} Identificar +Diferenciar: "espaços domésticos, públicos e conservação ambiental"
%Identificar as diferenças entre o espaço doméstico, os espaços públicos e as áreas de conservação ambiental, compreendendo a importância dessa distinção.

\BNCC{EF03HI11} Identificar +Diferenciar: "formas de trabalho da cidade e campo"; "uso da tecnologia"
%Identificar diferenças entre formas de trabalho realizadas na cidade e no campo, considerando também o uso da tecnologia nesses diferentes contextos.

\BNCC{EF03HI12} Identificar +Comparar: "relações de trabalho e lazer com outros tempos e espaços"
%Comparar as relações de trabalho e lazer do presente com as de outros tempos e espaços, analisando mudanças e permanências.

\BNCC{EF04HI01} Identificar +Reconhecer: "história como ação humana no tempo e no espaço"
%Reconhecer a história como resultado da ação do ser humano no tempo e no espaço, com base na identificação de mudanças e permanências ao longo do tempo.

\BNCC{EF04HI02} Identificar +Observar: "significado dos marcos da história da humanidade"; "nomadismo, agricultura, etc"
%Identificar mudanças e permanências ao longo do tempo, discutindo os sentidos dos grandes marcos da história da humanidade (nomadismo, desenvolvimento da agricultura e do pastoreio, criação da indústria etc.).

\BNCC{EF04HI03} Identificar +Relacionar: "transformações na cidade e as interferências na vida dos habitantes"
%Identificar as transformações ocorridas na cidade ao longo do tempo e discutir suas interferências nos modos de vida de seus habitantes, tomando como ponto de partida o presente.

\BNCC{EF04HI04} Identificar +Relacionar: "indivíduos e a natureza"; "nomadismo nas primeiras comunidades humanas"
%Identificar as relações entre os indivíduos e a natureza e discutir o significado do nomadismo e da fixação das primeiras comunidades humanas.

\BNCC{EF04HI05} Identificar +Relacionar: "resultado da ocupação do campo na natureza"
%Relacionar os processos de ocupação do campo a intervenções na natureza, avaliando os resultados dessas intervenções.

\BNCC{EF04HI06} Identificar +Relacionar: "adaptação ou marginalização no deslocamento de pessoas e mercadorias"
%Identificar as transformações ocorridas nos processos de deslocamento das pessoas e mercadorias, analisando as formas de adaptação ou marginalização.

\BNCC{EF04HI07} Identificar +Escrever: "caminhos terrestres, fluviais a marítimos no comércio"
%Identificar e descrever a importância dos caminhos terrestres, fluviais e marítimos para a dinâmica da vida comercial.

\BNCC{EF04HI08} Identificar +Relacionar: "mudanças na comunicação e os diferentes grupos sociais"
%Identificar as transformações ocorridas nos meios de comunicação (cultura oral, imprensa, rádio, televisão, cinema, internet e demais tecnologias digitais de informação e comunicação) e discutir seus significados para os diferentes grupos ou estratos sociais.

\BNCC{EF04HI09} Identificar +Relacionar: "avaliar papel de migrações em diferentes tempos e espaços"
%Identificar as motivações dos processos migratórios em diferentes tempos e espaços e avaliar o papel desempenhado pela migração nas regiões de destino.

\BNCC{EF04HI10} Identificar +Relacionar: "fluxos populacionais e sociedade brasileira"
%Analisar diferentes fluxos populacionais e suas contribuições para a formação da sociedade brasileira.

\BNCC{EF04HI11} Identificar +Observar: "mudanças associadas à migração"
%Analisar, na sociedade em que vive, a existência ou não de mudanças associadas à migração (interna e internacional).

\BNCC{EF05HI01} Identificar +Relacionar: "cultura, povo e o espaço geográfico ocupado"
%Identificar os processos de formação das culturas e dos povos, relacionando-os com o espaço geográfico ocupado.

\BNCC{EF05HI02} Identificar +Relacionar: "organização do poder e formas de ordenação social"
%Identificar os mecanismos de organização do poder político com vistas à compreensão da ideia de Estado e/ou de outras formas de ordenação social.

\BNCC{EF05HI03} Identificar +Observar: "cultura e religião na identidade de povos antigos"
%Analisar o papel das culturas e das religiões na composição identitária dos povos antigos.

\BNCC{EF05HI04} Identificar +Relacionar: "cidadania, diversidade e direitos humanos"
%Associar a noção de cidadania com os princípios de respeito à diversidade, à pluralidade e aos direitos humanos.

\BNCC{EF05HI05} Identificar +Relacionar: "cidadania, direitos dos povos e das sociedades"
%Associar o conceito de cidadania à conquista de direitos dos povos e das sociedades, compreendendo-o como conquista histórica.

\BNCC{EF05HI06} Identificar +Comparar: "significados sociais, políticos e culturais em linguagens e tecnologias"
%Comparar o uso de diferentes linguagens e tecnologias no processo de comunicação e avaliar os significados sociais, políticos e culturais atribuídos a elas.

\BNCC{EF05HI07} Identificar +Relacionar: "presença e/ou a ausência de grupos na produção dos marcos de memória"
%Identificar os processos de produção, hierarquização e difusão dos marcos de memória e discutir a presença e/ou a ausência de diferentes grupos que compõem a sociedade na nomeação desses marcos de memória.

\BNCC{EF05HI08} Identificar +Relacionar: "formas de marcação da passagem do tempo em distintas sociedades"
%Identificar formas de marcação da passagem do tempo em distintas sociedades, incluindo os povos indígenas originários e os povos africanos.

\BNCC{EF05HI09} Identificar +Comparar: "visões sobre temas que impactam o cotidiano"; "diferentes fontes, incluindo orais"
%Comparar pontos de vista sobre temas que impactam a vida cotidiana no tempo presente, por meio do acesso a diferentes fontes, incluindo orais.

\BNCC{EF05HI10} Produzir +Registrar: "inventariar patrimônios materiais e imateriais"; "mudanças e permanências dos patrimônios"
%Inventariar os patrimônios materiais e imateriais da humanidade e analisar mudanças e permanências desses patrimônios ao longo do tempo.

\BNCC{EF01ER01} Identificar +Perceber: "semelhanças e diferenças"; "religião"
%Identificar e acolher as semelhanças e diferenças entre o eu, o outro e o nós.

\BNCC{EF01ER02} Identificar +Reconhecer: "identificação"; "religião"
%Reconhecer que o seu nome e o das demais pessoas os identificam e os diferenciam.

\BNCC{EF01ER03} Identificar +Reconhecer: "respeito das características físicas e subjetivas"; "religião"
%Reconhecer e respeitar as características físicas e subjetivas de cada um.

\BNCC{EF01ER04} Identificar +Relacionar: "diversidade de formas de vida"; "religião"
%Valorizar a diversidade de formas de vida.

\BNCC{EF01ER05} Identificar +Reconhecer: "acolher sentimentos, lembranças, memórias e saberes"; "religião"
%Identificar e acolher sentimentos, lembranças, memórias e saberes de cada um.

\BNCC{EF01ER06} Identificar +Observar: "sentimentos, ideias e crenças em diferentes espaços"; "religião"
%Identificar as diferentes formas pelas quais as pessoas manifestam sentimentos, ideias, memórias, gostos e crenças em diferentes espaços.

\BNCC{EF02ER01} Identificar +Reconhecer: "diferentes espaços de convivência"; "religião"
%Reconhecer os diferentes espaços de convivência.

\BNCC{EF02ER02} Identificar: "costumes, crenças e formas diversas de viver"; "religião"
%Identificar costumes, crenças e formas diversas de viver em variados ambientes de convivência.

\BNCC{EF02ER03} Produzir +Registrar: "memórias pessoais, familiares e escolares (fotos, músicas, narrativas, álbuns)"
%Identificar as diferentes formas de registro das memórias pessoais, familiares e escolares (fotos, músicas, narrativas, álbuns...).

\BNCC{EF02ER04} Identificar: "símbolos religiosos em espaços de convivência"
%Identificar os símbolos presentes nos variados espaços de convivência.

\BNCC{EF02ER05} Identificar: "distinguir e respeitar símbolos religiosos diferentes"
%Identificar, distinguir e respeitar símbolos religiosos de distintas manifestações, tradições e instituições religiosas.

\BNCC{EF02ER06} Identificar +Reconhecer: "alimentos sagrados para diferentes religiões"
%Exemplificar alimentos considerados sagrados por diferentes culturas, tradições e expressões religiosas.

\BNCC{EF02ER07} Identificar +Reconhecer: "significados atribuídos a alimentos em religiões"
%Identificar significados atribuídos a alimentos em diferentes manifestações e tradições religiosas.

\BNCC{EF03ER01} Identificar +Reconhecer: "respeitar territórios religiosos"
%Identificar e respeitar os diferentes espaços e territórios religiosos de diferentes tradições e movimentos religiosos.

\BNCC{EF03ER02} Identificar +Reproduzir: "caracterizar territórios religiosos"
%Caracterizar os espaços e territórios religiosos como locais de realização das práticas celebrativas.

\BNCC{EF03ER03} Identificar +Reconhecer: "respeitar práticas religiosas"
%Identificar e respeitar práticas celebrativas (cerimônias, orações, festividades, peregrinações, entre outras) de diferentes tradições religiosas.

\BNCC{EF03ER04} Identificar +Reproduzir: "caracterizar territórios religiosos"
%Caracterizar as práticas celebrativas como parte integrante do conjunto das manifestações religiosas de diferentes culturas e sociedades.

\BNCC{EF03ER05} Identificar +Reconhecer: "roupas usadas em diferentes religiões"
%Reconhecer as indumentárias (roupas, acessórios, símbolos, pinturas corporais) utilizadas em diferentes manifestações e tradições religiosas.

\BNCC{EF03ER06} Identificar +Reproduzir: "roupas usadas em diferentes religiões"
%Caracterizar as indumentárias como elementos integrantes das identidades religiosas.

\BNCC{EF04ER01} Identificar +Reconhecer: "ritos religiosos presentes no cotidiano"
%Identificar ritos presentes no cotidiano pessoal, familiar, escolar e comunitário.

\BNCC{EF04ER02} Identificar +Reconhecer: "funções dos ritos religiosos em diferentes tradições"
%Identificar ritos e suas funções em diferentes manifestações e tradições religiosas.

\BNCC{EF04ER03} Identificar +Reproduzir: "ritos religiosos de passagem"
%Caracterizar ritos de iniciação e de passagem em diversos grupos religiosos (nascimento, casamento e morte).

\BNCC{EF04ER04} Identificar: "orações, cultos, gestos, cantos, dança, meditação em diferentes religiões"
%Identificar as diversas formas de expressão da espiritualidade (orações, cultos, gestos, cantos, dança, meditação) nas diferentes tradições religiosas.

\BNCC{EF04ER05} Identificar +Relacionar: "religião em pinturas, arquitetura, esculturas, ícones, símbolos, imagens"
%Identificar representações religiosas em diferentes expressões artísticas (pinturas, arquitetura, esculturas, ícones, símbolos, imagens), reconhecendo-as como parte da identidade de diferentes culturas e tradições religiosas.

\BNCC{EF04ER06} Identificar: "nomes, significados e representações de divindades"
%Identificar nomes, significados e representações de divindades nos contextos familiar e comunitário.

\BNCC{EF04ER07} Identificar +Reconhecer: "divindades de diferentes religiões"
%Reconhecer e respeitar as ideias de divindades de diferentes manifestações e tradições religiosas.

\BNCC{EF05ER01} Identificar: "acontecimentos sagrados de diferentes religiões"
%Identificar e respeitar acontecimentos sagrados de diferentes culturas e tradições religiosas como recurso para preservar a memória.

\BNCC{EF05ER02} Identificar: "mitos de criação em diferentes religiões"
%Identificar mitos de criação em diferentes culturas e tradições religiosas.

\BNCC{EF05ER03} Identificar: "concepções de mundo, natureza, ser humano, divindades, vida e morte em diferentes mitos de criação"; "religiões"
%Reconhecer funções e mensagens religiosas contidas nos mitos de criação (concepções de mundo, natureza, ser humano, divindades, vida e morte).

\BNCC{EF05ER04} Identificar: "tradição oral na religião"
%Reconhecer a importância da tradição oral para preservar memórias e acontecimentos religiosos.

\BNCC{EF05ER05} Identificar: "tradição oral nas culturas indígenas, afro-brasileiras, ciganas"; "religião"
%Identificar elementos da tradição oral nas culturas e religiosidades indígenas, afro-brasileiras, ciganas, entre outras.

\BNCC{EF05ER06} Identificar: "sábios e a tradição oral"; "religiões"
%Identificar o papel dos sábios e anciãos na comunicação e preservação da tradição oral.

\BNCC{EF05ER07} Identificar: "tradição oral e ética"; "religiões"
%Reconhecer, em textos orais, ensinamentos relacionados a modos de ser e viver.
